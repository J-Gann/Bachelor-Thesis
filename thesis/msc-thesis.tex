% This template was initially provided by Dulip Withanage.
% Modifications for the database systems research group
% were made by Conny Junghans,  Jannik Strötgen and Michael Gertz

\documentclass[
     12pt,         % font size
     a4paper,      % paper format
     BCOR10mm,     % binding correction
     DIV14,        % stripe size for margin calculation
%     liststotoc,   % table listing in toc
%     bibtotoc,     % bibliography in toc
%     idxtotoc,     % index in toc
%     parskip       % paragraph skip instad of paragraph indent
     ]{scrreprt}

%%%%%%%%%%%%%%%%%%%%%%%%%%%%%%%%%%%%%%%%%%%%%%%%%%%%%%%%%%%%

% PACKAGES:

% Use German :
\usepackage[english]{babel}
% Input and font encoding
\usepackage[latin1]{inputenc}
\usepackage[T1]{fontenc}
% Index-generation
\usepackage{makeidx}
% Einbinden von URLs:
\usepackage{url}
% Special \LaTex symbols (e.g. \BibTeX):
%\usepackage{doc}
% Include Graphic-files:
\usepackage{graphicx}
% Include doc++ generated tex-files:
%\usepackage{docxx}
% Include PDF links
%\usepackage[pdftex, bookmarks=true]{hyperref}

% Fuer anderthalbzeiligen Textsatz
\usepackage{setspace}

% hyperrefs in the documents
\usepackage[bookmarks=true,colorlinks,pdfpagelabels,pdfstartview = FitH,bookmarksopen = true,bookmarksnumbered = true,linkcolor = black,plainpages = false,hypertexnames = false,citecolor = black,urlcolor=black]{hyperref} 
%\usepackage{hyperref}

%%%%%%%%%%%%%%%%%%%%%%%%%%%%%%%%%%%%%%%%%%%%%%%%%%%%%%%%%%%%

% OTHER SETTINGS:

% Pagestyle:
\pagestyle{headings}

% Choose language
\newcommand{\setlang}[1]{\selectlanguage{#1}\nonfrenchspacing}


\begin{document}

% TITLE:
\pagenumbering{roman}
\begin{titlepage}
     \vspace*{1cm}
     \begin{center}
          \vspace*{3cm}
          \textbf
          {
               \Large University of Heidelberg\\
               \smallskip
               \Large Institute for Computer Science\\
               \smallskip
               \Large Working group database systems\\
               \smallskip
          }

          \vspace{3cm}

          \textbf{\large Bachelor thesis}

          \vspace{0.5\baselineskip}
          {
               \huge
               \textbf{Messaging Architecture for Integration of Customer Self-Services}
          }

     \end{center}

     \vfill
     {
          \large
          \begin{tabular}[l]{ll}
               Name:                 & Jonas Gann              \\
               Matriculation number: & 3367576                 \\
               Supervisor:           & Prof. Dr. Michael Gertz \\
               Date of submission:   & \today
          \end{tabular}
     }

\end{titlepage}

\onehalfspacing

\thispagestyle{empty}

\vspace*{100pt}
\noindent
I assure that I have written this bachelor thesis on my own and only used the specified sources and
resources and that I followed the principles and recommendations "Responsibility in Science" of the
University of Heidelberg.

\vspace*{50pt}
\noindent

\underline{\phantom{mmmmmmmmmmmmmmmmmmmm}}

\medskip
\noindent
Date of Submission: \today
\newpage

\chapter*{Zusammenfassung}

\newpage

\chapter*{Abstract}

\newpage

% MAIN PART:
% Table of contents (Inhaltsverzeichnis)
\tableofcontents
\cleardoublepage
\pagenumbering{arabic}

% List of figures (Abbildungsverzeichnis):
%\listoffigures
% List of tables (Tabellenverzeichnis):
%\listoftables

%%%%%%%%%%%%%%%%%%%%%%%%%%%%%%%%%%%%%%%%%%%%%%%%%%%%%%%%%%%%%%%

\chapter{Introduction}\label{intro}

\section{Motivation}

\subsection{Example(s) of CSS Problems Today}
\begin{enumerate}
     \item CSS incomplete in governmental administration
           \begin{enumerate}
                \item identity card renewal
                \item change of home address
                \item change of residency status
           \end{enumerate}

     \item CSS incomplete in universities
           \begin{enumerate}
                \item transfer of profile: grades, certificates, etc. between universities
                \item applications in general (study place, semester abroad, ...)
           \end{enumerate}

     \item CSS different for each enterprise / organization
           \begin{enumerate}
                \item multiple identities
                \item multiple profiles => addresses, phone numbers, mail address
                \item time consuming to manage same information in all systems
           \end{enumerate}
\end{enumerate}

\subsection{Solution for CSS Problems}
\begin{enumerate}
     \item One CSS providing company, which integrates multiple / all EA
     \item One identity, profile, address, phone number, location, ...
     \item Easy management of identity used by multiple enterprises / organizations
\end{enumerate}

\subsection{Integration necessity}
\begin{enumerate}
     \item Each enterprise / organization has its own identities, profiles, etc. in multiple systems
     \item The CSS providing company needs to manage those systems / data of each enterprise / organization
     \item Enterprise Integration solves this Problem
\end{enumerate}

\section{Goals of the work}

\subsection{Integration Challenges}
\begin{enumerate}
     \item Heterogeneous Enterprise Architecture Systems
           \begin{enumerate}
                \item Different Applications
                \item Different Application Vendors
                \item Different / No Application Interfaces
                \item Legacy Systems
           \end{enumerate}

     \item Different (proprietary) data models within and between Enterprises and Organizations
           \begin{enumerate}
                \item Different (property)name for same data objects (syntactic integration)
                \item Different meanings for same (property)name (semantic integration)
           \end{enumerate}

     \item Stability of Integration
           \begin{enumerate}
                \item Future Changes of EA
                \item Scalability
                \item Failure of EA or Integration Components
           \end{enumerate}

     \item Scarce Resources
           \begin{enumerate}
                \item Integration Development Speed
                      \begin{enumerate}
                           \item Necessary Development <=> Reuse of existing Technology
                           \item Complexity / Size of Integration
                      \end{enumerate}

                \item Maintenance of finished Integration

                \item Hardware / Software Costs of Integration
                      \begin{enumerate}
                           \item Licenses for Software
                           \item Scalability of Integration => Necessary Computing Power
                      \end{enumerate}

           \end{enumerate}

\end{enumerate}

\subsection{Presented Solutions}
\begin{enumerate}
     \item Messaging Integration
           \begin{enumerate}
                \item Loose Coupling
                      \begin{enumerate}
                           \item Loose Coupling simplifies adaption to changing EA => simpler Maintenance
                           \item Loose Coupling simplifies integration of new EA systems => integration of heterogeneous EA
                           \item Loose Coupling allows Reuse of "Modules" => faster development
                      \end{enumerate}

                \item Messaging enables communication with many systems through Adapters

                \item Messaging provides mechanisms for Stability
                      \begin{enumerate}
                           \item Store-and-Forward
                           \item Load Balancing
                      \end{enumerate}

           \end{enumerate}

     \item Integration Patterns
           \begin{enumerate}
                \item Patterns speed up construction of Integration Architecture
                \item Patterns are proven solutions
                \item Patterns abstract from concrete technologies
                      \begin{enumerate}
                           \item Simplifies understanding of integration concept
                           \item Allows implementation with different technologies
                      \end{enumerate}

           \end{enumerate}

\end{enumerate}


\section{Structure of the work}

%%%%%%%%%%%%%%%%%%%%%%%%%%%%%%%%%%%%%%%%%%%%%%%%%%%%%%%%%%%%
\newpage

\chapter{Foundation and related work}

\section{Customer Self-Service}

\subsection{Definition}
\begin{enumerate}
     \item Definition in the context of the thesis
\end{enumerate}

\subsection{Documentation}
\begin{enumerate}
     \item How can CSS-Scenarios be documented?
\end{enumerate}

\subsection{List of Scenarios}

\subsection{Purpose in Thesis}
\begin{enumerate}
     \item Finding relevant EAPs
     \item Requirement analysis of business Connector
     \item Interface definition of business connector
\end{enumerate}


\section{Architecture Patterns}
\subsection{Definition}
\begin{enumerate}
     \item Definition
     \item Metadata
     \item Views
     \item Architecture bricks
     \item Data bricks
     \item Business processes
\end{enumerate}

\subsection{Purpose in Thesis}
\begin{enumerate}
     \item Basis for real-life enterprise architectures
     \item Provides architecture and data bricks the integration can build up on
\end{enumerate}

\subsection{Architecture- and Data bricks}
\begin{enumerate}
     \item Description of EAPs used in the thesis
     \item Reasons why each EAP is relevant for CSS
           \begin{enumerate}
                \item Derived by comparing CSS-Scenarios with business processes
           \end{enumerate}
     \item Architecture bricks and data bricks of EAPs relevant for CSS-Scenarios
     \item Documentation of resulting Architecture and data bricks
\end{enumerate}

\subsection{Requirements of Enterprise Architecture}
\begin{enumerate}
     \item List of requirements towards the integration Architecture
           \begin{enumerate}
                \item Adapters for systems
                \item Required Format of incoming data
           \end{enumerate}
\end{enumerate}

\section{Business Connector}

\subsection{Definition}
\begin{enumerate}
     \item Provides functionality of "external" service provider
     \item Purpose is to simplify integration
     \item Can be assumed to provide sufficient interfaces
\end{enumerate}

\subsection{Purpose in Thesis}
\begin{enumerate}
     \item Provides interface for CSS functionalities
     \item Should be integrated in architecture and data bricks
\end{enumerate}

\subsection{Documentation}
\begin{enumerate}
     \item How to document the business Connector in this context

     \item List of provided functionalities along with their interfaces
           \begin{enumerate}
                \item Derived from CSS-Scenarios
                \item Derived from IDAS Connector
           \end{enumerate}

     \item List of requirements regarding the EA
           \begin{enumerate}
                \item Access to architecture and data bricks
           \end{enumerate}

     \item List of requirements regarding the integration Architecture
           \begin{enumerate}
                \item Format of data
                \item Order of delivery
                \item Time relevance of data
                \item Guaranteed delivery of data
           \end{enumerate}

\end{enumerate}

\section{Integration Patterns}

\subsection{Definition}
\begin{enumerate}
     \item Messaging Methodology
     \item Patterns describe often solved messaging problems
\end{enumerate}

\subsection{Purpose in Thesis}
\begin{enumerate}
     \item Already explained in Motivation?
\end{enumerate}

\subsection{Patterns used in Thesis}
\begin{enumerate}
     \item Description of Patterns later used in the integration architecture
\end{enumerate}

%%%%%%%%%%%%%%%%%%%%%%%%%%%%%%%%%%%%%%%%%%%%%%%%%%%%%%%%%%%%
\newpage

\chapter{My contribution}

\section{Overview}

\section{Requirements}

\subsection{Requirements of Enterprise Architecture}

\subsection{Requirements of Business Connector}

\subsection{Requirements of Integration System}

\section{Scenario Integrations}
\begin{enumerate}
     \item Look at integration architecture for each CSS-Scenario
\end{enumerate}

\subsection{Integration Architecture Documentation Method}
\begin{enumerate}
     \item Entity Diagram(s)
     \item Flow Diagram(s)
     \item etc.
\end{enumerate}

\subsection{Scenario1}
\begin{enumerate}
     \item Description of Scenario
\end{enumerate}

\subsubsection{Scenario1 Integration Architecture}
\begin{enumerate}
     \item Entity Diagram(s)
     \item Flow Diagram(s)
     \item etc.
\end{enumerate}

\subsubsection{System Integration}
\begin{enumerate}
     \item Explanation how the systems are integrated
     \item Why was each pattern used?
     \item How are requirements met?
\end{enumerate}

\subsubsection{Data Integration}
\begin{enumerate}
     \item Explanation how the data objects are integrated
     \item Why was data transformed / mapped?
     \item How are requirements met
\end{enumerate}

\subsection{Scenario2}

\subsubsection{Scenario2 Integration Architecture}

\subsubsection{System Integration}

\subsubsection{Data Integration}

\section{Operating Manual}

%%%%%%%%%%%%%%%%%%%%%%%%%%%%%%%%%%%%%%%%%%%%%%%%%%%%%%%%%%%%
\chapter{Experimental evaluation}

\section{Customer Landscape}

\section{Customer Requirements}

\section{Integration}

\subsection{Using the Operating Manual}

\subsection{Relevant Bricks}

\subsubsection{Data Bricks}

\subsubsection{Architecture Bricks}

\subsection{Resulting Integration}

\subsubsection{Integration Documentation}

\subsubsection{System Integration}

\subsubsection{Data Integration}

\section{Evaluation}

\subsection{Satisfaction of Requirements}

\subsubsection{Enterprise Architecture Requirements}

\subsubsection{Customer Requirements}

\subsubsection{Business Connector Requirements}

\subsubsection{Integration Architecture Requirements}

\subsection{Resource-Heaviness}

\subsubsection{Required time for application}

\subsubsection{Size of Resulting System}

\subsubsection{Increasing cost due to scaling}

%%%%%%%%%%%%%%%%%%%%%%%%%%%%%%%%%%%%%%%%%%%%%%%%%%%%%%%%%%%%
\chapter{Summary and outlook}

% References (Literaturverzeichnis):
% a) Style (with abbreviations: use alpha):
% see
% https://de.wikibooks.org/wiki/LaTeX-W%C3%B6rterbuch:_bibliographystyle
% for the different formats and styles

\bibliographystyle{apalike}
% b) The File:
\bibliography{references}

\end{document}
