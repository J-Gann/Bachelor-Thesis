% This template was initially provided by Dulip Withanage.
% Modifications for the database systems research group
% were made by Conny Junghans,  Jannik Strötgen and Michael Gertz

\documentclass[
     12pt,         % font size
     a4paper,      % paper format
     BCOR10mm,     % binding correction
     DIV14,        % stripe size for margin calculation
%     liststotoc,   % table listing in toc
%     bibtotoc,     % bibliography in toc
%     idxtotoc,     % index in toc
%     parskip       % paragraph skip instad of paragraph indent
     ]{scrreprt}

%%%%%%%%%%%%%%%%%%%%%%%%%%%%%%%%%%%%%%%%%%%%%%%%%%%%%%%%%%%%

% PACKAGES:

% Use German :
\usepackage[english]{babel}
% Input and font encoding
\usepackage[latin1]{inputenc}
\usepackage[T1]{fontenc}
% Index-generation
\usepackage{makeidx}
% Einbinden von URLs:
\usepackage{url}
% Special \LaTex symbols (e.g. \BibTeX):
%\usepackage{doc}
% Include Graphic-files:
\usepackage{graphicx}
% Include doc++ generated tex-files:
%\usepackage{docxx}
% Include PDF links
%\usepackage[pdftex, bookmarks=true]{hyperref}

% Fuer anderthalbzeiligen Textsatz
\usepackage{setspace}

% hyperrefs in the documents
\usepackage[bookmarks=true,colorlinks,pdfpagelabels,pdfstartview = FitH,bookmarksopen = true,bookmarksnumbered = true,linkcolor = black,plainpages = false,hypertexnames = false,citecolor = black,urlcolor=black]{hyperref} 
%\usepackage{hyperref}

%%%%%%%%%%%%%%%%%%%%%%%%%%%%%%%%%%%%%%%%%%%%%%%%%%%%%%%%%%%%

% OTHER SETTINGS:

% Pagestyle:
\pagestyle{headings}

% Choose language
\newcommand{\setlang}[1]{\selectlanguage{#1}\nonfrenchspacing}


\begin{document}

% TITLE:
\pagenumbering{roman}
\begin{titlepage}
     \vspace*{1cm}
     \begin{center}
          \vspace*{3cm}
          \textbf
          {
               \Large University of Heidelberg\\
               \smallskip
               \Large Institute for Computer Science\\
               \smallskip
               \Large Working group database systems\\
               \smallskip
          }

          \vspace{3cm}

          \textbf{\large Bachelor thesis}

          \vspace{0.5\baselineskip}
          {
               \huge
               \textbf{Messaging Architecture for Integration of Customer Self-Services}
          }

     \end{center}

     \vfill
     {
          \large
          \begin{tabular}[l]{ll}
               Name:                 & Jonas Gann              \\
               Matriculation number: & 3367576                 \\
               Supervisor:           & Prof. Dr. Michael Gertz \\
               Date of submission:   & \today
          \end{tabular}
     }

\end{titlepage}

\onehalfspacing

\thispagestyle{empty}

\vspace*{100pt}
\noindent
I assure that I have written this bachelor thesis on my own and only used the specified sources and
resources and that I followed the principles and recommendations "Responsibility in Science" of the
University of Heidelberg.

\vspace*{50pt}
\noindent

\underline{\phantom{mmmmmmmmmmmmmmmmmmmm}}

\medskip
\noindent
Date of Submission: \today
\newpage

\chapter*{Zusammenfassung}

\newpage

\chapter*{Abstract}

\newpage

\tableofcontents
\cleardoublepage
\pagenumbering{arabic}

\chapter{Context}

The german ministry of economy and energy recognizes that "we have long since arrived in a digitized world". This realization 
of the german government comes along with important federal and european legislation. Most prominent are DSGVO, OZG and DVG, 
modernizing data protection, governmental administration and health care.

Enterprises, organizations and authorities are required to provide additional digital services for its customers and users.
Those include for example the deletion of non-mandatory data and the digital application for unemployment benefit.

Working with existing system- and data architectures, integration of new solutions is necessary.

\chapter{Objectives}

This bachelor thesis describes digital self-service as a solution for the additional requirements. It explains the basis of what 
customer self-service (CSS) is, what relevant legislation is about and how CSS providers can help in the digital transformation.

The technological challenge of integrating CSS providers into existing system architectures is focus of this bachelor thesis.
An integration architecture is presented, which provides solutions for the following questions (ToDo: In Textform schreiben):

\begin{enumerate}
     \item How can services of a CSS provider be accessed by the enterprise architecture?
     \item Which systems of a typical enterprise architecture are required?
     \item Which data objects of a typical enterprise architecture are required?
     \item Which additional systems and data objects are required?
     \item How can heterogeneous enterprise architectures be integrated the same way?
     \item How can the integration be non-invasive?
     \item How can the speed of integration development be increased?
     \item How can the speed of integration deployment be increased?
     \item How can the integration system be reliable and maintainable?
\end{enumerate}

The architecture is evaluated in respect to technological feasibility and real-life applicability. The results are incorporated into 
an operating manual, providing guidance in developing and deploying the presented integration architecture.

\chapter{Structure of Work}

\chapter{Customer Self-Service (CSS)}

\section{Definition}

\section{Description}

\section{Benefits}

\subsection{Customer}

\subsection{Organization}

\section{Examples}

\chapter{Governmental Regulations}

\section{Purpose}

\section{Importance for Organizations}

\section{DSGVO}

\section{OZG}

\section{DVG}

\section{Challenges for Organizations}

\chapter{Compliance through CSS}

\section{Description}

\section{CSS-Scenarios}

\section{Requirements}

\chapter{CSS Providers}

\section{Description}

\section{CSS Solutions}

\section{Business Connector}

\chapter{Enterprise Architectures}

\section{Modeling of the real world}

\section{Modeling Challenges}

\section{Enterprise Architecture Patterns (EAP)}

\chapter{Relevant Systems and Data}

\section{EAPs}

\section{Architecture Bricks}

\section{Data Bricks}

\section{Integration Requirements}

\subsection{Regarding Integration Architecture}

\subsection{Regarding Business Connector}

\chapter{Integration}

\section{Definition}

\section{Requirements}

\subsection{Loose Coupling}

\subsection{Homogeneous Landscapes}

\section{Enterprise Integration Patterns (EIP)}

\subsection{Pattern 1}

\subsection{Pattern 2}

\chapter{Business Connector}

\section{Functionalities and Interfaces}

\section{Integration Requirements}

\subsection{Regarding Enterprise Architecture}

\subsection{Regarding Integration Architecture}

\section{Documentation}

\subsection{Connector as Architecture Brick}

\chapter{Integration Architecture}

\section{Scenario 1}

\subsection{Integration Documentation}

\subsection{System Integration}

\subsection{Data Integration}

\section{Scenario 2}

\subsection{Integration Documentation}

\subsection{System Integration}

\subsection{Data Integration}

\chapter{Integration Architecture Evaluation}

\section{Technology}

\section{Customer Example}

\section{Operating Manual}

\bibliographystyle{apalike}
\bibliography{references}

\end{document}
