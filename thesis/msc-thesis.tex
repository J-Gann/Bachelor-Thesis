% This template was initially provided by Dulip Withanage.
% Modifications for the database systems research group
% were made by Conny Junghans,  Jannik Strötgen and Michael Gertz

\documentclass[
     12pt,         % font size
     a4paper,      % paper format
     BCOR10mm,     % binding correction
     DIV14,        % stripe size for margin calculation
%     liststotoc,   % table listing in toc
%     bibtotoc,     % bibliography in toc
%     idxtotoc,     % index in toc
%     parskip       % paragraph skip instad of paragraph indent
     ]{scrreprt}

%%%%%%%%%%%%%%%%%%%%%%%%%%%%%%%%%%%%%%%%%%%%%%%%%%%%%%%%%%%%

% PACKAGES:

% Use German :
\usepackage[english]{babel}
% Input and font encoding
\usepackage[latin1]{inputenc}
\usepackage[T1]{fontenc}
% Index-generation
\usepackage{makeidx}
% Einbinden von URLs:
\usepackage{url}
% Special \LaTex symbols (e.g. \BibTeX):
%\usepackage{doc}
% Include Graphic-files:
\usepackage{graphicx}
% Include doc++ generated tex-files:
%\usepackage{docxx}
% Include PDF links
%\usepackage[pdftex, bookmarks=true]{hyperref}

% Fuer anderthalbzeiligen Textsatz
\usepackage{setspace}

% hyperrefs in the documents
\usepackage[bookmarks=true,colorlinks,pdfpagelabels,pdfstartview = FitH,bookmarksopen = true,bookmarksnumbered = true,linkcolor = black,plainpages = false,hypertexnames = false,citecolor = black,urlcolor=black]{hyperref} 
%\usepackage{hyperref}

\usepackage[colorinlistoftodos,prependcaption,textsize=tiny]{todonotes}

\usepackage{color,soul}


%%%%%%%%%%%%%%%%%%%%%%%%%%%%%%%%%%%%%%%%%%%%%%%%%%%%%%%%%%%%

% OTHER SETTINGS:

% Pagestyle:
\pagestyle{headings}

% Choose language
\newcommand{\setlang}[1]{\selectlanguage{#1}\nonfrenchspacing}


\begin{document}

\listoftodos

% TITLE:
\pagenumbering{roman}
\begin{titlepage}
     \vspace*{1cm}
     \begin{center}
          \vspace*{3cm}
          \textbf
          {
               \Large University of Heidelberg\\
               \smallskip
               \Large Institute for Computer Science\\
               \smallskip
               \Large Working group database systems\\
               \smallskip
          }

          \vspace{3cm}

          \textbf{\large Bachelor thesis}

          \vspace{0.5\baselineskip}
          {
               \huge
               \textbf{Messaging Architecture for Integration of Customer Self-Services}
          }

     \end{center}

     \vfill
     {
          \large
          \begin{tabular}[l]{ll}
               Name:                 & Jonas Gann              \\
               Matriculation number: & 3367576                 \\
               Supervisor:           & Prof. Dr. Michael Gertz \\
               Date of submission:   & \today
          \end{tabular}
     }

\end{titlepage}

\onehalfspacing

\thispagestyle{empty}

\vspace*{100pt}
\noindent
I assure that I have written this bachelor thesis on my own and only used the specified sources and
resources and that I followed the principles and recommendations "Responsibility in Science" of the
University of Heidelberg.

\vspace*{50pt}
\noindent

\underline{\phantom{mmmmmmmmmmmmmmmmmmmm}}

\medskip
\noindent
Date of Submission: \today
\newpage

\chapter*{Zusammenfassung}

\newpage

\chapter*{Abstract}

\newpage

\tableofcontents
\cleardoublepage
\pagenumbering{arabic}

\chapter{Context}
\todo[inline, caption={Expand Context Chapter}]{Add more content to this chapter}


Digital customer self-service (DCSS) is a great tool for improving customer experience and reducing cost. Many enterprises therefore heavily
rely on it for main business services. Especially recent governmental regulations like "Onlinezugangsgesetz" (OZG), "Datenschutzgrundverordnung"
(DSGVO) and "Digitale Versorgung Gesetz" (DVG) have increased requirements for DCSS capabilities.

Adding DCSS to a system architecture is a common task. It usually involves integration of new DCSS components
into an existing system architecture. Due to the complicated nature of system architectures, integration proves to be a difficult
task.

One possibility for adding DCSS capabilities is the usage of solutions from DCSS providers. They keep companies and organizations up to date
with DCSS requirements and help with integration. Another possibility is the usage of websites and applications developed in self-work.

\chapter{Objectives}

\todo[inline, caption={Expand Objectives Chapter}]{Add more content to this chapter}

The objective of this bachelor thesis is to improve the capability of enterprises and organizations to integrate DCSS providers \hl{(or solutions?)}.
The thesis explains the basis of what DCSS and its providers are, which relevant scenarios exist and how DCSS providers can help
in the digital transformation.

The technological challenge of integrating DCSS providers into existing system architectures is focus of this bachelor thesis.

Based on DCSS scenarios resulting from governmental regulations (due to their current relevance), an integration architecture is presented, which describes
integration of these scenarios. It provides information on which type of systems and data objects an existing
architecture requires. It models how CSS providers and system architectures can communicate -
deliver data and instructions. And it shows how, with fast development and deployment speed, integration can be applied to
heterogeneous system architectures, be non-invasive and reliable.

The integration architecture is evaluated in respect to technological feasibility and real-life applicability. The results are
incorporated into an operating manual, providing guidance in developing and deploying the presented integration architecture.

\chapter{Structure of Work}

\todo[inline, caption={Write Structure of Work Chapter}]{Briefly explain each chapter and their relationships}

\chapter{Digital Customer Self-Service (DCSS)}

\todo[inline, caption={Expand DCSS Chapter}]{Add more content to this chapter}

This chapter introduces the concept of digital customer self-service. It defines DCSS, describes what it is by usage of examples and mentions benefits such as challenges.

\textbf{Definition}: Digital self-service is the consumption of services provided by an enterprise over the internet without human to human interaction.

It is important to distinguish between "service" and "self-service". The difference is shown through two examples:

\textbf{Example 1}: A service is the availability to purchase items whereas digital self-service is the order placement over a website.

\textbf{Example 2}: A service is the availability of video on demand whereas digital self-service is the access of video files over a website.

Digital self-service is one of many possible ways to access a service: An order placement could also be done through email
or phone call and access to videos could be granted through same-day-delivery of DVDs.

Customers are able to do self-service because companies provide them with self-service tools. Creating them is not trivial
and is the main business model of many companies. Video on demand websites for example do not only flourish as a result of providing access to
videos (this business model existed long before them) but as a result of the digital self-service tools they provide.

Reasons for the widespread usage of digital self-service are significant benefits in ease of service usability for customers and cost reduction for enterprises.
Customers can self-service from almost every place, easily manage customer data like phone number and address, configure services like subscription plans and
access lots of information on available products or trouble shooting steps in an immediate way.
Digital customer self-service improves usability, saves time, increases availability, saves money and therefore increases customer experience
\cite[cf. 243]{CSSInternetAge}.
Enterprises, especially in progressive countries, aim to keep employment at a minimum, as it is a big cost factor. Digital self-service turns customers into
unpaid employees \cite{served}.

Decreasing employment cost while increasing customer experience sounds like a win-win situation. Sadly, digital self-service does have its downsides.
One downside is most immediate to the elderly. In order to use DCSS, computers have to be available and one has to be experienced in their usage.
Technological complexity, however, is not only a problem for customers but also for enterprises providing DCSS. Based on existing system architectures,
adding DCSS functionalities can turn out to be a difficult task. Some form of integration is often necessary. With increased system complexity and business
dependability come additional maintenance cost.

As tody most large enterprises provide DCSS, it can be assumed, that at a given size of an enterprise (measured in number of customers), the benefits of DCSS
surpass its disadvantages.

\chapter{Governmental Regulations}

\todo[inline, caption={Research Governmental Regulations Chapter}]{Explain the three regulations in general, what their purpose is what their challenges are}

\section{DSGVO}

\section{OZG}

\section{DVG}


\chapter{DCSS Scenarios}

\todo[inline, caption={Research DCSS Scenarios Chapter}]{
     \begin{enumerate}
          \item Should governmental regulations be the focus of scenarios? => good resources for OZG
     \end{enumerate}
}

Usage of DCSS OZG scenarios (\textbf{process diagrams and data schemas available}
\cite{ozgScenarios}), DSGVO scenarios (textual description available \cite{dsgvo}) and DVG scenarios
(textual desciption available \cite{gematik-e-rezept})
\\\\
The scenarios include more than just CSS and are partly non-digital. Therefore processes relevant for DCSS have to be selected
and it has to be defined which processes are part of a CSS provider, the business systems or the integration architecture:

\begin{enumerate}
     \item What part of a scenario is digital?
     \item What part of a scenario can be done through a DCSS provider / is relevant for DCSS?
     \item What part of a scenario is done through \textbf{existing} business systems?
           \begin{enumerate}
                \item new necessary business logic part of (configurable) integration? => non invasive
           \end{enumerate}
     \item What part of a scenario should be done by the integration architecture?
     \item What are commonalities of scenarios?
           \begin{enumerate}
                \item OZG scenarios often describe digital handling of applications
                \item In OZG scenarios, often multiple governmental institutions are involved. This could be modeled as one system architecture with the problem of distributed
                      data and functionality which the integration architecture would solve.
           \end{enumerate}
\end{enumerate}

\section{OZG}

\subsection{Unemployment Benefit (AG2)}
\textbf{Available Resources}
\begin{itemize}
     \item Data Schema
     \item Process Diagram
     \item Process Diagram of Application
     \item Click-Dummy
\end{itemize}

\subsection{Training Promotion (BAf�G)}
\textbf{Available Resources}
\begin{itemize}
     \item Data Schema
     \item Process Diagram
     \item Click-Dummy
\end{itemize}

\subsection{Drivers License}
\textbf{Available Resources}
\begin{itemize}
     \item Data Schema
     \item Process Diagram
     \item Process Diagram of Application
     \item Click-Dummy
\end{itemize}

\subsection{Company Registration and Approval}
\textbf{Available Resources}
\begin{itemize}
     \item Data Schema
     \item Process Diagram
     \item Click-Dummy
\end{itemize}

\subsection{Visa Issuance}
\textbf{Available Resources}
\begin{itemize}
     \item Data Schema
     \item Process Diagram
\end{itemize}

\section{DSGVO}

\subsection{Transparent Accessability}
The enterprise or organization can have the duty to make information about personal data accessible in transparent and understandable ways

\subsection{Deletion of Personal Data}
The user can have the right of removal of personal data

\subsection{Correction of Personal Data}
The user can have the right of modification of personal data.

\subsection{Information about collected Personal Data}
The user can have the right to be informed about various information of collected and stored personal data such as usage, duration of storage, ...

\subsection{Restriction of Processing}
The user can have the right to restrict the way in which personal data is processed

\subsection{Availability of Transferral of Personal Data}
The user can have the right to transfer personal data to himself or separate entities (company, organization).

\section{DVG}

\subsection{e-Prescription}
Reference can be found in the "�rzteblatt" \cite{�rzteblatt}.
\\\\
The user can have the right to manage his prescriptions via an App. Doctors give digital prescriptions to users app,
which can digitally be used in pharmacies also without physically being there.
\\\\
Regulation: "Gesetz f�r mehr Sicherheit in der Arzneimittelversorgung (GSAV)"
\\\\
e-Prescription processes \cite{gematik-e-rezept}
\begin{enumerate}
     \item Doctor creates digital prescription if he has access to "Telematikinfrastruktur" and "Heilberufausweis"
     \item Prescription is saved on "Telematikinfrastruktur"
     \item Doctor displays a QR-Code (prescription-token)
     \item Patient can receive QR-Code in App (or get a printout)
     \item Patient can check availability of drug in pharmacies via the app
     \item Patient can "apply" the prescription to a pharmacy via the app
     \item prescription is now irrevocably bound to the pharmacy and gets reserved
     \item pharmacy can use prescription-token (in QR-Code) to access prescription in "Telematikinfrastruktur"
     \item pharmacy notifies patient if or when the drug is available
     \item drug can be picked up by patient or be delivered
     \item If user picks drug up, he can show QR-Code in app or on printout (he got from doctor)
     \item pharmacy can access "Telematikinfrastruktur" to validate prescription
     \item pharmacy gives drug if validated
\end{enumerate}
~\\
What are services and what are self-services in this scenario:
\begin{itemize}
     \item Service: Access to prescription. Self-service: Digital management of prescriptions (prescription-tokens) inside app
     \item Service: Receiving drug for valid prescription. Self-service: Digital delivery of prescription (prescription-token)
           to pharmacy and digital instruction of delivery of drug
\end{itemize}




\section{Requirements}


\chapter{CSS Providers}

\todo[inline, caption={Write CSS Providers Chapter}]{Explain what CSS Providers are (what services they can provide -> what IDAS does) and how they can help creating DCSS.
     Also explain the concept of a business connector as technological communication point of the CSS provider}

\section{Description}

\section{CSS Solutions}

\section{Business Connector}


\chapter{Enterprise Architectures}

\todo[inline, caption={Write Enterprise Architectures Chapter}]{Introduce Enterprise Architectures and challenges such as heterogeneous systems when trying to model them}

\begin{enumerate}
     \item Heterogeneous Enterprise Architecture Systems
           \begin{enumerate}
                \item Different Applications
                \item Different Application Vendors
                \item Different / No Application Interfaces
                \item Legacy Systems
           \end{enumerate}
     \item Different (proprietary) data models within and between Enterprises and Organizations
           \begin{enumerate}
                \item Different (property)name for same data objects (syntactic integration)
                \item Different meanings for same (property)name (semantic integration)
           \end{enumerate}
\end{enumerate}

\section{Modeling of the real world}

\section{Modeling Challenges}

\section{Enterprise Architecture Patterns (EAP)}


\chapter{Relevant Systems and Data}

\todo[inline, caption={Research Relevant Systems and Data Chapter}]{Explain the approach of modeling enterprise architectures by EAPs and explain the used EAPs.
     Analyze which system and data bricks of the system architecture could be necessarry for DCSS}

\section{EAPs}

\section{Architecture Bricks}

\section{Data Bricks}

\section{Integration Requirements}

\subsection{Regarding Integration Architecture}

\subsection{Regarding Business Connector}


\chapter{Integration}

\todo[inline, caption={Write Integration Chapter}]{Introduce the concept of integration. What are challenges in general and what would be an ideal integration.
     Also explain the approach towards integration used (messaging, patterns)}

\begin{enumerate}
     \item Scarce Resources
           \begin{enumerate}
                \item Integration Development Speed
                      \begin{enumerate}
                           \item Necessary Development <=> Reuse of existing Technology
                           \item Complexity / Size of Integration
                      \end{enumerate}
                \item Maintenance of finished Integration
                \item Hardware / Software Costs of Integration
                      \begin{enumerate}
                           \item Licenses for Software
                           \item Scalability of Integration => Necessary Computing Power
                      \end{enumerate}
                \item Messaging Integration
                      \begin{enumerate}
                           \item Loose Coupling
                                 \begin{enumerate}
                                      \item Loose Coupling simplifies adaption to changing EA => simpler Maintenance
                                      \item Loose Coupling simplifies integration of new EA systems => integration of heterogeneous EA
                                      \item Loose Coupling allows Reuse of "Modules" => faster development
                                 \end{enumerate}
                           \item Messaging enables communication with many systems through Adapters
                           \item Stability of Integration
                                 \begin{enumerate}
                                      \item Future Changes of EA
                                      \item Scalability
                                      \item Failure of EA or Integration Components
                                 \end{enumerate}
                           \item Messaging provides mechanisms for Stability
                                 \begin{enumerate}
                                      \item Store-and-Forward
                                      \item Load Balancing
                                 \end{enumerate}
                      \end{enumerate}
                \item Integration Patterns
                      \begin{enumerate}
                           \item Patterns speed up construction of Integration Architecture
                           \item Patterns are proven solutions
                           \item Patterns abstract from concrete technologies
                                 \begin{enumerate}
                                      \item Simplifies understanding of integration concept
                                      \item Allows implementation with different technologies
                                 \end{enumerate}
                      \end{enumerate}
           \end{enumerate}
\end{enumerate}

\section{Definition}

\section{Requirements}

\subsection{Loose Coupling}

\subsection{Homogeneous Landscapes}

\section{Enterprise Integration Patterns (EIP)}

\subsection{Pattern 1}

\subsection{Pattern 2}


\chapter{Business Connector}

\todo[inline, caption={Research Business Connector Chapter}]{Construct the model of the business connector}

\section{Functionalities and Interfaces}

\section{Integration Requirements}

\subsection{Regarding Enterprise Architecture}

\subsection{Regarding Integration Architecture}

\section{Documentation}

\subsection{Connector as Architecture Brick}


\chapter{Integration Architecture}

\todo[inline, caption={Integration Architecture Chapter}]{Create the integration architecture}

\section{Scenario 1}

\subsection{Integration Documentation}

\subsection{Required System and Data Bricks}

\subsection{System Integration}

\subsection{Data Integration}

\section{Scenario 2}

\subsection{Integration Documentation}

\subsection{Required System and Data Bricks}

\subsection{System Integration}

\subsection{Data Integration}


\chapter{Integration Architecture Evaluation}

\section{Technology}

\section{Customer Example}

\section{Operating Manual}

\bibliographystyle{apalike}
\bibliography{references}

\end{document}
