% This template was initially provided by Dulip Withanage.
% Modifications for the database systems research group
% were made by Conny Junghans,  Jannik Str�tgen and Michael Gertz

\documentclass[
     12pt,         % font size
     a4paper,      % paper format
     BCOR10mm,     % binding correction
     DIV14,        % stripe size for margin calculation
%     liststotoc,   % table listing in toc
%     bibtotoc,     % bibliography in toc
%     idxtotoc,     % index in toc
%     parskip       % p
aragraph skip instad of paragraph indent
     ]{scrreprt}

%%%%%%%%%%%%%%%%%%%%%%%%%%%%%%%%%%%%%%%%%%%%%%%%%%%%%%%%%%%%

% PACKAGES:

% Use German :
\usepackage[english]{babel}
% Input and font encoding
\usepackage[latin1]{inputenc}
\usepackage[T1]{fontenc}
% Index-generation
\usepackage{makeidx}
% Einbinden von URLs:
\usepackage{url}
% Special \LaTex symbols (e.g. \BibTeX):
%\usepackage{doc}
% Include Graphic-files:
\usepackage{graphicx}
% Include doc++ generated tex-files:
%\usepackage{docxx}
% Include PDF links
%\usepackage[pdftex, bookmarks=true]{hyperref}

% Fuer anderthalbzeiligen Textsatz
\usepackage{setspace}

% hyperrefs in the documents
\usepackage[bookmarks=true,colorlinks,pdfpagelabels,pdfstartview = FitH,bookmarksopen = true,bookmarksnumbered = true,linkcolor = black,plainpages = false,hypertexnames = false,citecolor = black,urlcolor=black]{hyperref} 
%\usepackage{hyperref}

\usepackage[colorinlistoftodos,prependcaption,textsize=tiny]{todonotes}

\usepackage{color,soul}


%%%%%%%%%%%%%%%%%%%%%%%%%%%%%%%%%%%%%%%%%%%%%%%%%%%%%%%%%%%%

% OTHER SETTINGS:

% Pagestyle:
\pagestyle{headings}

% Choose language
\newcommand{\setlang}[1]{\selectlanguage{#1}\nonfrenchspacing}


\begin{document}

% TITLE:
\pagenumbering{roman}
\begin{titlepage}
     \vspace*{1cm}
     \begin{center}
          \vspace*{3cm}
          \textbf
          {
               \Large University of Heidelberg\\
               \smallskip
               \Large Institute for Computer Science\\
               \smallskip
               \Large Working group database systems\\
               \smallskip
          }

          \vspace{3cm}

          \textbf{\large Bachelor thesis}

          \vspace{0.5\baselineskip}
          {
               \huge
               \textbf{Messaging Architecture for Integration of Customer Self-Services}
          }

     \end{center}

     \vfill
     {
          \large
          \begin{tabular}[l]{ll}
               Name:                 & Jonas Gann              \\
               Matriculation number: & 3367576                 \\
               Supervisor:           & Prof. Dr. Michael Gertz \\
               Date of submission:   & \today
          \end{tabular}
     }

\end{titlepage}

\onehalfspacing

\thispagestyle{empty}

\vspace*{100pt}
\noindent
I assure that I have written this bachelor thesis on my own and only used the specified sources and resources and that I followed the principles and recommendations "Responsibility in Science" of the University of Heidelberg.

\vspace*{50pt}
\noindent

\underline{\phantom{mmmmmmmmmmmmmmmmmmmm}}

\medskip
\noindent
Date of Submission: \today
\newpage

\chapter*{Zusammenfassung}

\newpage

\chapter*{Abstract}

\newpage

\tableofcontents
\cleardoublepage
\pagenumbering{arabic}

\chapter{Context}

Many enterprises provide digital tools which improve service usability for their customers. For example: If somebody wants to watch a movie, he does not drive to the nearest store to buy a DVD but instead uses websites providing video on demand (VoD). The reason is: VoD enterprises provide tools for digital customer self-service (DCSS). A customer can easily access the video over a website on his own. DCSS becomes increasingly relevant, as the German government introduced requirements for DCSS in many recent regulations. Most prominent are  "Onlinezugangsgesetz" (OZG), "Datenschutzgrundverordnung" (DSGVO) and "Digitale Versorgung Gesetz" (DVG). 

DCSS is made available through DCSS providers. They can consist of websites and applications developed in self-work or be services provided by separate companies. Adding DCSS providers to a system architecture is a common task. It usually involves integration of new DCSS components into an existing system architecture. Due to the complicated nature of system architectures, integration proves to be a difficult task.

\chapter{Objective}

The objective of this bachelor thesis is to improve the capability of enterprises and organizations to integrate DCSS providers. The thesis explains the basis of what DCSS and its providers are, which relevant scenarios exist and how DCSS providers can help in the digital transformation.

The technological challenge of integrating DCSS providers into existing system architectures is focus of this bachelor thesis.

Based on scenarios resulting from OZG, an integration architecture is presented, which describes their integration. It provides information on which type of systems and data objects an existing architecture requires. It models how CSS providers and system architectures can communicate - deliver data and instructions. And it shows how, with fast development and deployment speed, integration can be applied to heterogeneous system architectures, be non-invasive and reliable. The integration architecture enables expansion with self-service capabilities of different areas (e.g. online payment / shopping / information self-service) through future work.

The integration architecture is evaluated in respect to technological feasibility and real-life applicability. The results are incorporated into an operating manual, providing guidance in developing and deploying the presented integration architecture.

\chapter{Structure of Work}


\chapter{Fundamentals}

\section{Digital Customer Self-Service}

\todo[inline, caption={Expand DCSS Chapter}]{Add more content to this chapter}

This chapter introduces the concept of digital customer self-service. It defines DCSS, describes what it is by usage of examples and mentions benefits such as challenges.

\textbf{Definition}: Digital self-service is the consumption of services provided by an enterprise over the internet without human to human interaction.

It is important to distinguish between "service" and "self-service". The difference is shown through two examples:

\textbf{Example 1}: A service is the availability to purchase items whereas digital self-service is the placement of orders over a website.

\textbf{Example 2}: A service is the availability of video on demand whereas digital self-service is the access of video files over a website.

Digital self-service is one of many possible ways to access a service: An order placement could also be done through email or phone call and access to videos could be granted through same-day-delivery of DVDs.

Customers are able to do self-service because companies provide them with self-service tools. Creating them is not trivial and is the main business model of many companies. Video on demand websites for example do not only flourish as a result of providing access to videos (this business model existed long before them) but as a result of the digital self-service tools they provide.

Reasons for the widespread usage of digital self-service are significant benefits in ease of service usability for customers and cost reduction for enterprises. Customers can self-service from almost every place, easily manage customer data like phone number and address, configure services like subscription plans and access lots of information on available products or trouble shooting steps in an immediate way. Digital customer self-service improves usability, saves time, increases availability, saves money and therefore increases customer experience
\cite[cf. 243]{CSSInternetAge}.
Enterprises, especially in progressive countries, aim to keep employment at a minimum, as it is a big cost factor. Digital self-service turns customers into unpaid employees \cite{served}.

Decreasing employment cost while increasing customer experience sounds like a win-win situation. Sadly, digital self-service does have its downsides. One downside is most immediate to the elderly. In order to use DCSS, computers have to be available and one has to be experienced in their usage. Technological complexity, however, is not only a problem for customers but also for enterprises providing DCSS. Based on existing system architectures, adding DCSS functionalities can turn out to be a difficult task. Some form of integration is often necessary. With increased system complexity and business dependability come additional maintenance cost.

As today most large enterprises provide DCSS, it can be assumed, that at a given size of an enterprise (measured in number of customers), the benefits of DCSS surpass its disadvantages.

Governments see digitalization as a chance for improvements in many important social and economical areas\todo{references}. The process of digitalization can in some cases be described as reinventing existing solutions for usability with computers. One of the goals is to make solutions more easily accessible for users and as a method, provide digital self-service. Governmental regulations relating to digitalization therefore often encourage or demand enterprises and organizations to provide DCSS.

\section{OZG}



\section{DCSS Provider}

\subsection{Description}

\subsection{CSS Solutions}

\subsection{Business Connector}

\section{Enterprise Architecture}

\begin{enumerate}
     \item Heterogeneous Enterprise Architecture Systems
           \begin{enumerate}
                \item Different Applications
                \item Different Application Vendors
                \item Different / No Application Interfaces
                \item Legacy Systems
           \end{enumerate}
     \item Different (proprietary) data models within and between Enterprises and Organizations
           \begin{enumerate}
                \item Different (property)name for same data objects (syntactic integration)
                \item Different meanings for same (property)name (semantic integration)
           \end{enumerate}
\end{enumerate}

\subsection{Modeling of the real world}

\subsection{Modeling Challenges}

\subsection{Enterprise Architecture Patterns (EAP)}


\section{Integration}

\begin{enumerate}
     \item Scarce Resources
           \begin{enumerate}
                \item Integration Development Speed
                      \begin{enumerate}
                           \item Necessary Development <=> Reuse of existing Technology
                           \item Complexity / Size of Integration
                      \end{enumerate}
                \item Maintenance of finished Integration
                \item Hardware / Software Costs of Integration
                      \begin{enumerate}
                           \item Licenses for Software
                           \item Scalability of Integration => Necessary Computing Power
                      \end{enumerate}
                \item Messaging Integration
                      \begin{enumerate}
                           \item Loose Coupling
                                 \begin{enumerate}
                                      \item Loose Coupling simplifies adaption to changing EA => simpler Maintenance
                                      \item Loose Coupling simplifies integration of new EA systems => integration of heterogeneous EA
                                      \item Loose Coupling allows Reuse of "Modules" => faster development
                                 \end{enumerate}
                           \item Messaging enables communication with many systems through Adapters
                           \item Stability of Integration
                                 \begin{enumerate}
                                      \item Future Changes of EA
                                      \item Scalability
                                      \item Failure of EA or Integration Components
                                 \end{enumerate}
                           \item Messaging provides mechanisms for Stability
                                 \begin{enumerate}
                                      \item Store-and-Forward
                                      \item Load Balancing
                                 \end{enumerate}
                      \end{enumerate}
                \item Integration Patterns
                      \begin{enumerate}
                           \item Patterns speed up construction of Integration Architecture
                           \item Patterns are proven solutions
                           \item Patterns abstract from concrete technologies
                                 \begin{enumerate}
                                      \item Simplifies understanding of integration concept
                                      \item Allows implementation with different technologies
                                 \end{enumerate}
                      \end{enumerate}
           \end{enumerate}
\end{enumerate}

\subsection{Definition}

\subsection{Requirements}

\subsubsection{Loose Coupling}

\subsubsection{Homogeneous Landscapes}


\chapter{Preparatory Work}

\section{OZG Scenarios}

Usage of DCSS OZG scenarios (\textbf{process diagrams and data schemas available}
\cite{ozgScenarios})
\\\\
The scenarios include more than just CSS and are partly non-digital. Therefore processes relevant for DCSS have to be selected
and it has to be defined which processes are part of a CSS provider, the business systems or the integration architecture:

\begin{enumerate}
     \item What part of a scenario is digital?
     \item What part of a scenario can be done through a DCSS provider / is relevant for DCSS?
     \item What part of a scenario is done through \textbf{existing} business systems?
           \begin{enumerate}
                \item new necessary business logic part of (configurable) integration? => non invasive
           \end{enumerate}
     \item What part of a scenario should be done by the integration architecture?
     \item What are commonalities of scenarios?
           \begin{enumerate}
                \item OZG scenarios often describe digital handling of applications
                \item In OZG scenarios, often multiple governmental institutions are involved. This could be modeled as one system architecture with the problem of distributed
                      data and functionality which the integration architecture would solve.
           \end{enumerate}
\end{enumerate}

\begin{itemize}
     \item Unemployment Benefit (AG2)
\begin{itemize}
     \item Data Schema
     \item Process Diagram
     \item Process Diagram of Application
     \item Click-Dummy
\end{itemize}

\item Training Promotion (BAf�G)
\begin{itemize}
     \item Data Schema
     \item Process Diagram
     \item Click-Dummy
\end{itemize}

\item Drivers License
\begin{itemize}
     \item Data Schema
     \item Process Diagram
     \item Process Diagram of Application
     \item Click-Dummy
\end{itemize}

\item Company Registration and Approval
\begin{itemize}
     \item Data Schema
     \item Process Diagram
     \item Click-Dummy
\end{itemize}

\item Visa Issuance
\begin{itemize}
     \item Data Schema
     \item Process Diagram
\end{itemize}


\end{itemize}


\section{Relevant Systems and Data}

\subsection{EAPs}

\subsection{Architecture Bricks}

\subsection{Data Bricks}

\subsection{Integration Requirements}

\subsubsection{Regarding Integration Architecture}

\subsubsection{Regarding Business Connector}



\section{Business Connector}

\subsection{Functionalities and Interfaces}

\subsection{Integration Requirements}

\subsubsection{Regarding Enterprise Architecture}

\subsubsection{Regarding Integration Architecture}

\subsection{Documentation}

\subsubsection{Connector as Architecture Brick}


\chapter{Integration Architecture}

\section{Scenario 1}

\subsection{Integration Documentation}

\subsection{Required System and Data Bricks}

\subsection{System Integration}

\subsection{Data Integration}

\section{Scenario 2}

\subsection{Integration Documentation}

\subsection{Required System and Data Bricks}

\subsection{System Integration}

\subsection{Data Integration}


\chapter{Integration Architecture Evaluation}

\section{Technology}

\section{Customer Example}

\section{Operating Manual}

\chapter{Outlook}

\begin{itemize}
     \item Expanding integration architecture with capabilities for DCSS scenarios of different areas than online administration self-service
     \begin{itemize}
          \item online payment self-service (PayPal, psd2)
          \item online shopping self-service (Amazon)
          \item online health care self-service (DVG)
          \item online entertainment self-service (Netflix, YouTube, Spotify)
          \item online information / support self-service (StackOverflow)
     \end{itemize}
\end{itemize}

\bibliographystyle{apalike}
\bibliography{references}

\end{document}
