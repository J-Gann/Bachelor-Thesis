% This template was initially provided by Dulip Withanage.
% Modifications for the database systems research group
% were made by Conny Junghans,  Jannik Strötgen and Michael Gertz

\documentclass[
     12pt,         % font size
     a4paper,      % paper format
     BCOR10mm,     % binding correction
     DIV14,        % stripe size for margin calculation
%     liststotoc,   % table listing in toc
%     bibtotoc,     % bibliography in toc
%     idxtotoc,     % index in toc
%     parskip       % paragraph skip instad of paragraph indent
     ]{scrreprt}

%%%%%%%%%%%%%%%%%%%%%%%%%%%%%%%%%%%%%%%%%%%%%%%%%%%%%%%%%%%%

% PACKAGES:

% Use German :
\usepackage[english]{babel}
% Input and font encoding
\usepackage[latin1]{inputenc}
\usepackage[T1]{fontenc}
% Index-generation
\usepackage{makeidx}
% Einbinden von URLs:
\usepackage{url}
% Special \LaTex symbols (e.g. \BibTeX):
%\usepackage{doc}
% Include Graphic-files:
\usepackage{graphicx}
% Include doc++ generated tex-files:
%\usepackage{docxx}
% Include PDF links
%\usepackage[pdftex, bookmarks=true]{hyperref}

% Fuer anderthalbzeiligen Textsatz
\usepackage{setspace}

% hyperrefs in the documents
\usepackage[bookmarks=true,colorlinks,pdfpagelabels,pdfstartview = FitH,bookmarksopen = true,bookmarksnumbered = true,linkcolor = black,plainpages = false,hypertexnames = false,citecolor = black,urlcolor=black]{hyperref} 
%\usepackage{hyperref}

%%%%%%%%%%%%%%%%%%%%%%%%%%%%%%%%%%%%%%%%%%%%%%%%%%%%%%%%%%%%

% OTHER SETTINGS:

% Pagestyle:
\pagestyle{headings}

% Choose language
\newcommand{\setlang}[1]{\selectlanguage{#1}\nonfrenchspacing}


\begin{document}

% TITLE:
\pagenumbering{roman}
\begin{titlepage}
     \vspace*{1cm}
     \begin{center}
          \vspace*{3cm}
          \textbf
          {
               \Large University of Heidelberg\\
               \smallskip
               \Large Institute for Computer Science\\
               \smallskip
               \Large Working group database systems\\
               \smallskip
          }

          \vspace{3cm}

          \textbf{\large Bachelor thesis}

          \vspace{0.5\baselineskip}
          {
               \huge
               \textbf{Messaging Architecture for Integration of Customer Self-Services}
          }

     \end{center}

     \vfill
     {
          \large
          \begin{tabular}[l]{ll}
               Name:                 & Jonas Gann              \\
               Matriculation number: & 3367576                 \\
               Supervisor:           & Prof. Dr. Michael Gertz \\
               Date of submission:   & \today
          \end{tabular}
     }

\end{titlepage}

\onehalfspacing

\thispagestyle{empty}

\vspace*{100pt}
\noindent
I assure that I have written this bachelor thesis on my own and only used the specified sources and
resources and that I followed the principles and recommendations "Responsibility in Science" of the
University of Heidelberg.

\vspace*{50pt}
\noindent

\underline{\phantom{mmmmmmmmmmmmmmmmmmmm}}

\medskip
\noindent
Date of Submission: \today
\newpage

\chapter*{Zusammenfassung}

\newpage

\chapter*{Abstract}

\newpage

% MAIN PART:
% Table of contents (Inhaltsverzeichnis)
\tableofcontents
\cleardoublepage
\pagenumbering{arabic}

% List of figures (Abbildungsverzeichnis):
%\listoffigures
% List of tables (Tabellenverzeichnis):
%\listoftables

%%%%%%%%%%%%%%%%%%%%%%%%%%%%%%%%%%%%%%%%%%%%%%%%%%%%%%%%%%%%%%%

\chapter{Introduction}\label{intro}

\section{Motivation}

\subsection{Evolution of digital Customer Self-Service}
\begin{enumerate}
     \item Increasing consumer standards
           \begin{enumerate}
                \item Passwordless sign in
                \item instant updates
           \end{enumerate}

     \item New regulations
           \begin{enumerate}
                \item DSGVO
                \item Onlinezugangsgesetz
                \item DVG (eAU, eRezept)
                \item psd2
           \end{enumerate}

\end{enumerate}

\subsection{CSS as a Service}
\begin{enumerate}
     \item Enterprises can offer CSS service, satisfying modern requirements and regulations
\end{enumerate}

\subsection{Service Integration}
\begin{enumerate}
     \item The provided CSS service has to be integrated into existing enterprise architecture
\end{enumerate}

\section{Goals of the work}

Solution for integration challenges
\begin{enumerate}
     \item Heterogeneous Enterprise Architecture Systems
           \begin{enumerate}
                \item Different Applications
                \item Different Application Vendors
                \item Different / No Application Interfaces
                \item Legacy Systems
           \end{enumerate}

     \item Different (proprietary) data models within and between Enterprises and Organizations
           \begin{enumerate}
                \item Different (property)name for same data objects (syntactic integration)
                \item Different meanings for same (property)name (semantic integration)
           \end{enumerate}

     \item Stability of Integration
           \begin{enumerate}
                \item Future Changes of EA
                \item Scalability
                \item Failure of EA or Integration Components
           \end{enumerate}

     \item Scarce Resources
           \begin{enumerate}
                \item Integration Development Speed
                      \begin{enumerate}
                           \item Necessary Development <=> Reuse of existing Technology
                           \item Complexity / Size of Integration
                      \end{enumerate}

                \item Maintenance of finished Integration

                \item Hardware / Software Costs of Integration
                      \begin{enumerate}
                           \item Licenses for Software
                           \item Scalability of Integration => Necessary Computing Power
                      \end{enumerate}

           \end{enumerate}

\end{enumerate}

\section{Structure of the work}

%%%%%%%%%%%%%%%%%%%%%%%%%%%%%%%%%%%%%%%%%%%%%%%%%%%%%%%%%%%%
\newpage

\chapter{Foundation and related work}

\section{Customer Self-Service}
\begin{enumerate}
     \item Definition in the context of the thesis
     \item Purpose in Thesis
     \item Overview of CSS-Scenarios found in literature
\end{enumerate}

\section{Architecture Patterns}
\begin{enumerate}
     \item Definition
     \item Purpose in Thesis
     \item Explanation of used Patterns
\end{enumerate}

\subsection{Business Connector}
\begin{enumerate}
     \item Definition in context of the thesis
     \item Purpose in Thesis
\end{enumerate}

\section{Messaging Patterns}
\begin{enumerate}
     \item Definition
     \item Messaging
           \begin{enumerate}
                \item Answer: Why is messaging used?
           \end{enumerate}
     \item Patterns
           \begin{enumerate}
                \item Answer: Why are patterns used?
           \end{enumerate}
     \item Explanation of used patterns
\end{enumerate}

%%%%%%%%%%%%%%%%%%%%%%%%%%%%%%%%%%%%%%%%%%%%%%%%%%%%%%%%%%%%
\newpage

\chapter{Groundwork}

\subsection{CSS-Scenarios}
\begin{enumerate}
     \item Categorization of Scenarios
     \item Combining similar scenarios (into a more general one)
     \item Prioritization
     \item ToDo: How to document?
     \item Documentation of final list
\end{enumerate}

\subsection{Enterprise Architecture}
\begin{enumerate}
     \item Architecture- and Data bricks
           \begin{enumerate}
                \item Selection of EAPs relevant for CSS
                      \begin{enumerate}
                           \item Derived by comparing CSS-Scenarios with business processes
                      \end{enumerate}
                \item Evaluation of Architecture bricks and data bricks of EAPs relevant for CSS-Scenarios
                \item Documentation of resulting Architecture and data bricks
           \end{enumerate}

     \item Requirements of Enterprise Architecture
           \begin{enumerate}
                \item List of requirements towards the integration Architecture
                      \begin{enumerate}
                           \item Adapters for systems
                           \item Required Format of incoming data
                      \end{enumerate}
           \end{enumerate}
\end{enumerate}


\section{Business Connector}
\begin{enumerate}
     \item How to document the business Connector
     \item List of provided functionalities along with their interfaces
           \begin{enumerate}
                \item Derived from CSS-Scenarios
                \item Derived from IDAS Connector
           \end{enumerate}

     \item List of requirements regarding the EA
           \begin{enumerate}
                \item Access to architecture and data bricks
           \end{enumerate}

     \item List of requirements regarding the integration Architecture
           \begin{enumerate}
                \item Format of data
                \item Order of delivery
                \item Time relevance of data
                \item Guaranteed delivery of data
           \end{enumerate}

\end{enumerate}

%%%%%%%%%%%%%%%%%%%%%%%%%%%%%%%%%%%%%%%%%%%%%%%%%%%%%%%%%%%%
\newpage

\chapter{Integration Architecture}

\section{Overview}

\section{Requirements}

\begin{enumerate}
     \item Requirements of Enterprise Architecture
     \item Requirements of Business Connector
     \item Requirements of Integration System
\end{enumerate}

\subsection{Requirements of Enterprise Architecture}

\subsection{Requirements of Business Connector}

\subsection{Requirements of Integration System}

\section{Scenario Integrations}
\begin{enumerate}
     \item Look at integration architecture for each CSS-Scenario
\end{enumerate}

\subsection{Integration Architecture Documentation Method}
\begin{enumerate}
     \item Entity Diagram(s)
     \item Flow Diagram(s)
     \item etc.
\end{enumerate}

\subsection{Scenario1}
\begin{enumerate}
     \item Description of Scenario
\end{enumerate}

\subsubsection{Scenario1 Integration Architecture}
\begin{enumerate}
     \item Entity Diagram(s)
     \item Flow Diagram(s)
     \item etc.
\end{enumerate}

\subsubsection{System Integration}
\begin{enumerate}
     \item Explanation how the systems are integrated
     \item Why was each pattern used?
     \item How are requirements met?
\end{enumerate}

\subsubsection{Data Integration}
\begin{enumerate}
     \item Explanation how the data objects are integrated
     \item Why was data transformed / mapped?
     \item How are requirements met
\end{enumerate}

\subsection{Scenario2}

\subsubsection{Scenario2 Integration Architecture}

\subsubsection{System Integration}

\subsubsection{Data Integration}

\chapter{Evaluation}

\section{Technological Evaluation}

\subsection{Available Messaging Technologies}

\subsection{Implementation of Integration Architecture}

\subsection{Evaluation of result}

\section{Experimental Evaluation}

\subsection{Customer Example}

\subsection{Application of Integration Architecture}

\subsection{Evaluation of result}

\section{Operation Manual}
\begin{enumerate}
     \item Guide on how to implement and deploy the integration architecture
     \item incorporates results of previous evaluations
\end{enumerate}

%%%%%%%%%%%%%%%%%%%%%%%%%%%%%%%%%%%%%%%%%%%%%%%%%%%%%%%%%%%%
\chapter{Summary and outlook}

% References (Literaturverzeichnis):
% a) Style (with abbreviations: use alpha):
% see
% https://de.wikibooks.org/wiki/LaTeX-W%C3%B6rterbuch:_bibliographystyle
% for the different formats and styles

\bibliographystyle{apalike}
% b) The File:
\bibliography{references}

\end{document}
