% This template was initially provided by Dulip Withanage.
% Modifications for the database systems research group
% were made by Conny Junghans,  Jannik Strötgen and Michael Gertz

\documentclass[
     12pt,         % font size
     a4paper,      % paper format
     BCOR10mm,     % binding correction
     DIV14,        % stripe size for margin calculation
%     liststotoc,   % table listing in toc
%     bibtotoc,     % bibliography in toc
%     idxtotoc,     % index in toc
%     parskip       % paragraph skip instad of paragraph indent
     ]{scrreprt}

%%%%%%%%%%%%%%%%%%%%%%%%%%%%%%%%%%%%%%%%%%%%%%%%%%%%%%%%%%%%

% PACKAGES:

% Use German :
\usepackage[english]{babel}
% Input and font encoding
\usepackage[latin1]{inputenc}
\usepackage[T1]{fontenc}
% Index-generation
\usepackage{makeidx}
% Einbinden von URLs:
\usepackage{url}
% Special \LaTex symbols (e.g. \BibTeX):
%\usepackage{doc}
% Include Graphic-files:
\usepackage{graphicx}
% Include doc++ generated tex-files:
%\usepackage{docxx}
% Include PDF links
%\usepackage[pdftex, bookmarks=true]{hyperref}

% Fuer anderthalbzeiligen Textsatz
\usepackage{setspace}

% hyperrefs in the documents
\usepackage[bookmarks=true,colorlinks,pdfpagelabels,pdfstartview = FitH,bookmarksopen = true,bookmarksnumbered = true,linkcolor = black,plainpages = false,hypertexnames = false,citecolor = black,urlcolor=black]{hyperref} 
%\usepackage{hyperref}

%%%%%%%%%%%%%%%%%%%%%%%%%%%%%%%%%%%%%%%%%%%%%%%%%%%%%%%%%%%%

% OTHER SETTINGS:

% Pagestyle:
\pagestyle{headings}

% Choose language
\newcommand{\setlang}[1]{\selectlanguage{#1}\nonfrenchspacing}


\begin{document}

% TITLE:
\pagenumbering{roman}
\begin{titlepage}
     \vspace*{1cm}
     \begin{center}
          \vspace*{3cm}
          \textbf
          {
               \Large University of Heidelberg\\
               \smallskip
               \Large Institute for Computer Science\\
               \smallskip
               \Large Working group database systems\\
               \smallskip
          }

          \vspace{3cm}

          \textbf{\large Bachelor thesis}

          \vspace{0.5\baselineskip}
          {
               \huge
               \textbf{Messaging Architecture for Integration of Customer Self-Services}
          }

     \end{center}

     \vfill
     {
          \large
          \begin{tabular}[l]{ll}
               Name:                 & Jonas Gann              \\
               Matriculation number: & 3367576                 \\
               Supervisor:           & Prof. Dr. Michael Gertz \\
               Date of submission:   & \today
          \end{tabular}
     }

\end{titlepage}

\onehalfspacing

\thispagestyle{empty}

\vspace*{100pt}
\noindent
I assure that I have written this bachelor thesis on my own and only used the specified sources and
resources and that I followed the principles and recommendations "Responsibility in Science" of the
University of Heidelberg.

\vspace*{50pt}
\noindent

\underline{\phantom{mmmmmmmmmmmmmmmmmmmm}}

\medskip
\noindent
Date of Submission: \today
\newpage

\chapter*{Zusammenfassung}

\newpage

\chapter*{Abstract}

\newpage

\tableofcontents
\cleardoublepage
\pagenumbering{arabic}

\chapter{Context}

The german ministry of economy and energy recognizes that "we have long since arrived in a digitized world". This realization 
of the german government comes along with important federal and european legislation. Most prominent are DSGVO, OZG and DVG, 
modernizing data protection, governmental administration and health care.

Enterprises, organizations and authorities are required to provide digital self-service for its customers and users.
This includes for example the deletion of non-mandatory data and the digital application for unemployment benefit.

Working with existing system- and data architectures, integration of digital customer self-service solutions is necessary.

\chapter{Objectives}

This bachelor thesis explains the basis of what digital customer self-service (DCSS) is, what relevant legislation is about and how 
CSS providers can help in the digital transformation.

The technological challenge of integrating CSS providers into existing system architectures is focus of this bachelor thesis.
An integration architecture is provided which models integration of the top 10 OZG DCSS processes while satisfying DSGVO 
requirements.

The integration architecture solves the following questions:
\begin{enumerate}
     \item How can services of a CSS provider be accessed by the enterprise architecture?
     \item Which systems of a typical enterprise architecture are required?
     \item Which data objects of a typical enterprise architecture are required?
     \item Which additional systems and data objects are required?
     \item How can heterogeneous enterprise architectures be integrated the same way?
     \item How can the integration be non-invasive?
     \item How can the speed of integration development be increased?
     \item How can the speed of integration deployment be increased?
     \item How can the integration system be reliable and maintainable?
\end{enumerate}

The architecture is evaluated in respect to technological feasibility and real-life applicability. The results are incorporated into 
an operating manual, providing guidance in developing and deploying the presented integration architecture.

\chapter{Structure of Work}

ToDo: Briefly explain every Chapter and their relationships

\chapter{Digital Customer Self-Service (DCSS)}
This chapter introduces the concept of digital customer self-service. It defines DCSS, describes what it is by using examples and mentions benefits and challenges.

\textbf{Definition}: Digital self-service is the consumption of services provided by an enterprise over the internet without human to human interaction.

It is important to distinguish between "service" and "self-Service". The difference is shown through two examples:

\textbf{Example 1}: A service is the availability to purchase items whereas digital self-service is the order placement over a website.

\textbf{Example 2}: A service is the availability of video on demand whereas digital self-service is the access of video files over a website.

Digital self-service is characteristic one of many possible ways to access a service: An order placement could also be done through email 
or phone call and access to videos could be granted through same-day-delivery of DVDs.

Reasons for the widespread usage of digital self-service are significant benefits in ease of service usability for customers and cost reduction for enterprises.
Customers can self-service from almost every place, easily manage customer data like phone number and address, configure services like subscription plans and 
access lots of information on available products or trouble shooting steps in an immediate way.
Digital customer self-service improves usability, saves time, increases availability, saves money and therefore increases customer experience 
\cite[cf. 243]{CSSInternetAge}.
Enterprises, especially in progressive countries, aim to keep employment at a minimum, as it is a big cost factor. Digital self-service turns customers into 
unpaid employees \cite{served}.

Decreasing employment cost while increasing customer experience sounds like a win-win situation. Sadly, digital self-service does have its downsides.
One downside is most immediate to the elderly. In order to use DCSS, computers have to be available and one has to be experienced in their usage. 
Technological complexity, however, is not only a problem for customers but also for enterprises providing DCSS. Based on existing system architectures, 
adding DCSS functionalities can turn out to be a difficult task. Some form of integration is often necessary. With increased system complexity and business 
dependability come additional maintenance cost.

As most large enterprises provide DCSS, it can be assumed, that at a given size of an enterprise (measured in number of costumers), the benefits of DCSS 
surpass the disadvantages.

\chapter{Governmental Regulations}

\section{DSGVO}

\section{OZG}

\section{DVG}

\section{Challenges for Organizations}


\chapter{DCSS Scenarios}

ToDo: Explain DCSS scenarios resulting from top 10 OZG processes and DSGVO

\subsection{Arbeitslosengeld 2}

\subsection{BAf�G}

\subsection{F�hrerschein}

\subsection{Deletion of Personal Data}

\subsection{Correction of Personal Data}

\subsection{Information about collected Personal Data}

\section{Requirements}


\chapter{CSS Providers}

ToDo: Explain what CSS Providers are (what services they can provide -> what IDAS does) and how they can help creating DCSS. 
Also explain the concept of a business connector as technological communication point of the CSS provider

\section{Description}

\section{CSS Solutions}

\section{Business Connector}


\chapter{Enterprise Architectures}

ToDo: Introduce Enterprise Architectures and challanges such as heterogeneous systems when trying to model them

\section{Modeling of the real world}

\section{Modeling Challenges}

\section{Enterprise Architecture Patterns (EAP)}


\chapter{Relevant Systems and Data}

ToDo: Explain the approach of modeling enterprise architectures by EAPs and explain the used EAPs

\section{EAPs}

\section{Architecture Bricks}

\section{Data Bricks}

\section{Integration Requirements}

\subsection{Regarding Integration Architecture}

\subsection{Regarding Business Connector}


\chapter{Integration}

ToDo: Introduce the concept of integration. What are challenges in general and what would be an ideal integration.
Also explain the approach towards integration used (messaging, patterns)

\section{Definition}

\section{Requirements}

\subsection{Loose Coupling}

\subsection{Homogeneous Landscapes}

\section{Enterprise Integration Patterns (EIP)}

\subsection{Pattern 1}

\subsection{Pattern 2}


\chapter{Business Connector}

ToDo: Construct the model of the business connector

\section{Functionalities and Interfaces}

\section{Integration Requirements}

\subsection{Regarding Enterprise Architecture}

\subsection{Regarding Integration Architecture}

\section{Documentation}

\subsection{Connector as Architecture Brick}


\chapter{Integration Architecture}

ToDo: Create the integration architecture

\section{Scenario 1}

\subsection{Integration Documentation}

\subsection{System Integration}

\subsection{Data Integration}

\section{Scenario 2}

\subsection{Integration Documentation}

\subsection{System Integration}

\subsection{Data Integration}


\chapter{Integration Architecture Evaluation}

\section{Technology}

\section{Customer Example}

\section{Operating Manual}

\bibliographystyle{apalike}
\bibliography{references}

\end{document}
