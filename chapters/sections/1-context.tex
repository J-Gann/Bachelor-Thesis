Today, many services are accessible through the internet. Especially during the COVID-19 pandemic, stores had to switch to online trade due to lockdowns. Even after lockdowns, customers might continue to favour e-commerce \cite{9076858}. Online Service Providers (SP) often require customers to create user profiles to access their services. Management of personal information as well as interaction and communication between Service Providers and customers are often only possible through a web page or e-mail. It is time consuming to maintain even a small number of user profiles: To change personal information, for example, an e-mail address, the web page of each Service Provider has to be accessed, the correct credentials have to be submitted, and the information has to be manually filled in each time.

In some cases, user profiles are required by Service Providers even if they are not necessary. Especially in the case of small stores, running their individual online shops, user profiles are not necessary for customers to place an order. Personal information can be shared temporarily until processing of the order is completed and the package is delivered. Some online shops allow users to place orders by temporarily sharing personal information, however this limits interaction and communication possibilities to insecure e-mail exchanges.

A currently relevant topic with an interesting approach for identity management is the German Online Access Law (Online Zugangs Gesetz - OZG). It requires the German federal republic, each federal state and each commune to provide administration portals for digital access to administrative services \cite{ozg:general}. The federal republic and each member state operate their individual user profile systems accessible through their administration portals \cite{ozg:user_profiles_general}. 

The Online Access Law therefore constructs a system landscape that demonstrates the scenario of distributed partial identities on a small scale while maintaining the same problems as the private sector: Users have to create, maintain, and login into multiple user profiles to access administrative services distributed across administration portals. To solve this problem, a new solution for making OZG user profiles "interoperable" is in development \cite{ozg:user_profiles_general}. However, interoperable user profiles can not be accessed by the private sector.

Identity Management Provisioning (IMP) is an approach to identity management which aims to solve the problems of distributed partial identities by enabling users to bring their online identities with them when accessing the systems of Service Providers. It provides identity management as a service to users and Service Providers by enabling users to create and manage a single digital identity and use it to share personal information and interact with multiple Service Providers through "IMP relationships" and "IMP messages". With IMP relationships, users and Service Providers have a tool to securely share personal information which, depending on the use case, can be utilized in different ways. It can, for example, be used to create and manage a persistent SP user profile or to temporarily share personal information for the business process of a SP. IMP messages enable the user and Service Provider to securely exchange structured data and depending on the client application that receives and displays the data, design rich user interactions. This approach to identity management could be an alternative to user profiles in the private sector and in the context of the OZG.

Identity management systems do not stand alone but have to be accessible by various other systems in a system architecture to, for example, access personal information and perform authentication. They also play an important role in the design of business processes. Therefore, for Service Providers to utilize IMP, integration into existing business processes and existing system architectures is necessary.

