Finally, this chapter summarizes the most important results of this thesis and gives an outlook on possible future developments and contributions.

\section{Summary}

The bachelor thesis uses the scenario of the OZG as a small-scale model for the general problem of distributed partial identities. Focus is the utilization and system integration of IMP for Service Providers with existing user profile-based system architectures.

The first objective of the bachelor thesis is to examine the problems of user profile-based identity management and interoperable OZG user profiles and to describe to what extent IMP can be a solution. In chapter \ref{chapter:user_profile_identity_management} the thesis describes multiple problems of user profile-based identity management regarding usability, data protection and security. They lack usability as they lead to fragmentation into partial identities, accessible only through separate applications and web pages. Data protection is reduced due to the lack of oversight for the user about existing user profiles and shared personal information. The distribution of personal information to various Service Providers increases the risk of data being stolen. Especially in the case of the OZG, issues regarding data protection exist: \\
As a result of the "one-for-all" method, access to administrative services is distributed across multiple administration portals. Users applying for services therefore have to share large quantities of personal information with the administration portals of multiple federal states. As it is the institutions and not the administration portals that eventually process the personal information, sharing personal information directly with institutions would be favourable regarding data minimisation. \\
Possibilities for solving the mentioned problems through IMP are presented. IMP enables to increase usability by enabling users to interact, manage, and share personal information with Service Providers through only one application. Data protection is increased, as users can keep an overview of shared personal information. Security is improved, as IMP provides secure communication channels. In the case of the OZG, IMP is a possibility to increase usability and even eliminate the need for administration portals to process personal information, favouring data minimization. Based on the mentioned problems and solutions, the chapter concludes that IMP can be taken into consideration for solving problems of fragmented partial identities.

The bachelor thesis therefore moves to the second objective to design "IMP solutions" which describe the possibilities of how IMP services and tools can be specifically utilized on the example of - but not limited to - the OZG use case. The thesis presents two solutions for IMP utilization for Service Providers with existing user profile-based identity management. The first solution utilizes IMP in addition to existing user profiles. It provides an optional way for users to create and manage OZG user profiles through an IMP client. \\
This is done by enabling users to create new user profiles through existing IMP identities by establishing an IMP relationship. By exchanging IMP messages, users are able to access and manage their user profile. The IMP solution especially improves usability as OZG user profiles can be managed along with many other user profiles of other Service Providers through the same IMP client. The solution requires integration only with the administration portal component and can operate without disrupting the existing operation of the system architecture. 

The approach of enhancing accessibility to existing user profiles through an IMP client can be used to replace the need for OZG user profiles to be interoperable. Users can effortlessly create, manage, and login to user profiles for each federal state and administration portal. In addition to that, not only user profiles in the context of the OZG can be managed through IMP but user profiles of any integrated Service Provider. This IMP utilization therefore is a possibility to solve the usability problems of partial identities.

However, this IMP solution does not solve many of the data protection issues, as the user profile-based identity management system remains in operation. Therefore, the bachelor thesis presents a second IMP solution, with a focus on data protection.

The second solution utilizes IMP to replace user profiles. Instead of persistently sharing personal information with administration portals in the form of user profiles, IMP is used to temporarily share personal information with institutions for the period of time where an application is being processed. This is done by enabling users to establish different types of IMP relationships, each corresponding to a different application type. This IMP solution especially improves data protection, as only the institution responsible for processing the application receives personal information. In addition to that, sharing personal information only for a short period of time decreases the risk of personal information being misused or stolen.

The approach of replacing user profiles with temporary sharing of personal information for individual business processes can be used to eliminate partial identities while increasing data protection. However, while this approach might be usable in the context of the OZG, Service Providers in the private sector might not be interested in switching to this fundamentally different approach of identity management. Not as a result of technological problems but due to corporate policy reasons ("Never change a running system"). The advantage of IMP is: users can access Service Providers utilizing either IMP solutions through their IMP application.

The third objective of the bachelor thesis is to design a technological integration architecture for the integration of IMP solutions into existing system architectures of Service Providers. The thesis presents a modular messaging system, which enables the integration of IMP solutions into existing system architectures of Service Providers. Through the messaging system, the system architecture of the Service Provider can establish multiple types of IMP relationships and define, exchange, and process different message types. The messaging system maps relationships to entities that the SP systems understand. This can be, for example, the mapping of an IMP relationship to a user profile or to an application process. It also translates between the data models of IMP and Service Provider. As a result of using messaging and messaging modules, the integration system can be configured and extended to fit the requirements of different IMP solutions and system architectures of Service Providers.

The messaging system is shown, integrating both previously described IMP solutions into different system architectures. The first IMP solution was integrated into the administration portal component, the second IMP solution was integrated into the institution domain. This demonstrates that the messaging system is capable of integrating different IMP solutions into different system architectures.

\section{Outlook}

The bachelor thesis describes the integration of IMP on a theoretical level. Developing a prototype of the messaging system as part of an experimental evaluation can give further information about its feasibility, performance, and stability. The prototype can furthermore be used to test integration of the first IMP solution into the OZG system architecture of a federal state, to validate the functionality of the IMP solution and the messaging system, and to find missing features. Due to the modular design of the messaging system, new messaging modules implementing missing features can be added. 

To make the modular approach of the integration architecture more utilizable for Service Providers, a "Module Store" can be hosted, where enterprises can publish messaging modules for integration of enterprise systems they sell. Through an IMP tool, Service Providers could be enabled to generate a custom messaging system with messaging modules for selected systems. With "only" configuration of the messaging modules remaining, this could reduce the time required for implementing and deploying the integration architecture.

The thesis presents OZG as a Service Provider relevant for the integration of IMP.
Identity Management Provisioning, however, is relevant for many other Service Providers. For example, in the scope of the National Education Platform, "a digital portal is being created that in the future will support universities as well as national and international students who are seeking a stay abroad, a course of study, or a change of study location in Germany along their educational career through a digital offering that is interoperable across institutions" \cite{digitale_bildung}. In accordance with IMP solutions in the thesis, an IMP solution for this new use case can be designed and the presented messaging system can be configured and expanded to suit new requirements and to be used for system integration. 
