The bachelor thesis uses the scenario of the OZG as a small scale model for the general problem of distributed partial identities. Focus is the utilization and system integration of IMP for Service Providers with existing user profile based system architectures. A federal state, in context of the OZG, is used as an example for a Service Provider.

One objective of the bachelor thesis is to examine problems of user profile based identity management and interoperable OZG user profiles and describe to what extend IMP as an approach can be a solution.

Another objective of the bachelor thesis is to design "IMP solutions" which describe possibilities of how IMP services and tools can be concretely utilized on the example of - but not limited to - the OZG use case. IMP solutions demonstrate how IMP relationships and IMP messages can be utilized by users and Service Providers to share personal information and interact in order to solve problems of distributed partial identities related to usability, data protection and security.

Using OZG as an example, another objective of the bachelor thesis is to design a technological integration architecture for integration of IMP solutions into existing system architectures of Service Providers. Its integration capabilities, however, can not be limited to one IMP solution and one system architecture. On the contrary, it has to be capable of integrating various IMP solutions into different system architectures. The integration system therefore has to be configurable and expandable according to requirements of new IMP solutions and system architectures. To demonstrate the capabilities of the integration architecture in theory, IMP solutions presented as part of the thesis, have to be integrated into different system architectures in the context of the OZG.

IMP solutions and integration architecture presented in the bachelor thesis must not be tightly fitted to the scenario of the OZG but have to also be utilizable by Service Providers in other contexts.