Purpose of the thesis is to design an integration solution for identity management provisioning and identity management provisioning systems. 

In order to make the integration suitable to real life requirements, the OZG is selected as an example due to its current relevance and high requirements regarding identity management.

Two solutions for IMP integration and one message based IMPS integration architecture are presented.

The IMP solutions describe two important fundamentally different approaches of how service providers can utilize IMP for identity management. In the first solution IMP is used alongside existing user profiles, utilizing IMP as a way of improving existing user profile based identity management solutions without replacing them. This approach is especially important for service providers in the private sector which require integration with minimal risk and cost.

In the second solution, IMP is used as an alternative way for identity management without user profiles. This approach is especially important for service providers constructing new system architectures or have strong interest in data protection and data economy.

The message based IMPS integration architecture is designed to enable integration of multiple IMP solutions into different system architectures of service providers. This is demonstrated by presenting the integration of both IMP solutions contained in the thesis, each into a different system architecture.