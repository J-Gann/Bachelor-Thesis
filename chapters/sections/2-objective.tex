Identity management systems do not stand alone but have to be accessible by various other systems in a system architecture to for example access personal information and perform authentication. They also play an important role in the design of business processes. Therefore, for Service Providers to utilize IMP, integration into existing business processes and existing system architectures is necessary. 

A currently relevant topic with challenging requirements regarding identity management is the German Online Access Law (Online Zugangs Gesetz - OZG). It requires all administrative services of the German federal republic, each federal state and commune to be digitally available through interoperable user profiles. The current plan is to make user profiles only available for usage in context of the OZG. From a user perspective, this is yet another partial identity and user profile to manage.

One objective of the bachelor thesis is to design IMP solutions which enable utilization of IMP on the example of - but not limited to - the OZG use case. IMP solutions describe how IMP relationships and IMP messages can be utilized by users and Service Providers to share personal information and interact in order to improve, simplify, eliminate or replace existing business process. Besides usability and system complexity concerns, IMP solutions also make design decisions in regard to data protection and security.

Using OZG as an example, another objective of the bachelor thesis is to design a technological integration architecture for integration of IMP solutions into existing system architectures of Service Providers. Its integration capabilities should, however, not be limited to one IMP solution and one system architecture. On the contrary, it should be capable of integrating various IMP solutions into different system architectures. The integration system therefore should be configurable and expandable according to requirements of new IMP solutions and system architectures. To demonstrate the capabilities of the integration architecture, IMP solutions presented as part of the thesis, should be integrated into different system architectures in context of the OZG.