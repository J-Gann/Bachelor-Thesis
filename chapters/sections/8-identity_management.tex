This chapter describes the issues of user-based identity management in general and for the specific use case of the OZG. It also presents possibilities of how IMP can be utilized to solve the mentioned issues. At the end, the conclusion briefly states whether IMP can be considered for solving the problems of user profile-based identity management.

\section{Identity Management Issues} \label{section:identity_management_issues}

The OZG system architecture utilizes user profiles for identity management. This leads to several issues regarding usability, data protection, security, and maintenance costs. The following sections describe problems of user profiles in general as well as problems arising specifically in the context of the OZG.

\paragraph{Usability}
Each Service Provider has a different platform for the management of personal information and identity-related interactions. This can, for example, be web pages or applications for personal computers and smartphones. Keeping track of all created user profiles is difficult.

In the case of web pages, users will also have to remember various URLs, usernames, and passwords. The risk of forgetting login credentials is high.

To manage the personal information stored in each user profile and, for example, change an e-mail address, users have to access the individual platforms of each Service Provider. Depending on the amount of created profiles, this can be a time-consuming task. Especially if the user has to login to web pages using various authentication methods like passwords, single-sign-on, 2FA, or Magic-Link. Users receiving notifications on multiple platforms can quickly lose track and oversee important notifications.

Each platform is often presented with a different design, although the purpose remains similar. This makes it difficult for users to orientate themselves.

Bidirectional identity-related interactions like subscribing to a service, canceling a subscription, sending or receiving a security relevant mail often only takes place through a web page where e-mails are only used to notify the user to check his user profile ("You have unread messages!"). This is because e-mails do not provide a secure way of exchanging information and do not enable advanced interaction capabilities or exchange of structured content. Although Service Providers could develop mobile applications as an alternative for web pages and benefit from secure communication and advanced interaction capabilities, due to the required investments, smaller SPs might decide not to.

Service Providers rely on the validity of the information stored in the user profile. To guarantee that the information is correct, some SPs require users to take extra identification steps by, for example, performing a video call or requesting the user to show their ID at a local shop. As each Service Provider manages separate user profiles, the user might have to go through this time-consuming process repeatedly.

Users are not able to automatically share personal information from one user profile to another (except for some Single-Sign-On cases). As a result of that, creating a new user profile each time involves entering personal information all over again. 


In the case of the OZG, the usability of user profiles across multiple administration portals is limited: Although user profiles are interoperable, the administration portals are individually responsible for the application process. Applying for different administration services, users might have to visit separate administration portals. Each of the portals manages the application process differently. Active applications are not stored as part of the user profile but remain accessible only through the respective portals they were created through.\\ An overview of all active applications amongst all administration portals does not exist. The user has to go through each portal to find the application they, for example, want to cancel. Especially due to the „one-for-all“ approach users will have to access dedicated administration portals for certain types of administrative services. To mitigate this issue, the inbox of user profiles is used to notify the user of the status of all applications.

\paragraph{Data Protection}
Service Providers established a habit of collecting more personal information than necessary to provide their services. As a response, the "Datenschutz Grundverordnung" - DSGVO (General Data Protection Regulation) was passed. Amongst other things, it requires Service Providers to inform the user about which personal information is processed, the reason for processing the information, and how it is being processed. Conventional user profile systems which store all personal information but do not differentiate between possibly different processing legitimisation of attributes increase the risk of illegitimate data usage.

As a result of the multitude of user profiles and the variety of platforms to manage them, users can quickly loose track of which Service Provider processes which personal information. Multiple Service Providers storing sensible information increases the risk of personal information being stolen due to a data breach.


In the case of the OZG, the user shares personal information not only with the institutions responsible for processing an application but also with multiple systems operated by the member state. Especially the administration portal used to initiate the application is problematic as it receives personal information from the interoperable user profile, sends it to a form server, eventually receives the filled-in form and stores it. As a result of that, the administration portal has access to all personal information that in reality only the institution will eventually need to process. In addition, institutions actually do not require user profiles for processing applications. After an application is finished, the institution can delete all data stored as part of this process. The only reason user profiles are required is to verify the validity and authorization of incoming applications and to enable the user to comfortably fill in application forms and communicate through an inbox. 

\paragraph{Security}

User profiles are often secured through passwords. Especially for a large number of profiles, users are tempted to use identical passwords, increasing the risk of them being stolen and all user profiles being accessible. With the number of different Service Providers storing personal information and passwords, the risk of it being stolen increases as the security of system architectures differs between Service Providers.

Users also have to rely on the correct operation of the system architecture of the Service Provider and might not be able to prove failures in the operation of the system. If, for example, the user cancels a subscription but the system architecture quietly fails to process it, the user might have issues to provide proof.


\section{IMP Solution Opportunities} \label{section:imp_solution_opportunities}

Identity Management provisioning is capable of solving many of the issues described in the previous section.

\paragraph{Usability}

The goal of IMP solutions is to replace user profiles with a single IMP identity. Through the IMP client, a user can manage their IMP identity and all connected integrated Service Providers. Instead of creating user profiles, the user can establish a relationship between their IMP identity and the Service Provider. Depending on the use case, the relationship can be utilized differently. Using the IMP client, the user can keep track of all relationships without the need of remembering URLs, usernames, or passwords. It also enables the user to update personal information shared with Service Providers through one application. As the IMP client also contains an inbox, all notifications sent by each Service Provider can be accessed in the same place.

As part of established relationships, IMP clients and Service Providers can securely exchange bidirectional structured data, which along with the advanced interaction capabilities of the IMP client, can be used to create custom interaction scenarios: Service Providers can send structured data to the IMP client along with the information that it should display as a form. Depending on the content of the form and the use case of the Service Provider, the form can, for example, be a purchase agreement or a survey. 

Assuming that, through a secure identification and validation process, the identity management provider has made sure that the attributes users store as part of their IMP identity are correct, Service Providers wont require any additional identification steps through for example video calls.

Due to the IMP client being able to automatically fill in the attributes requested as part of a relationship, the user usually wont need to repeatedly manually enter personal information.

In the case of the OZG system architecture, IMP can be used by each member state to individually establish relationships with IMP identities of users to access personal information. This could replace the requirement for user profiles to be interoperable. For the most part, the member states would then be able to operate independently from each other. This solution is described in detail in chapter 4.

\paragraph{Data Protection}

IMP puts the users back in control of their personal information. \\
As part of the process of establishing a relationship, the Service Provider has to specify exactly which attributes have to be shared, give a detailed explanation of why the attributes are required to be shared, what the purpose of the relationship is and how the attributes are processed. The user can make an informed decision whether to establish the relationship and the Service Provider performs its duty to inform the user as required by the DSGVO.

As personal information is only shared as part of a relationship and with a specified purpose, the Service Provider has an effective tool to differentiate processing legitimisation of attributes based on the type of relationship they are accessible through. This helps the Service Provider in processing personal information according to the DSGVO.

As every relationship is manageable through the IMP client, the user wont loose track of which Service Provider has access to which personal information.

The risk of data being stolen decreases as most of the identity management and data storage is performed by the IMP system and the Service Provider processes only a small portion of the attributes of IMP identities.

In the case of the OZG system architecture, IMP can be used by individual institutions to establish relationships with IMP identities of users to retrieve applications for administrative services. This could eventually eliminate the need for user profiles. Personal information could be shared with only the responsible institution. This solution is described in detail in chapter 5.

\paragraph{Security}

The IMP client enables users to access their IMP identity and all relationships without the need of remembering a variety of credentials. This removes the risk of credentials being stolen. The IMP client itself can be secured without the need for a password. As the IMP system is capable of signing relationships and exchanged data, the interactions between IMP identity and Service Provider can be validated at any time. This enables both the user and Service Provider to provide proof for every interaction.

\section{Conclusion}

User profile based identity management especially in case of the OZG has several problems in regard to usability, data protection and security. As described, in many cases, IMP has the possibility of solving these issues and eliminating or at least enabling better control over fragmented partial identities. This is the case for interoperable OZG user profiles as well as user profiles in general. Therefore it makes sense to further investigate possibilities of utilizing IMP as a solution to partial identities. Using the small scale model of fragmented partial identities in context of the OZG, the following chapters 4 and 5 present different possibilities of utilizing IMP.