Based on the prominent example of the OZG, the bachelor thesis will construct a message based integration architecture which enables system architectures of service providers with existing user profiles to make their services usable with an Identity Management Provider. In order to make the integration architecture suitable to real life requirements, the OZG is selected as an example due to its current relevance and high requirements regarding identity management. The integration architecture is however not only applicable in cases relevant for OZG but also for other Service Providers with similar requirements. As requirements of SPs are not guaranteed to be similar to those of the OZG, the integration architecture is built to be expendable.

The \textbf{integration architecture} for the OZG enables \textbf{IMP functionalities} to be usable for selected \textbf{OZG scenarios} by integrating a \textbf{connector} in the \textbf{system architecture} of a member state.

\begin{itemize}
    \item OZG scenarios are based on documented digitalization efforts of administrative services in the context of the OZG and are for example the application for an administrative service or the communication with a governmental institution.
    \item The system architecture is based on documentation about architecture plans of member states.
    \item The functionalities of the IMP are based on a requirements analysis of OZG scenarios and additional research.
    \item The connector is defined based on functionalities the integration architecture requires from the IMP.
    \item The integration architecture is the main scientific contribution of this bachelor thesis. It is based on messaging patterns in order to utilize a combination of established and tested messaging solutions. In order to save investments it also reuses as many system components as possible and modifies as few system components as necessary.
\end{itemize}

The integration architecture ...