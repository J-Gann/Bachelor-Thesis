Since invention of the World Wide Web, the number of its users increased rapidly. Today, more than 90 percent of the german population uses the internet \cite{Onlinestudie}. Enterprises and organizations recognised this potential to provide their services to a large number of people as Service Providers (SP). A frequently used method of SPs is "online self-service" (OSS). Specialized software tools enable customers to use services through the internet without direct human interaction.

Many tools require identification of a user in case interactions or initiated business processes have to be associated with him. A system for management of identities of users therefore is part of numerous system architectures. Service Providers usually enable users to manage their partial identities using online self-service through user profiles. Today, customers of SPs are associated with multiple partial identities and user profiles - one for each SP.

The result is a variety of problems in management of digital identities \cite{IdentityCrisis}. Here are some examples:
\begin{itemize}
    \item Multiple SPs collect, use and share parts of identities which leads to distributed and fragmented identities.
    \item As a result of the distributed identities, risk of data breaches increases.
    \item Managing multiple user profiles is very inconvenient for users.
    \item It is often unclear who stores which data and for what purpose.
    \item SPs have an incentive to collect and share more user data than they need.
\end{itemize}

An improved model for identity management is necessary. In recent years, multiple new models have been invented. One model called Social Login became popular. It enables customers to create User Profiles for SPs through existing User Profiles of Social Networks like Facebook. This improves identity management by enabling SPs to rely on the external profile for unique identification, authorisation and personal attributes. Users can therefore manage identity related information through their Social Profile which is accessed by the SP.

This approach, however, still leaves the requirement of User Profiles for each SP because, depending on the provided service, additional attributes which are not part of the Social Profile, have to be manageable by the user. SPs also might require additional online self-service tools the social network does not provide.

The solution to this problem is an Identity Management Provider (IMP) which manages the whole online identity of a person and can replace the individual partial identities and user profiles. The system itself is provided and hosted by a SP for consumption by other SPs. To replace existing User Profiles, the IMP can not only rely on systems for registration, authentication, authorisation and provisioning. Additional requirements of SPs like communication, data wallet and management of additional attributes have to be satisfied.

In order for an SP to use identities and services of an IMP, an integration into the system architecture of the SP has to take place. Due to the complex nature of system architectures and identity management, this is a difficult task.

A currently relevant example for usage of online self-service with particular interesting requirements for identities is the German "Online Access Law" (Online Zugangs Gesetz: OZG). It requires all administrative services of the german federal republic, each member state and commune to be digitally available through interoperable user profiles. The current plan is to make the profiles only available for usage in context of the OZG. From a user perspective, this would be yet another partial identity and user profile to manage.