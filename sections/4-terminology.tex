
\paragraph{Service Provider}
Service providers (SP) are entities like enterprises or organisations which make services accessible over the internet.

\paragraph{Online Self-Service}
Online self-service (OSS) describes a method used by Service Providers to make their services available to the user. The method is characterized by being available over the internet, requiring no direct human interaction but instead enabling the user to access services in a self-reliant way. The user is usually supported by a variety of software solutions.

\paragraph{Self-Service Tools}
Self-service tools are software solutions for customers to enable a online self-service experience. Depending on the use case, the tools can be transactional or informing. Informing tools help users to retrieve relevant information, often replacing human customer support. Transactional tools help users to interact with services in a persistent way to for example save personal data or trigger business processes.

Information tools can be for example so called "WiKi" pages or search fields which assist the user in his process of finding information. Transactional tools can be for example online forms which assist the user in his process of triggering an order placement business process.

In contrast to information tools, transactional tools often require identification of users in order to associate them with changes they made to the system: If a user triggers the business process of ordering a product, the SP requires identification of the user.

\paragraph{Digital Identity}
The digital identity, for simplicity called identity in the thesis, is the sum of all digital personal information. Each piece of personal information is an attribute. Attributes can be identifiers which are able uniquely identify an entity. It is also possible to use an aggregation of attributes as identifier.

Identifiers which are commonly used by SPs are the phone number and E-Mail address. Attributes are for example nicknames, hobbies, interests or education. The full name of a person is often not sufficient unique identification, the aggregation of name, age and home address can therefore be used as identifier.

\paragraph{Partial Digital Identity}
A partial digital identity is a subset of an identity. Partial identities are commonly used by SPs to manage identity information which is relevant for them. 

\paragraph{Identity Management}
Identity management is the creation, utilization / reading , updating and deletion of identities or partial identities. The possible utilization of identities depends on the system for management of identities. It could be for example authorization for a service or communication through an inbox.

\paragraph{Identity Management System}
Identity management systems are solutions which enable SPs and customers to manage identities or partial identities.
They usually provide the following functionalities an properties:
\begin{itemize}
    \item 
\end{itemize}

\paragraph{User Profile}
User profiles are one method for identity management. They are characterized by each SP providing a separate identity management system which often consist of an identity provider (IdP) for storing identities, a customer relation management system (CRM) for accessing identities internally and a website with online self-service tools for providing access to customers.
