This section describes integration possibilities on the solution layer.
Although the eventual goal of an IMP integration is to replace user profiles, in this case, IMP identity and user profiles will be used together to solve issues described in chapter 3 without disrupting the operation of the existing system architecture. The advanced integration in chapter 5 will present a more invasive solution.

The presented solution enables users to access OZG services by connecting user profiles to existing IMP identities. Operations performed on the IMP identity will then be synchronized with the connected user profile.

\subsection{Create User Profile}
One way to connect user profile and IMP identity is by creating a new user profile based on an existing IMP identity. The web page of the administration portal responsible for guiding the user through the user profile creation steps can present an additional option of creating a user profile called "Login using IMP Identity". A relationship template can be displayed in form of a QR code with the following content:

\begin{itemize}
    \item Title: Create OZG User Profile
    \item Attributes: Name, Surname, ... 
    \item Shared Attributes: Administration Portal of the member state ...
    \item Reason: Create an OZG user profile and connect your IMP identity.
\end{itemize}

The user can scan the QR code using his IMP client and is presented with the relationship template. As part of the template, the user is requested to share attributes which thee IMP client can automatically fill in if they already exist. The attributes requested as part of the template have to be sufficient for creating an OZG user profile.
The user is also presented with an explanation for what the purpose of the relationship is and why the specified attributes are required. If the user accepts, a relationship request can be sent to the administration portal, where the attributes can be analyzed. Depending on the result of the analysis, the relationship can be accepted and therefore established or rejected.
Once the relationship is established, the OZG system architecture can create a new user profile based on the shared attributes and store the connection of user profile and IMP identity.

\subsection{Authentication}
Communication between IMP system and administration portal can assumed to be secure: Messages originating from an IMP identity for an IMP relationship can not be accessed or modified by a third party and are authorized by the corresponding user. Therefore messages exchanged as part of the established relationship can be used to provide the IMP identity access to OZG services. Some examples for this are shown in the following sections.

\subsection{Login to User Profile}
Messages trough the established IMP relationship can be used to enable the user to access OZG services through the existing web page, by using the IMP client for authentication. After receiving a login request for a user profile on the web page of the administration portal, an authentication request containing the session ID can be sent to the IMP relationship corresponding to the user profile. The IMP client can display an appropriate notification with options for the user to accept or reject the request. The IMP client sends the result back to the administration portal which according to the response authenticates the browser session or not.

\subsection{Attribute Synchronization}
Modifications of the personal information stored in the user profile can be done by exchanging messages trough the established IMP relationship. Through the IMP client, the user can send an attribute change request regarding the relationship. The system architecture can map the relationship to the connected user profile, evaluate the request and if accepted, perform the modification of personal information. It is possible to enable the user to update attributes of his user profile trough the web page which would require the OZG system to send attribute change requests to the IMP client. For simplicity, this option is left out and the profile management web page is expected to be inaccessible for user profiles connected to IMP.

\subsection{Communication}
Messages sent to the inbox of the user profile can be forwarded to the inbox of the IMP identity. As the exchange of messages through the IMP system is secure, also security relevant information can be shared.

\subsection{Application}
To minimize the invasiveness of the integration, applications for administrative services remain to be done on the administration portal trough user profiles.
