
\begin{itemize}
    \item The goal of this advanced integration is to take full advantage of IMP features
    
    \item The integration now focuses not only in the system architecture of the administration portal but also of the institutions
    
    \item A disadvantage of the previous integration solution is, that the administration portal maintains user profiles. This results in a duplication of user data management systems (IMP system and user profile system). In addition, trough the user profiles, the administration portal has access to personal information without the need for processing but routing it. This is a data protection issue.
    
    \item As institutions are the entities which eventually need to process personal information, personal information should only be shared with them.
    
    \item Using a IMP connector, institutions can establish relationships with users while the relationship represents the application for an administrative service.
    
    \item The institution does not maintain a identity management system in addition to the IMP system: An application is started being processed after the relationship is established and further communication regarding the application is done through the relationship. If the application is finished, the relationship is terminated, removing access to personal information for the institution.
    
    \item Institutions are capable of processing multiple administrative services. Each service has its own purpose, description and personal information required by the user. For each service, the institution processes, individual relationship templates can be created.
    
    \item In order for users to determine which institution is responsible for processing their request and for accessing the correct relationship template, the administration portal can be used as a "search-engine"
    
    \item The user searches for a administration service on the website of the administration portal and specifies the location he lives in. The administration portal determines the responsible institution and retrieves the correct relationship template from it.
    
    \item Based on the template, the user now interact only with the institution to apply for a service. The administration portal does not receive any further personal information.
    
    \item By requesting to change attributes of established relationships the institutions can process modifications of the request and decide if they accept or deny the request.
    
    \item As there is no standard for how the system of an institution operates, the integration can follow the integration of the administration portal by constructing a "form" from the shared personal information and enabling to deliver it in all kinds of formats through an adapter
    
    \item However, to further demonstrate the integration capabilities, the integration presents a possible integration for enabling application management through an E-Mail server on the institution side.
    
    \item Incoming applications can be sent as an E-Mail to employees assigned to a certain type of request. The relationship ID can be used as a reference ID in the title of the relationship or as a "spoofed" email sender. Further communication like attribute change requests can follow the same style and be aggregates by any email client to display all messages with the same reference ID.
    
    \item This method of applying for administrative services completely removes the need for administration portal and institution to maintain user profiles. It also enables users to share personal information only with entities which really need to process it. By enabling institution and user to directly interact, changes of the request content, status updates and exchange of mails can be done easily.
    
    \item Integration aufbauend auf erster oder ersetzend??
    
    \item Messaging architektur aus 1 genau so applizierbar in advanced integration
    
    \item im stil der overview der integrationen auch das connection diagram des OZG systems beschreiben. User Domain, Portal Domain, FIM Domain, Institution Domain, Data Exchange Domain
    
    \item basic use case umsetzen durch IMP trotz ablösen von 3 vorher existierenden domänen
    
    \item in diagrammen aufzeigen wo persönliche daten verarbeitet werden
    
    \item datenverabeitung vergleich zwischen diagrammen
    
    \item offen ob man andere domänen abschaltet (übergangszeit anlassen)
    
\end{itemize}