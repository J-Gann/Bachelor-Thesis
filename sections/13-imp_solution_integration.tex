The main issue of the IMP solution presented in chapter 4 is the utilization of user profiles and the resulting lack of data protection. The IMP solution described in this chapter puts the individual institutions in the center of the integration in order to eliminate the need for user profiles and the need for administration portals to process personal information.

Applications submitted to administration portals will eventually be submitted to responsible governmental institutions. Using IMP, relationships can be used to send applications directly to institutions without the need for an administration portal, form server or data exchange platform. Each institution has a predetermined set of administrative services it can process. For each administrative service, the institution can create relationship templates, containing a description of the service along with attributes required by the user:

\begin{itemize}
    \item Title: Apply for administrative service XYZ
    \item Attributes: Name, Surname, ... 
    \item Shared Attributes: Institution Name ...
    \item Reason: Apply for the administrative service XYZ ...
\end{itemize}

Based on these relationship templates, users and institutions can establish relationships. In contrast to the IMP solution of chapter 4, an established relationship does not result in the creation of a user profile. Instead, relationships represent a single application process and the attributes shared as part of the relationship are used by the institution to process the corresponding administrative service. After finishing the service, the relationship can be terminated, removing access to the shared attributes from the institution. 

While processing the administrative service, institution and user can securely and directly exchange data such as mails trough the IMP system without the need of an administration portal or user profile being involved.

Providing an attribute change feature as described in chapter 4, can enable users to request modification of attributes of an active application. The institution can then decide whether to accept or reject the request.

For users to access the relationship templates, each institution can host its own web page which lists all administration services along with the corresponding QR code containing the template.

However, depending on for example the home address of the user, the institution responsible for processing an administrative service can change. Therefore the administration portals connected trough the online gateway can be used as a "search-engine" for institutions. On any administration portal a user can select an administrative service and specify his home address. Based on this information, the portal is able to determine the institution responsible for processing the request. However, instead of requiring the user to login to a profile, the portal can redirect him to the web page of the institution where the relationship templates are being displayed.
