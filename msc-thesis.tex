% This template was initially provided by Dulip Withanage.
% Modifications for the database systems research group
% were made by Conny Junghans,  Jannik Strtgen and Michael Gertz

\documentclass[
     12pt,         % font size
     a4paper,      % paper format
     BCOR=10mm,version=first,     % binding correction
     DIV=14,version=first,        % stripe size for margin calculation
%     liststotoc,   % table listing in toc
%     bibtotoc,     % bibliography in toc
%     idxtotoc,     % index in toc
%     parskip       % paragraph skip instead of paragraph indent
     ]{scrreprt}

%%%%%%%%%%%%%%%%%%%%%%%%%%%%%%%%%%%%%%%%%%%%%%%%%%%%%%%%%%%%

% PACKAGES:

% Use German :
\usepackage[english]{babel}
% Input and font encoding
\usepackage[utf8]{inputenc}
\usepackage[T1]{fontenc}
% Index-generation
\usepackage{makeidx}
% Einbinden von URLs:
\usepackage{url}
% Special \LaTex symbols (e.g. \BibTeX):
%\usepackage{doc}
% Include Graphic-files:
\usepackage{graphicx}
% Include doc++ generated tex-files:
%\usepackage{docxx}
% Include PDF links
%\usepackage[pdftex, bookmarks=true]{hyperref}
\usepackage{csquotes}
\usepackage{color, colortbl, tabularx, ragged2e}
\definecolor{LightCyan}{rgb}{0.88,1,1}
\newcolumntype{C}{>{\raggedright\arraybackslash}X} % centered "X" column
% Fuer anderthalbzeiligen Textsatz
\usepackage{setspace}
\usepackage{multirow}
\usepackage{longtable}
\usepackage{float}
\usepackage{wrapfig}
\usepackage{subfiles} % Best loaded last in the preamble
\usepackage{pdfpages}

% hyperrefs in the documents
\usepackage[bookmarks=true,colorlinks,pdfpagelabels,pdfstartview = FitH,bookmarksopen = true,bookmarksnumbered = true,linkcolor = black,plainpages = false,hypertexnames = false,citecolor = black,urlcolor=black]{hyperref} 
%\usepackage{hyperref}


%%%%%%%%%%%%%%%%%%%%%%%%%%%%%%%%%%%%%%%%%%%%%%%%%%%%%%%%%%%%

% OTHER SETTINGS:

% Pagestyle:
\pagestyle{headings}

% Choose language
\newcommand{\setlang}[1]{\selectlanguage{#1}\nonfrenchspacing}

\usepackage{biblatex}
\addbibresource{references.bib}

\begin{document}

% TITLE:
\pagenumbering{roman}
\begin{titlepage}
     \vspace*{1cm}
     \begin{center}
          \vspace*{3cm}
          \textbf
          {
               \Large Ruprecht-Karls-Universität Heidelberg\\
               \smallskip
               \Large Institut für Informatik\\
               \smallskip
               \Large Lehrstuhl für Datenbanksysteme\\
               \smallskip
          }

          \vspace{3cm}

          \textbf{\large Bachelor Arbeit}

          \vspace{0.5\baselineskip}
          {
               \huge
               \textbf{Integrating Identity Management Providers based on Online Zugangs Gesetz}
          }

     \end{center}

     \vfill
     {
          \large
          \begin{tabular}[l]{ll}
               Name:                 & Jonas Gann              \\
               Matrikelnummer: & 3367576                 \\
               Betreuer:           & Prof. Dr. Michael Gertz \\
               Datum der Abgabe:   & \today
          \end{tabular}
     }

\end{titlepage}

\onehalfspacing

\thispagestyle{empty}

\vspace*{100pt}
\noindent
Ich versichere, dass ich diese Bachelorarbeit selbstständig verfasst und nur die angegebenen Quellen und Hilfsmittel verwendet habe.

\vspace*{50pt}
\noindent

\underline{\phantom{mmmmmmmmmmmmmmmmmmmm}}

\medskip
\noindent
Date of Submission: \today
\newpage

\chapter*{Zusammenfassung}

\newpage

\chapter*{Abstract}

\newpage

\tableofcontents
\cleardoublepage
\pagenumbering{arabic}

\chapter{Introduction}

\section{Context}

\subfile{sections/1-context}

\section{Objective}

\subfile{sections/2-objective}

\section{Structure of Work}

\subfile{sections/3-structure_of_work}

\chapter{Background and Related Work}

\section{Terminology}

\subfile{sections/4-terminology}

\section{Identity Management Provisioning (IMP)}

\subfile{sections/5-imp}

\section{Online Access Law (OZG)}

\subfile{sections/6-online_access_law}

\section{Messaging}

\subfile{sections/7-messaging}

\chapter{OZG Identity Management}

\subfile{sections/8-identity_management}

\chapter{IMP Utilization in OZG}

\subfile{sections/9-imp_utilization_in_ozg}

\chapter{Advanced Utilization}

\subfile{sections/12-advanced_utilization}

\chapter{Conclusion}

\begin{itemize}
    \item Mögliche IMP Anbieter finden und aufführen
\end{itemize}

\chapter{Outlook}

\printbibliography

\end{document}
