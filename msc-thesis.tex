% This template was initially provided by Dulip Withanage.
% Modifications for the database systems research group
% were made by Conny Junghans,  Jannik Strtgen and Michael Gertz

\documentclass[
     12pt,         % font size
     a4paper,      % paper format
     BCOR=10mm,version=first,     % binding correction
     DIV=14,version=first,        % stripe size for margin calculation
%     liststotoc,   % table listing in toc
%     bibtotoc,     % bibliography in toc
%     idxtotoc,     % index in toc
%     parskip       % paragraph skip instad of paragraph indent
     ]{scrreprt}

%%%%%%%%%%%%%%%%%%%%%%%%%%%%%%%%%%%%%%%%%%%%%%%%%%%%%%%%%%%%

% PACKAGES:

% Use German :
\usepackage[english]{babel}
% Input and font encoding
\usepackage[utf8]{inputenc}
\usepackage[T1]{fontenc}
% Index-generation
\usepackage{makeidx}
% Einbinden von URLs:
\usepackage{url}
% Special \LaTex symbols (e.g. \BibTeX):
%\usepackage{doc}
% Include Graphic-files:
\usepackage{graphicx}
% Include doc++ generated tex-files:
%\usepackage{docxx}
% Include PDF links
%\usepackage[pdftex, bookmarks=true]{hyperref}
\usepackage{csquotes}
\usepackage{color, colortbl, tabularx, ragged2e}
\definecolor{LightCyan}{rgb}{0.88,1,1}
\newcolumntype{C}{>{\raggedright\arraybackslash}X} % centered "X" column
% Fuer anderthalbzeiligen Textsatz
\usepackage{setspace}
\usepackage{multirow}
\usepackage{longtable}
\usepackage{float}

% hyperrefs in the documents
\usepackage[bookmarks=true,colorlinks,pdfpagelabels,pdfstartview = FitH,bookmarksopen = true,bookmarksnumbered = true,linkcolor = black,plainpages = false,hypertexnames = false,citecolor = black,urlcolor=black]{hyperref} 
%\usepackage{hyperref}


%%%%%%%%%%%%%%%%%%%%%%%%%%%%%%%%%%%%%%%%%%%%%%%%%%%%%%%%%%%%

% OTHER SETTINGS:

% Pagestyle:
\pagestyle{headings}

% Choose language
\newcommand{\setlang}[1]{\selectlanguage{#1}\nonfrenchspacing}

\usepackage{biblatex}
\addbibresource{references.bib}

\begin{document}

% TITLE:
\pagenumbering{roman}
\begin{titlepage}
     \vspace*{1cm}
     \begin{center}
          \vspace*{3cm}
          \textbf
          {
               \Large University of Heidelberg\\
               \smallskip
               \Large Institute for Computer Science\\
               \smallskip
               \Large Working group database systems\\
               \smallskip
          }

          \vspace{3cm}

          \textbf{\large Bachelor thesis}

          \vspace{0.5\baselineskip}
          {
               \huge
               \textbf{Integrating Identity Management Providers based on Online Zugangs Gesetz}
          }

     \end{center}

     \vfill
     {
          \large
          \begin{tabular}[l]{ll}
               Name:                 & Jonas Gann              \\
               Matriculation number: & 3367576                 \\
               Supervisor:           & Prof. Dr. Michael Gertz \\
               Date of submission:   & \today
          \end{tabular}
     }

\end{titlepage}

\onehalfspacing

\thispagestyle{empty}

\vspace*{100pt}
\noindent
I assure that I have written this bachelor thesis on my own and only used the specified sources and resources and that I followed the principles and recommendations "Responsibility in Science" of the University of Heidelberg.

\vspace*{50pt}
\noindent

\underline{\phantom{mmmmmmmmmmmmmmmmmmmm}}

\medskip
\noindent
Date of Submission: \today
\newpage

\chapter*{Zusammenfassung}

\newpage

\chapter*{Abstract}

\newpage

\tableofcontents
\cleardoublepage
\pagenumbering{arabic}

\chapter{Introduction}

\section{Context}
Since invention of the World Wide Web, the number of its users increased rapidly. Today, more than 90 percent of the German population use the internet \cite{Onlinestudie}. Enterprises and organizations, recognised this potential to provide their services to a large number of people as Service Providers (SP). A frequently used method of SPs is "online self-service" (OSS). Specialized software tools enable customers to use services through the internet without direct human interaction.

Many tools require identification of a user in case interactions or initiated business processes have to be associated with him. A system for management of identities of users therefore is part of numerous system architectures. Service Providers usually enable users to manage their partial identities using online self-service through so called User Profiles. Today, customers of SPs are associated with multiple partial identities and user profiles - one for each SP

The result is variety of problems in management of digital identities \cite{IdentityCrisis}. Here are some examples:
\begin{itemize}
    \item Multiple SPs collect, use and share parts of identities which leads to distributed and fragmented identities.
    \item As a result of the distributed identities, risk of data breaches increases.
    \item Managing multiple use profiles is very inconvenient for users.
    \item It is often unclear who stores which data and for what purpose.
    \item SPs have an incentive to collect and share more user data than they need.
\end{itemize}

An improved model for identity management is necessary. In recent years, multiple new models have been invented. One model called Social Login became popular. It enables customers to create User Profiles for SPs through existing User Profile of Social Networks like Facebook. This improves identity management by enabling SPs to rely on the external profile for unique identification, authorisation and personal attributes. Users can therefore manage identity related information through their Social Profile which is accessed by the SP.

This approach, however, still leaves the requirement of User Profiles for each SP because, depending on the provided service, additional attributes which are not part of the Social Profile, have to be manageable by the user. SPs also might require additional online self-service tools the social network does not provide.

The solution to this problem is an Identity Management Provider (IMP) which manages the whole online identity of a person and can replace the individual partial identities and User Profiles. The system itself is provided and hosted by a SP for consumption by other SPs. To replace existing User Profiles, the IMP can not only rely on systems for registration, authentication, authorisation and provisioning. Additional requirements of SPs like communication, data wallet and management of additional attributes have to be satisfied.

In order for an SP to use identities and services of an IMP, an integration into the system architecture of the SP has to take place. Due to the complex nature of system architectures and identity management, this is a difficult task.

A currently relevant example for usage of online self-service with particular interesting requirements for identities is the German "Online Access Law" (Online Zugangs Gesetz: OZG). It requires all administrative services of the German federal republic, each member state and commune to be digitally available through interoperable user profiles. The current plan is to make the profiles only available for usage in context of the OZG. From a user perspective, this would be yet another partial identity and user profile to manage.

\section{Objective}
Based on the prominent example of the OZG, the bachelor thesis will construct a message based integration architecture which enables system architectures of service providers with existing user profiles to make their services usable with an Identity Management Provider. In order to make the integration architecture suitable to real life requirements, the OZG is selected as an example due to its current relevance and high requirements regarding identity management. The integration architecture is however not only applicable in cases relevant for OZG but also for other Service Providers with similar requirements. As requirements of SPs are not guaranteed to be similar to those of the OZG, the integration architecture is built to be expendable.

The \textbf{integration architecture} for the OZG enables \textbf{IMP functionalities} to be usable for selected \textbf{OZG scenarios} by integrating a \textbf{connector} in the \textbf{system architecture} of a member state.

\begin{itemize}
    \item OZG scenarios are based on documented digitalization efforts of administrative services in the context of the OZG and are for example the application for an administrative service or the communication with a governmental institution.
    \item The system architecture is based on documentation about architecture plans of member states.
    \item The functionalities of the IMP are based on a requirements analysis of OZG scenarios and additional research.
    \item The connector is defined based on functionalities the integration architecture requires from the IMP.
    \item The integration architecture is the main scientific contribution of this bachelor thesis. It is based on messaging patterns in order to utilize a combination of established and tested messaging solutions. In order to save investments it also reuses as many system components as possible and modifies as few system components as necessary.
\end{itemize}

The integration architecture ...

\section{Structure of Work}
The "Fundamentals" chapter explains terminology and basic concepts.

In the following chapter called "Identity Management System" the thesis analyzes requirements for identity management through the perspective of the user and through the perspective of service providers for the OZG. It is presented, how an identity management provider could solve current identity management problems and which challenges for an integration exist.

The chapter "Integration Proposal" first documents a possible IMP system and connector, based on the requirement analysis of the previous chapter. It then presents the integration architecture with iteratively increasing requirements through textual description, multiple flow diagrams and messaging architectures.

A "Solution Evaluation" analyzes benefits and problems of the proposed solution.

The "Outlook" will conclude the thesis by presenting possibilities for advanced integration architectures.

\chapter{Background and Related Work}

\section{Terminology}

\paragraph{Service Provider}
Service providers (SP) are entities like enterprises or organisations which make services accessible over the internet.

\paragraph{Online Self-Service}
Online self-service (OSS) describes a method used by Service Providers to make their services available to the user. The method is characterized by being available over the internet, requiring no direct human interaction but instead enabling the user to access services in a self-reliant way. The user is usually supported by a variety of software solutions.

\paragraph{Self-Service Tools}
Self-service tools are software solutions for customers to enable a online self-service experience. Depending on the use case, the tools can be transactional or informing. Informing tools help users to retrieve relevant information, often replacing human customer support. Transactional tools help users to interact with services in a persistent way to for example save personal data or trigger business processes.

Information tools can be for example so called "WiKi" pages or search fields which assist the user in his process of finding information. Transactional tools can be for example online forms which assist the user in his process of triggering an order placement business process.

In contrast to information tools, transactional tools often require identification of users in order to associate them with changes they made to the system: If a user triggers the business process of ordering a product, the SP requires identification of the user.

\paragraph{Digital Identity}
The digital identity, for simplicity called identity in the thesis, is the sum of all digital personal information. Each piece of personal information is an attribute. Attributes can be identifiers which are able uniquely identify an entity. It is also possible to use an aggregation of attributes as identifier.

Identifiers which are commonly used by SPs are the phone number and E-Mail address. Attributes are for example nicknames, hobbies, interests or education. The full name of a person is often not sufficient unique identification, the aggregation of name, age and home address can therefore be used as identifier.

\paragraph{Partial Digital Identity}
A partial digital identity is a subset of an identity. Partial identities are commonly used by SPs to manage identity information which is relevant for them. 

\paragraph{Identity Management}
Identity management is the creation, utilization / reading , updating and deletion of identities or partial identities. The possible utilization of identities depends on the system for management of identities. It could be for example authorization for a service or communication through an inbox.

\paragraph{Identity Management System}
Identity management systems are solutions which enable SPs and customers to manage identities or partial identities.
They usually provide the following functionalities an properties:
\begin{itemize}
    \item 
\end{itemize}

\paragraph{User Profile}
User profiles are one method for identity management. They are characterized by each SP providing a separate identity management system which often consist of an identity provider (IdP) for storing identities, a customer relation management system (CRM) for accessing identities internally and a website with online self-service tools for providing access to customers.

\section{Identity Management Provisioning (IMP)}
Identity management provisioning (IMP) is a method for identity management which is characterized by an identity management provider hosting an identity management system for "non-partial" identities. The identity management system provides one-stop online self-service for users and manages the storage of identities. SPs are provided access to the identities of the identity management system.

The following are functionalities of an IMP system in addition to those of conventional identity management systems:

\paragraph{Attribute Management}
It is common for identity management systems to enable users to manage a predetermined set of attributes. IMP systems however do not just manage partial identities and therefore enable users to manage any possible attribute of their identity. Users therefore can add, remove and modify any attribute.

As often, attributes of a digital identity are defined by service providers, they are also able to create new attributes or delete and modify existing ones, if approved by the user. 

\paragraph{Online Self-Service}
Many attributes, especially those created by SPs, do not make sense to the user without context and my be heavily dependant on systems of a SP. Therefore, the IMP provides solutions for online self-service which supports the management of attributes.

\paragraph{Integration}
Service providers which use IMP systems as solution for identity management require access to its services and have to make their existing systems work with them. The IMP solution therefore provides interfaces which enable the SP to access services. As it is in the best interest of the IMP solution to integrate SPs, advanced integration methods reducing the integration effort for SPs can be provided. 

\section{Online Access Law (OZG)}
In 2017 the German government passed the law for improvement of online access to administrative services (Online Zugangs Gesetz: OZG) which requires federal republic and member states to execute the following regulations until 2022 \cite{BMI:OZG_Wortlaut}:
\begin{enumerate}
    \item \textbf{Digital availability of administrative services} \\
    An administrative service is the electronic processing of administrative procedures which are available from outside the governmental institution.  As it is not clear which administrative services exactly are meant by the definition of the OZG, the BMI created a catalogue \cite{BMI:Verwaltungsleistungen}. The OZG requires these services to be digitally available. As a guideline to what is considered sufficient availability, the BMI defined a maturity model \cite{BMI:Digitale_Services}.
    \item \textbf{Digital access to administrative services through administration portals of a portal network} \\
    Federal republic, each member state and each commune must provide an administration portal. Portals of communes must be linked to the portal of the corresponding member state. Portals of federal republic and member states must be connected through a portal network. \cite{BMI:Portalverbund} Each portal must provide a "seek and find" feature, which enables users to find all administrative services provided by any administration portal \cite{Cotar:Drucksache_19/19089}. 
    \item \textbf{Interoperable user profiles for accessing administrative services} \\
    Federal republic and member states must provide user profiles which can be used to identify the corresponding person while requesting access to administrative services, to save personal information according to the once-only principle, to receive and send messages via a digital mailbox and to pay for services \cite{Cotar:Drucksache_19/19089}. The user profiles must be interoperable for every administration portal of the portal network.
\end{enumerate}

Execution of the OZG can be separated into two projects:

\paragraph{Digitalization}
Digitalization of administrative services focuses on accessibility towards users but not modernization of its internal execution by administrative institutions. The goal of digitalization in this case is, to make administrative services available towards a user in a digitized way:

A digitized administrative service can be accessed by a user through a website. The website is hosted either by the federal republic or a member state and is called "Administration Portal". Access is usually provided through an application form. The administrative service can be managed through a user profile which is provided by the federal republic or the member state. Management of an administrative service usually includes starting the service by sending in a form, communicating with responsible institutions through the inbox of the profile and receiving a result. The user profile can also enable users to upload documents to a data wallet and to save personal information for automatically filling in forms.

Digitalization of an administrative service is the modification of the underlying processes to incorporate the usage of the described features of the user profile and administration portal. In total, the BMI lists 575 relevant services, some of them provided by the federal republic, some by the member states and yet other by the communes \cite{BMI:Onlinezugangsgesetz}.

\paragraph{Networking}
The networking focuses on connecting governmental systems to make all digitized administrative services available for every user. This includes most importantly the connection of administration portals to a portal network through an online gateway and the interoperability of user profiles.

In order to save investments a method called "one for all" is used when hosting administrative services. One member state or the federal republic provide access to a service on their administration portal and distribute the requests to the responsible institutions "under the hood". As administration portals are connected through an "online gateway", each portal contains a search feature, which enables users to find all administrative services through any portal. Interoperable user profiles enable usage of each profile for management of administrative services on all portals.

\subsection{Basic Use Case}
The BMI lists a total of almost 600 administrative services, each being described by a different process \cite{BMI:Informatiosplattform}. Each process consists of multiple sub-processes. When comparing the processes, one can see, that some sub-processes occur very often and in similar arrangements. Those arrangements can be called the basic use case of OZG services.

The basic use case describes the submission of an application for an administrative service and consists of the following arrangement of sub-processes \cite{NRW:Umsetzung}:

\paragraph{1. Create User Profile}
The user creates a user profile on the administration portal of a member state by entering personal information and a authentication method like username and password or the German digital identity card.

\paragraph{2. Login to User Profile}
Independent of the member state the user created the interoperable user profile for, he can use it in the current administration portal. The user has to finish authentication steps, which can consist for example of a username and password. Depending on the authentication method, a trust level of the current session is determined. With approval from the user, personal information from his profile is provided towards the form-server.

\paragraph{3. Selection of Administrative Service}
The user visits any administration portal of the portal network and types a search term in the corresponding search field. The user is then presented with a list of search results consisting of administrative services. If the user selects a search result, he is forwarded to the correct page inside the domain of the administration portal which hosts the selected service. This page provides information about the administration service along with instructions on how to access the form for application. The interactive form, usually provided by a form-server, can be for example included in the page through a web-component.

\paragraph{4. Filling in Application}
Personal information from the user profile is used by the form-server to fill as many fields of the form as possible. The form with prefilled information is then presented to the user. The interactive form can then be modified by the user. Depending on the form-server, additional functionalities like conditions between fields or sanity checking can be provided.

\paragraph{5. Submission of Application}
The user again authenticates for his user profile and authorises the form-server to submit the application. The form-server saves the application until he is notified to delete it. The form-server submits the application to the administration portal which hosts the administrative service.

\paragraph{6. Reception of Application by Administration Portal}
The administration portal receives the application from a form-server and usually just forwards it to the data-exchange platform and notifies the form-server that the he can delete the application.

\paragraph{7. Submission of Application to Data-Exchange Platform}
The data-exchange platform distributes the application to the correct administrative institution responsible for processing it.

\paragraph{8. Management of Applications}
The administration portal stores active applications and enables the user to manage them.

\paragraph{9. Communication}
As part of an active application, the responsible institution can respond to the inbox of the user with status updates, requests or results.  Through the active application, he user can send messages to the responsible institution to request a status update or send additional information and documents.

\subsection{System Architecture}
This subsection describes the system architecture of a member state relevant for the execution the basic use case of the OZG based on the system architecture of Nordrhein-Westfalen. Systems, which are responsible for coordination of OZG execution between member states and the federal republic are not the focus.

\subsubsection{System Components}
The system architecture consists of the following system components. Each component consists of a list of services it provides either to the user or to other components. Components are defined based on strongly connected categories of services \cite{NRW:Umsetzung}:

\begin{figure}[h]
\caption{Overview IMP System}
    \centering
    \includegraphics[scale=0.25]{Diagrams/OZG System Overview.png}
\end{figure}

\paragraph{Administration Portal}
The administration portal is the component which directly interacts with users through a web site an provides the basic OZG use case through interaction with other system components. Through the web page of the portal, the user can create and manage his user profile, receive and send messages, upload and download documents, find administrative services, select an application form for an administrative service and be forwarded to the respective form server. Without direct visibility for the user, the portal also transmits personal information from the user profile to the form server, receives finished applications from the form server, determines responsible institutions and submits applications to the data exchange platform.
The administration portal consists of a web server, responsible for hosting web pages, and a facade for service components. The most important components, the service facade provides an interface for are:

\begin{itemize}
    \item \textbf{Responsibility Finder}
    
    This service component is used to find administrative institutions which are responsible for an application. Depending on the type of application and on the home address of the user, different institutions might be responsible.
    
    \item \textbf{Process Management}
    
    This service component manages when the administration portal has to start which process in order to provide the basic OZG use case described earlier.
    
    \item \textbf{Search}
    
    This service component creates a list of URLs of administrative services relevant for a specified search term. This enables the portal to provide a search functionality of administrative services for the user.
    
    \item \textbf{User Profile Management}
    
    This service component enables the administration portal to interact with the user profile component to retrieve JSON Web Tokens, update and read personal information and document wallet, to read messages of the inbox and sent messages to an institution.
    
    \item \textbf{Application Management}
    
    This service component manages applications issued through the administration portal through communication with the form server. The component enables the user to start an application by forwarding him to the correct form on the form server and optionally providing personal information to the form server. Form server and portal communicate about the status and content of the application. The user is able to keep track about the status of active applications through the profile, view their content, send messages to responsible institutions and cancel them.
    
\end{itemize}

\paragraph{Form-Server}
The from-server provides digital forms associated to administrative services. They can be accessed by a user profile through online self-service and submitted as applications to administrative institutions. The main services are \cite{dNRW:Standardisierungskonzeptzur}

\begin{itemize}
    \item \textbf{Retrieval of a digital form}
    
    There exist many administrative services and depending on the member state or commune, the required layout of the application can be different. Therefore, a system called federal information management (Föderales Informationsmanagement: FIM) is used to standardize administrative services. FIM consists of three categories of building blocks.
    
    \textit{Service blocks} contain human readable descriptions about a administrative service.
    
    \textit{Data field blocks} contain the standardized description of data fields required for the application of administrative services.
    
    \textit{Process blocks} contain standardized descriptions about the process of an administrative service.

    Relevant for the form-server are data field blocks. A data field block can contain one element of five categories.
    
    \textit{Data fields} are the "smallest entity" of a data field block and describes one standardized piece of information. Depending on the type of information - if it is for example a checkbox or input field - additional metadata is included.
    
    \textit{Data field groups} consist of multiple data fields and other data field groups, relating to a category of information. A data field group can for example be "person" or "company".
    
    \textit{Rules} describe all kinds of logical conditions of and between data fields. This includes for example the automatic validation of the correctness of an entered value or the activation and deactivation of data fields depending on an entered value.
    
    \textit{Code lists} are lists of predefined values the user can select. This can for example be a list of all countries.
    
    \textit{Data field schemata} is the combination of entities from all previously described categories and describes the structure of a form.
    
    The building blocks are centrally managed by federal republic, member states and communes. This simplifies the creation of for example a new form by reusing existing data schema blocks and adding additional required data fields to a new data field schema.
    
    Each data field block can be uniquely identified through an ID. Therefore, if the form-server is provided with a FIM-ID, he can retrieve the corresponding data filed block from the central storage.
    
    \item \textbf{Initiation of application}
    
    The from-server can be requested to initiate an application using a certain form based on a FIM ID of a data field schemata. Along with the request, an identification and authorisation of a user profile has to be passed. Additional personal information of the user profile can be passed for automated filling in of the form.
    
    \item \textbf{Status update}
    
    The form-server sends updates regarding the current status of the application.

    \item \textbf{Automatic filling in of form}
    
    If the form server was provided with personal information during the initiation request for an application, it can automatically fill in the form.
    
    \item \textbf{Presentation of form}
    
    The form server can host a website where it displays an interactive form. The user can access this website either through an URL or through a web component.
    
    \item \textbf{Application interruption}
    
    The application can be interrupted or aborted. Interruption may occur if the user stays inactive for a certain period of time while filling in the form. The user can also manually interrupt the editing process. If the application is interrupted, the form-server stores the unfinished application for a certain amount of time. The user can then access the application at a later time.
    
    If the user decides to cancel editing the form, the application is deleted from the from-server.
    
    Depending on the actions of the user, the status of the application changes accordingly and the form server sends an status update.
    
    \item \textbf{Storage of application}
    The form-server stores applications for a specified period of time if they were interrupted or submitted.
    
    \item \textbf{Provisioning stored applications}
    
    If the form server is provided an identification and authorisation of a user, it searches all stored applications which belong to the user and sends back a list of URLs, through which the user can access them.
    
    \item \textbf{Deletion of stored application}

    If the form server is provided an identification and authorisation of a user and an identification of an application, it can delete the stored application.

    \item \textbf{Manually filling in data fields}
    
    When the form is presented to the user, it can happen, that not all information could be automatically filled in. In this case, the form can be interactively filled in by the user.
    
    \item \textbf{Uploading documents}
    
    The user can add documents as attachments to an application. The documents can either be uploaded from the local machine or the data wallet of the user profile.

    \item \textbf{Submission of application}
    
    When the user submits the application by for example pressing a "submit" button, the form-server sends a status update. The finished application can now be transmitted to a requesting system.

\end{itemize}

\paragraph{User Profile}
Each member state provides its own user profile for identity management. The main services are \cite{NRW:Umsetzung}:

\begin{itemize}
    \item \textbf{Identification and Authentification} \cite{dNRW:Anbindungsleitfaden}
    
    The user profile can be used for identification by transmitting personal information like name and address after successfully authentification. Authentification is the verification of an authentication. Users can authenticate for the user profile for example through a username / password combination.
    
    \item \textbf{Determination Trust Level} \cite{dNRW:Anbindungsleitfaden}
    
    The trust level of a logged in user profile is determined while registration and usage. The user can register and login to the user profile using a username / password combination or the German ID card. The trust level "High" is only granted, if the user registered using the German ID card \textbf{AND} logs in using the German ID card. All other combinations of registration and login result in a "normal" trust level.
    
    \item \textbf{Management Personal Information} \cite{dNRW:Anbindungsleitfaden} 
    
    Personal information of the user profile can be modified. This requires the user to authenticate. In case the user profile is connected with an online identity card, some attributes like name and age can not be changed.
    
    \item \textbf{Provisioning Personal Information} \cite{dNRW:Anbindungsleitfaden} \cite{dNRW:Schnittstellen}

    The user profile can be requested to send personal information to a specified URL. This requires the user to authenticate.
    
\end{itemize}

\paragraph{Institution}
Institutions are the entities which eventually process the incoming applications and provide the users with solutions. They receive a digital application through the data-exchange platform and send back the result as a message through the inbox.

\paragraph{Data-Exchange Platform}
The data-exchange platform delivers applications from the administration portal to the correct administrative institution.

\paragraph{Data Wallet}
The data wallet is a sub-component of the user profile and enables the user to manage documents for his user profile. The main services are:

\begin{itemize}

    \item \textbf{Upload of document}
    
    The user can upload a document from his local machine to the data wallet.
    
    \item \textbf{Show documents}
    
    The user can list all documents which are stored in the data wallet.
    
    \item \textbf{Delete documents}
    
    The user can delete documents from the data wallet.
    
    \item \textbf{Attach documents to application}
    
    The user can attach a copy of a document stored in his data wallet to an application on a from server.

\end{itemize}

\paragraph{Inbox}
The inbox is a sub-component of the user profile and enables the user to send messages through his user profile. The main services are:
    
\begin{itemize}
    
    \item \textbf{Send a message}
    
    The user can send a message to authorities. This can be for example an institution which processes an application.
    
    \item \textbf{Receive a message}
    
    The user can receive messages from authorities for example to update the status of an application.
    
    \item \textbf{List messages}
    
    The user can see the messages he received by logging in to his user profile on the administration portal.
    
    \item \textbf{Notification through E-Mail}
    
    The user can be notified of a new message through E-Mail.

\end{itemize}

\subsubsection{Interfaces}
In order to implement the basic use case, the components have to interact with each. The following diagram visualizes, which components interact. Each interaction is described afterwards. The arrows describe the main direction of information flow in an interaction.

\begin{center}
    \includegraphics[width=8cm]{Diagrams/Interaction Diagram.png}
\end{center}

\begin{enumerate}

    \item The web browser of the user displays the web pages hosted by the administration portal.

    \item The form-server can initiate the application if requested by the administration portal. The portal submits information for identification, authentication and authorisation of a user through an JSON Web Token and a user profile ID. In order for the form server to determine the correct form to display, the portal submits a FIM-ID which corresponds to a data field schemata. In case the user approved, the portal can also collect personal information from the user profile and submit it to the form server for automated filling in of the form.
    
    \item The form server provides the user a web page which displays an interactive form.
    
    \item The form server notifies the administration portal of changes in the status of applications. If the administration portal is notified of a submitted application, it requests the form server to transmit the application data.
    
    \item The administration portal requests the user profile for identification and authentication to login a user on the web page of the portal. The portal also requests authorisation from the user profile to for example access certain personal information.
    
    \item The administration portal enables the user to manage services of the user profile. It provides OSS tools for managing attributes stored in the user profile, to send new messages and to manage documents in the data wallet.
    
    \item The administration platform submits application data to the data exchange platform for delivery to the responsible institution.
    
    \item The data exchange platform notifies the administration portal of the delivery status of the application data.
    
    \item The data exchange platform notifies the institution of a new application and enables it to download the corresponding data.
    
    \item The institution can send messages to the inbox of a user profile to for example request additional documents.
    
    \item The user can send messages to the institution which for example asked for additional documents.
    
\end{enumerate}
    
\subsubsection{Data Objects}
This section describes what data each component processes. The property value describes a certain type of data. The source describes which component stores the value. This distinguishes the access of a property by reference and by copy. If the source of a property is another component, the property is accessed by reference. The description gives further detail on how the property is processed by the component.

\begin{table}[!h]
    \begin{tabularx}{\textwidth}{|l|l|l|C|}
    \rowcolor{LightCyan}
    \hline
    \multicolumn{4}{|l|}{Administration Portal: Service Web Page} \\
    \hline
    Property & Storage & Origin & Description  \\
    \hline
    \hline
    Name & Web Page & FIM & Name of an administrative service \\
    \hline
    URL & Portal & Portal & The URL of the Web Page \\
    \hline
    Description & Web Page & FIM & Description of an administrative service \\
    \hline
    FIM-ID & Web Page & FIM & FIM-ID of a data field schemata which describes attributes required from the user when applying for the service \\
    \hline
    \end{tabularx}
\end{table}

\begin{table}[!h]
    \begin{tabularx}{\textwidth}{|l|l|l|C|}
    \rowcolor{LightCyan}
    \hline
    \multicolumn{4}{|l|}{Administration Portal: User} \\
    \hline
    Property & Storage & Origin & Description  \\
    \hline
    \hline
    Attributes & User Profile & User Profile & The portal can manage attributes of user profiles \\
    \hline
    JWT & Portal & User Profile & The portal can retrieve a JSON Web Token from the profile after successful authentication by a user \\
    \hline
    URL Form & Portal & Form Server & The portal stores the URL where the form server hosts the interactive form \\
    \hline
    \end{tabularx}
\end{table}

\begin{table}[!h]
    \begin{tabularx}{\textwidth}{|l|l|l|C|}
    \rowcolor{LightCyan}
    \hline
    \multicolumn{4}{|l|}{Administration Portal: Application} \\
    \hline
    Property & Storage & Origin & Description  \\
    \hline
    \hline
    FMS-ID & Portal & Form Server & The portal receives the ID of applications from the form server \\
    \hline
    Status & Portal & Form Server & The portal receives status updates for each application \\
    \hline
    XML Form Data & Portal & Form Server & The portal can retrieve the data which was filled into a form \\
    \hline
    \end{tabularx}
\end{table}

\begin{table}[!h]
    \begin{tabularx}{\textwidth}{|l|l|l|C|}
    \rowcolor{LightCyan}
    \hline
    \multicolumn{4}{|l|}{User Profile: Personal Information} \\
    \hline
    Property & Storage & Origin & Description  \\
    \hline
    \hline
    Attributes & User Profile & FIM & The list of attributes the user can fill in at his profile are based on FIM data fields managed in the FIM repository. \\
    \hline
    Attribute Values & User Profile & User & The user can manually add values to a list of predetermined attributes in his profile \\
    \hline
    Credentials & User Profile & User & The User Profile stores encrypted credentials of the user to authenticate him \\
    \hline
    User Profile ID & User Profile & User Profile & The User Profile creates and stores a unique ID for the user \\
    \hline
    \end{tabularx}
\end{table}

\begin{table}[!h]
    \begin{tabularx}{\textwidth}{|l|l|l|C|}
    \rowcolor{LightCyan}
    \hline
    \multicolumn{4}{|l|}{User Profile: Authorisatiozation} \\
    \hline
    Property & Storage & Origin & Description  \\
    \hline
    \hline
    JWT & Other Component & User Profile & The user profile can create and transfer JSON Web Tokens which authorise other components to access resources of the user profile \\
    \hline
    \end{tabularx}
\end{table}

\begin{table}[!h]
    \begin{tabularx}{\textwidth}{|l|l|l|C|}
    \rowcolor{LightCyan}
    \hline
    \multicolumn{4}{|l|}{User Profile: Inbox} \\
    \hline
    Property & Storage & Origin & Description  \\
    \hline
    \hline
    Received Messages & User Profile & User & The user profile stores messages which were addressed to the user \\
    \hline
    Sent Messages & User Profile & User & The user profile stores messages which were sent by the user \\
    \hline
    \end{tabularx}
\end{table}

\begin{table}[!h]
    \begin{tabularx}{\textwidth}{|l|l|l|C|}
    \rowcolor{LightCyan}
    \hline
    \multicolumn{4}{|l|}{User Profile: Data Wallet} \\
    \hline
    Property & Storage & Origin & Description  \\
    \hline
    \hline
    Documents & User Profile & User & The user profile stores documents which were uploaded by the user \\
    \hline
    \end{tabularx}
\end{table}

\begin{table}[!h]
    \begin{tabularx}{\textwidth}{|l|l|l|C|}
    \rowcolor{LightCyan}
    \hline
    \multicolumn{4}{|l|}{Form Server} \\
    \hline
    Property & Storage & Origin & Description  \\
    \hline
    \hline
    FMS-ID & Form Server & Form Server & The form server creates a unique ID for each application \\
    \hline
    JWT & Form Server & User Profile & The form server uses a JSON Web Token to verify that the entity issuing the request is authorised \\
    \hline
    User Profile ID & Form Server & User Profile & The ID of the user who is associated to the application \\
    \hline
    FIM-ID & Form Server & FIM & The FIM ID of a data field schemata is used to determine how to construct the form \\
    \hline
    Form & Form Server & FIM & The list of attributes the user can fill in at the form are based on the FIM data field schemata. The form server maps the FIM data fields of the FIM data field schemata to attributes the form server understands \\
    \hline
    Status & Form Server & Form Server & The status describes which actions were performed on the application \\
    \hline
    Application Data & Form Server & User & The form server stores the data which the user filled in to the form as an XML file \\
    \hline
    Web Page & Form Server & Form Server & The form server hosts a web page with the interactive form  \\
    \hline 
    Web Page URL & Form Server & Form Server & The form server hosts the web page of the form at an individual URL \\
    \hline 
    \end{tabularx}
\end{table}

\subsubsection{Sequence Diagram}

\begin{figure}[!h]
    \centering
    \includegraphics[width=17cm]{Diagrams/Basic Use Case Sequence Diagram.png}
\end{figure}


\chapter{Identity Management}

\section{Functional Requirement Analysis}

\subsection{User Perspective}

\begin{itemize}
    \item The user wants to manage his personal information in one place
    \begin{itemize}
        \item The user wants to CRUD attributes
        \item The user wants SPs to CRUD his attributes
        \item The user wants a user friendly interface where he can CRUD his attributes
        \item In case the user cannot understand an attribute or its possible values, he needs OSS tools which enables him to enter the correct value
        \item The user wants to enter personal information only once (=> SPs should reuse existing attributes)
        \item The user wants a new or modified attributes to be instantly used everywhere
        \item The user wants a deleted attribute to be instantly deleted everywhere
        \item The user wants to have a list of often used attributes, in order to prefill them
    \end{itemize}
    \item The user wants to manage authentication in one place
    \begin{itemize}
        \item The user does not want to authenticate
        \item The user wants that only he has access to the identity
    \end{itemize}
    \item The user wants to manage authorisation in one place
    \begin{itemize}
        \item The user wants to manage who can perform which actions in his name (f.e CRUD attributes, place order, cancel subscription)
        \item The user wants an authorisation to only be granted if he manually approves
        \item The user wants an authorisation to be associated with a single entity, a reason, which actions can be performed and duration of the authorisation
        \item The user wants to communicate with the authorised entity
        \item The user wants to have much information about the authorised entity
        \item The user wants to be able to revoke an authorisation
        \item The user wants to save the history of authorisations
    \end{itemize}
    \item The user wants to use all OSS tools in one place
    \begin{itemize}
        \item Communication through inbox
        \item Subscribe for a service of an SP
        \item Place order in a shop of an SP
        \item ...
    \end{itemize}
    \item The user wants

\end{itemize}

\subsection{Service Provider Perspective}

\begin{itemize}
    \item SPs want to manage personal information of users
    \begin{itemize}
        \item SPs want to CRUD attributes of the user
        \item In case modification of an attribute has strong implications on the operation of a system, the modification of an attribute needs to be limited
        \item SPs want to access personal information already collected by other SPs
        \item SPs want 
    \end{itemize}
    \item identification
    \item authentication
    \item authorisation
    \item 
\end{itemize}

\subsection{OZG Perspective}

\section{IMP System Proposal}
There exist many ways to realize an IMP system. Based on the previously analyzed requirements of users and OZG systems, a possible solution for an IMP system is presented.

\subsection{Components}
The IMP system operates on three domains and consists 3 system components. The user domain contains systems, a user directly interacts with. As for the system architecture of the OZG, can be a web browser. The solution for the IMP system provides an IMP client as an app for smartphones. The IMP domain contains systems hosted by the IMP provider. The IMP server is part of this domain. The Service Provider domain contains the system architecture of the service provider and integration components of the IMP solution.
In the following sections, IMP client and IMP server are explained in more detail, the integration system will be the focus of the next chapter.

\begin{figure}[h]
\caption{Overview IMP System}
    \centering
    \includegraphics[scale=0.25]{Diagrams/IMP System Overview.png}
\end{figure}

\subsubsection{IMP Client}

The IMP client is a smartphone application which enables the user to manage his IMP identity. It enables the user to:

\paragraph{Create an IMP identity}

After the user installs the application on his smartphone, he is able to create an IMP identity. The identity consists of a public and private key-pair. The private key will never leave the smartphone. The public key is shared with the IMP server along with information on how to communicate with the client.

\paragraph{Manage Attributes}

The user is able to add, remove, update and read any key-value pair as an attribute of his identity. Attributes are encrypted with the 

\paragraph{Documents}

The user is able to add, remove, update and read any document as an attribute of his identity. All documents are stored on the smartphone. 

\paragraph{Send and Receive Messages}
\paragraph{Send and Receive Relationship Requests}
\paragraph{Send and Receive Share Requests}
\paragraph{Display Forms}

\subsubsection{IMP Server}
The IMP server interacts with the clients in order to deliver information from one client to the other. The server has the following functionalities:
\begin{itemize}
    \item Store identities
    \item Create and store relationship template
    \item Create and store request template
    \item Deliver relationship requests
    \item Deliver requests
    \item Deliver messages
\end{itemize}

\subsection{Functionalities}
As already introduced, the IMP solution provides the user with multiple functionalities. In this section, important features are described in more detail.

\subsubsection{Relationship}

IMP identities can establish an IMP relationship. Through this relationship, both identities can securely share attributes and communicate.

To establish a relationship, a relationship template is required. The template is created by an identity which wants to receive relationship requests and is shared with the identity which wants to request a relationship. The relationship template is stored on the IMP server and can be retrieved trough template ID. Besides technical information, the template usually contains:
\begin{enumerate}
    \item Information which the templator would like to share about itself 
    \item Information which the templator would like the requestor to share
    \item Requested additional information about the requestor 
    \item Meta information 
\end{enumerate}

\subsubsection{Share Attribute}

\subsubsection{Change Attribute}


\subsection{Data Objects}

\begin{table}[!h]
    \begin{tabularx}{\textwidth}{|l|l|l|C|}
    \rowcolor{LightCyan}
    \hline
    \multicolumn{4}{|l|}{Form Server} \\
    \hline
    Property & Storage & Origin & Description  \\
    \hline
    \hline
    \hline
    \end{tabularx}
\end{table}

\chapter{IMP Integration Architecture}

The goal of the integration architecture is to make as many functionalities required for the basic OZG use case available through the IMP client as possible. As described in chapter 2.3, the steps of the basic OZG use case are:
\begin{enumerate}
    \item{Create User Profile}
    \item{Login to User Profile}
    \item{Selection of Administrative Service}
    \item{Filling in Application}
    \item{Submission of Application}
    \item{Reception of Application by Administration Portal}
    \item{Submission of Application to Data-Exchange Platform}
    \item{Management of Applications}
    \item{Communication}
\end{enumerate}

\section{Requirement Analysis}

\subsection{Integration Challenges}

The following challenges are a result of the functionalities an IMP system provides in addition to conventional identity management systems. The origin of the challenges is the interoperability of the IMP identity with multiple SPs. It is the purpose of the integration architecture to solve as many of these challenges as possible.

\subsection{Basic Use Case}

\subsection{Extensions}

\section{Integration Proposal}

\subsection{IMP Connector}

\subsection{Messaging}

\section{Minimal Integration Architecture}
To enable service providers a quick introduction of IMP systems into their system architecture, this first integration architecture describes a minimally invasive integration. After deployment of this integration architecture, all OZG services are still usable as before but can be enhanced by IMP solutions. For this integration to be minimally invasive, only one connector exists and integrates only with the administration portal. No other OZG system component is modified or directly accessed.

The goal of the integration is to enhance each step of the basic OZG use case through integration with IMP solutions. Therefore in the overview section, for each step, possible integration's of OZG systems are evaluated. The following sections contain more detailed descriptions of the selected integration strategies.

\begin{figure}[h]
    \centering
    \includegraphics[scale=0.15]{Diagrams/Integration Architecture 1/Overview.png}
\end{figure}

\subsection{Overview}

This section gives an overview of integration possibilities for each step of the basic OZG use case. For each step, multiple integration strategies are evaluated and one strategy is selected for more detailed description in following sections.

\paragraph{Create User Profile}

The first step of the basic OZG use case is the creation of a user profile through the web page of the administration portal.

Creation and management of multiple user profiles quickly becomes a burden for users. If all service providers would switch to using only identities provided by an IMP solution, users would only have to manage one digital identity. However performing this change is no simple task for a service provider. An abrupt change of the system architecture can lead to many problems like extended periods of down times. This integration architecture therefore focuses on a simplified initial integration of IMP identities where OZG user profiles can still be used to access OZG services.

It would be possible to integrate the IMP identity as an alternative to the user profile. The user could then use OZG services either by creating an OZG user profile or by providing an IMP identity. Another option would be to enable a connection of an existing IMP identity and an existing OZG user profile. The user would then have to create a user profile in order to access OZG services but could connect an IMP identity for improved identity management. The benefit of using the IMP identity as an alternative is, that the user has better control over his personal information, as the attributes of the identity are not stored by the OZG system architecture but the IMP solution. OZG systems would have to request personal information from the IMP identity while providing a reason for doing so. After the user updates attributes of its IMP identity, the OZG systems would automatically use the most recent values when requesting access to personal information of the IMP identity. However, this form of integration is already pretty advanced and would require invasive integration in order for the administration portal to add an additional process for executing the basic use case with the IMP identity instead of the user profile. As the focus of this integration architecture is non invasiveness, the connection of IMP identity and user profile is preferred.

The IMP solution provides a service for establishing a connection between entities. It is called an IMP relationship. Through a relationship, two entities can identify each other, assume authentication of the other party and can securely communicate and share attributes of their IMP identities. To utilize an IMP relationship between the administration portal and the IMP identity as connection between user profile and IMP identity, the relationship has to contain identification of the user profile in form of for example a user ID. The administration portal would then be able to map relationships along with the IMP identity ID to user profiles.

In order for an IMP identity to request a relationship with the administration portal, the portal has to provide a relationship template to the IMP client of the user. The template can be made accessible to the IMP client trough a new web page which displays the template ID as QR code if the user is logged in. Each time a logged in user requests the web page, the web server requests the IMP connector to create a relationship template.

The IMP connector provides an API for creation of relationship templates as part of its REST interface. Through a HTTP POST call, the web server transmits the appropriate request to the connector and receives the template as response. The web page of the administration portal can then render the template as a QR code on the web page.

\begin{figure}[h]
    \centering
    \includegraphics[scale=0.3]{Diagrams/Integration Architecture 1/Overview/Relationship Template REST.png}
\end{figure}

This is one example, where the OZG system could use the REST interface for integration with the IMP system. However, messaging can improve the integration solution to be more maintainable, stable and scalable and will be the preferred option in following scenarios. Using messaging, the web server can publish a message containing the request for a new relationship template on a "Relationship Template Request" channel, where the connector receives it, creates the template and publishes a reply message containing the template ID on the "Relationship Template Reply" channel.

\begin{figure}[h]
    \centering
    \includegraphics[scale=0.3]{Diagrams/Integration Architecture 1/Overview/Relationship Template Messaging.png}
\end{figure}

It could happen, that in future, the API of the connector changes. If the web server integrated the connector through its REST interface, this would require the request code of the web server to be changed. If messaging is used, messages leaving the "Relationship Template Request" channel could be mapped to the new API signature through an additional messaging component called message translator. This is an example for how a messaging can improve maintainability of an integration architecture.

\begin{figure}[h]
    \centering
    \includegraphics[scale=0.3]{Diagrams/Integration Architecture 1/Overview/Relationship Template Messaging Improved.png}
\end{figure}


When the user is presented with the web page containing the QR code, the user can scan it with its IMP client. Based on the template ID, the client retrieves the template from the IMP server and displays it for the user. If the user accepts the template, the client sends a relationship request to the IMP server, who forwards it to the corresponding connector.

Upon receiving the request, the connector has to notify the administration portal of an incoming relationship request. If the web server does not provide an HTTP endpoint for communicating with the connector, it won't be able to receive a notification about the arrival of a new relationship request. The server would have to regularly request a status update which is not efficient. Using messaging, the connector can publish the content of the relationship request to a "Relationship Request" channel where the subscribed administration portal will be notified. The administration portal can process the relationship request and publish a message containing a response on the "Relationship Response" channel, where the connector will be notified. Through the IMP server, the response server is routed to the IMP client and the relationship is established.
In order to map relationships to user profiles, the administration portal manages a database which contains entries consisting of IMP identity ID, relationship ID and user profile ID.

\begin{figure}[h]
    \centering
    \includegraphics[scale=0.3]{Diagrams/Integration Architecture 1/Overview/Relationship Request.png}
\end{figure}


\paragraph{Attribute Synchronization}

The goal of the connection of IMP identity and OZG user profile is to enable management of personal information through the IMP client. Attributes stored as part of the IMP identity and attributes stored as part of the user profile have to be manageable through the IMP client.

One option would be for the IMP client to overwrite all attributes of the OZG user profile with those stored in the IMP identity and update values in the user profile as soon as an attribute of the IMP identity changes. This would require the user profile management web page of the administration portal to be locked for profiles with a connected IMP identity. Another option would be to enable the user to manage attributes of IMP identity and user profile separately through the IMP client.

The benefit of the first option is, that the user only has to manage attributes for his IMP identity and the user profile will be updated automatically. However, this would require the user to always share the most current personal information with the OZG systems. Depending on the service provider, updates of attributes might not be possible in every case or would require additional actions by the service provider. If the user for example changes the birth date of its IMP identity, the OZG system might not want to approve this change immediately. Another option would be for the user to manage IMP identity and user profile separately. Changes of the OZG user profile can then happen on a request reply basis. The user would also be able to manage its OZG user profile through the web page of the administration portal and request attribute changes from there. The disadvantage of this option is, that the management of personal information is more complicated than the first option. Through attribute management tools on the IMP client, the experience can be improved.

As the second option provides necessary features, which the first option does not have, it is preferred.

The management of both IMP identity attributes and attributes of the OZG user profile can be done as part of the relationship established for their connection. As part of the connection, IMP identity and administration portal share attributes. The user profile can be determined to store the exact same values as shared by the IMP identity as part of this relationship. Therefore, in order to change attributes of the user profile, the value of the shared attributes has to change. This can be done through a request-reply process. One party requests the change of a shared attribute, which will only be done if the other party accepts. This enables the user to manage the OZG user profile through the IMP client by requesting a change for attributes shared as part of a relationship. The user could also manage his user profile through the web page of the administration portal, resulting in a request form the administration portal to change shared attributes of the relationship.

As administration portal and IMP identity established a relationship, the connector can be excluded from the diagram for better visualisation. Behind the scenes, all interactions between administration portal and IMP client still go through connector and IMP server.

If the user requests the change of a shared attribute through the IMP client, it sends an attribute change request to the administration portal by publishing a message on the "Inbound Attribute Change Request" channel. After the administration portal decides to accept the request, it can publish a message to the "Outbound Attribute Change Response" channel. This message will on one hand be received by the IMP client and on the other hand result in an update of the user profile. 

\begin{figure}[h]
    \centering
    \includegraphics[scale=0.3]{Diagrams/Integration Architecture 1/Overview/Attribute Change IMP Client.png}
\end{figure}

If the user requests the change of a shared attribute through the web page of the administration portal, the web server sends an attribute change request to the "Outbound Attribute Change Request" channel which is received by the IMP client. If the user accepts the request, the client publishes a message on the "Inbound Attribute Change Response" channel where it results in an update of the user profile.

\begin{figure}[h]
    \centering
    \includegraphics[scale=0.3]{Diagrams/Integration Architecture 1/Overview/Attribute Change Web Page.png}
\end{figure}

\paragraph{Login to a User Profile}

The second step of the basic OZG use case is to login to a user profile on the web page of the administration portal to access OZG services. As previously described, the integration architecture does not replace the OZG user profile. The user is therefore still able to login to his user profile on the administration portal and take advantage of all OZG services. The IMP system can however provide an additional way to access OZG services through the IMP client. One possibility was already introduced: the management of personal information.

In order for the administration portal to perform actions on behalf of a user, it has to verify, that the request is valid. Through the web page, this is often done by establishing a browser session, through a cookie and after the authentication process, updating the authentication status of the session. When the web server receives a request from a authenticated session, it can use a JSON Web Token, created during the registration process to access resources of other system components. This process is necessary because any person can send requests to the web server and possibly pretend to be a different user. In contrast to the web browser, requests coming through the IMP system can be assumed to be valid and therefore originate from the correct IMP identity and user. Therefore it is not necessary for the user to authenticate for the OZG system when using an IMP identity. As the IMP client is an app installed on the device of the user, it might even be possible to completely remove user authentication when accessing the client as most users secure their phones through a lock pattern already.

\begin{figure}[h]
    \centering
    \includegraphics[scale=0.3]{Diagrams/Integration Architecture 1/Overview/Login.png}
\end{figure}

\paragraph{Application for Administrative Service}
The third step is selection of an administrative service through the web page of the administration portal. This service is not directly related to identity management and will therefore not be provided by the IMP client.
The steps four to eight are not integrated by the IMP solution as this would require invasive interaction with components other than the administration portal.

As a result of that, most of the steps necessary for applying for an administrative service remain to be done through the web page of the administration portal with a logged in user profile. The next integration architecture will present a solution where most of the steps will be done through the IMP systems.

\paragraph{Communication}

The ninth step of the basic OZG use case is to communicate with responsible institutions. As both the user profile and the IMP identity provide an inbox, the IMP client can be used to send and receive messages through the inbox of the OZG user profile.

The messages related to a OZG user profile can be exchanged as part of the relationship between the administration portal and the IMP identity. Messages received by the user profile can be forwarded by the administration portal to the inbox of the IMP identity. Sending messages from the IMP client to institutions however is not possible, as the IMP client does not know the address of the institution responsible for an application. However, if the IMP client already received a message from an institution, it can enable the user to respond.


\subsection{Connection}

\subsection{Personal Information}

\subsection{Authentication}

\subsection{Communication}

\section{Integration Architecture 2}

The goal of this integration architecture is to make as much functionalities required for the basic OZG use case available through the IMP client. Now, the OZG user profiles are replaced with IMP identities and the integration architecture integrates with any system component necessary.

\begin{figure}[h]
    \centering
    \includegraphics[scale=0.15]{Diagrams/Integration Architecture 2/Overview.png}
\end{figure}

\subsection{Overview}

The first step of the basic OZG use case is the creation of a user profile through the web page of the administration portal. This is not necessary any more, as user profiles are replaced by IMP identities. The user will only have to create an IMP identity.

The second step of the basic OZG use case is to login to a user profile on the web page of the administration portal. As user profiles get replaced by IMP identities, the IMP client can be used to access OZG services. The IMP client does not require authentication, as it is installed on the smart phone of the user. The user can secure his smartphone through for example biometric authentication.

The third step of the basic OZG use case is to select an administrative service which is still done through the web page of the administration portal.

The fourth step of the basic OZG use case is to fill in the application. The application is not started by the user filling in a form on the form server and submitting the result to the administration portal any more, but by requesting a relationship with the institution responsible for processing a selected application.

The web page of the administration portal displaying a selected administrative service does not contain an integration with a form server any more. Instead it displays the QR code for a relationship template provided by the institution responsible for processing the application.

For each administrative service, the institution is responsible for processing, it issues a HTTP POST request to the institution connector requesting the creation of a relationship template. The connector responds to the institution with the template IDs.

In order for the administration portal to display the correct relationship template, he requests the user to manually enter his home address. After the administration portal determined the responsible institution, he retrieves the correct template ID and displays it to the user as QR code.

If the user scans the QR code with his IMP client, the client retrieves the relationship template corresponding to the template ID of the QR code and display it. The relationship template contains a description of the selected administrative service as reason for the relationship. The template also requests the IMP identity to share several attributes, which the institution requires in order to process the application. This replaces the functionalities of the form server.

The fifth step of the basic OZG use case is to submit the application. After the user accepts the relationship template, the client sends a relationship request to the institution which created the template along with the shared attributes. The institution can accept the request and start processing the application. The IMP identity in this case has to share personal information only with the specific institution responsible for processing the application.

The sixth and seventh steps of the basic use case are no longer necessary as the application is properly delivered to the institution as result of the fifth step.

The eighth step of the basic OZG use case is the management of active applications. As each application is represented by a relationship between an institution and an IMP identity, the application can be managed by requesting a termination of the relationship and sharing the status of the relationship as part of an attribute of the institution. Termination of the relationship after approval of both parties, also prohibits the institution from further usage of personal information of the IMP identity.

The ninth step of the basic OZG use case is the communication between user profile and institutions responsible for active applications. As an active relationship is represented by a relationship, the two parties can directly communicate through the relationship.

\chapter{Solution Evaluation}

\section{Conclusion}

\chapter{Outlook: Advanced IMP Integration}

\chapter{Tables}

        \begin{table}[!h]
            \begin{tabularx}{\textwidth}{|l|C|l|}
            \hline
            Property & Description & Data Source \\
            \hline
            \rowcolor{LightCyan}
            \multicolumn{3}{|l|}{Input} \\
            \hline
            
            JSON Web Token & The portal received this JWT as a result of the login process with the user profile component. This token enables the form-server to verify the authentication of the user. & User Profile \\
            
            \hline
            
            User Profile ID & This ID is a unique identifier of a user profile and is used by the form server to associate the initiated application with a user profile. & User Profile \\
            
            \hline
            
            Personal Data & If the user approves, the administration portal submits personal data, the form-server can use to automatically fill in the form. & User Profile \\
            
            \hline
            
            FIM Schema ID & The administration portal stores a hard coded FIM-ID for every administration service. This FIM-ID corresponds to the correct data field schemata, the form-server can display & Hard Coded in Portal \\
            
            \hline
            \rowcolor{LightCyan}
            \multicolumn{3}{|l|}{Output} \\
            \hline
            
            Status & The form-server answers with the status of the initiated application. & From Server \\
            
            \hline
            
            FMS-ID & While initializing the application, the form-server creates an ID, which uniquely identifies the application. This FMS-ID is sent back to the portal so that it can request information about the application at a later time. & Form Server \\
            
            \hline
            \end{tabularx}
            \caption{Interface of the form-server for initialization of an application}
            \label{table:interface_form_initialization}
        \end{table}

        \begin{table}[!h]
            \begin{tabularx}{\textwidth}{|l|C|l|}
            \hline
            Property & Description & Data Source \\
            \hline
            \rowcolor{LightCyan}
            \multicolumn{3}{|l|}{Input} \\
            \hline
            
            FMS-ID & The FMS-ID was created by the form-server during the initialisation of the application and sent to the administration portal. Each status update includes this ID in order for the portal to associate the update with an application. & Form Server \\
            
            \hline
            
            Status & The status provides information about interactions of the user or the form-server with the application. The possible states are:
            
            \begin{itemize}
                \item \textbf{null}: This application has this status right after initialisation.
                \item \textbf{submitted}: The application has this status after the user triggered the submit functionality of the form-server. Usually the form is now locked for further modification by the user.
                \item \textbf{filed}: The form-server submitted the application to the portal but did not yet receive a confirmation.
                \item \textbf{confirmed}: The form-server received a confirmation from the portal.
                \item \textbf{deleted}: The form-server deleted the application.
            \end{itemize}
            
            & Form Server \\
            \hline
            \end{tabularx}
            \caption{Interface of the administration portal for receiving status updates by the form-server}
            \label{table:interface_status}
        \end{table}

        \begin{table}[!h]
            \begin{tabularx}{\textwidth}{|l|C|l|}
            \hline
            Property & Description & Data Source \\
            \hline
            \rowcolor{LightCyan}
            \multicolumn{3}{|l|}{Input} \\
            \hline
            
            FMS-ID & The FMS-ID was created by the form-server during the initialisation of the application and sent to the administration portal along with the status update of the submission. The administration portal sends this ID in order for the form-server to know which application data to transfer. & Form Server \\
            
            \hline
            \rowcolor{LightCyan}
            \multicolumn{3}{|l|}{Output} \\
            \hline
            
            FMS-ID & The FMS-ID is transmitted in order for the portal to associate the response with an application. & Form Server \\
            
            \hline
            
            JSON Web Token & The JWT which was initially sent from the portal to the form-server is now transmitted back again, in order for the portal to verify, that the form-server is authorized to submit an application for the user. & User Profile \\
            
            \hline
            
            User Profile ID & The user profile ID was initially transmitted to the form-server by the portal and is now transmitted back in order for the portal to associate the application with a user profile. & User Profile \\
            
            \hline
            
            XML File & The XML file contains the personal data retrieved through the form. & Form Server \\
            
            \hline
            \end{tabularx}
            \caption{Interface of the form-server for requesting application data by the administration portal.}
            \label{table:interface_application_data}
        \end{table}

\printbibliography


\end{document}
