% This template was initially provided by Dulip Withanage.
% Modifications for the database systems research group
% were made by Conny Junghans,  Jannik Strtgen and Michael Gertz

\documentclass[
     12pt,         % font size
     a4paper,      % paper format
     BCOR=10mm,version=first,     % binding correction
     DIV=14,version=first,        % stripe size for margin calculation
%     liststotoc,   % table listing in toc
%     bibtotoc,     % bibliography in toc
%     idxtotoc,     % index in toc
%     parskip       % paragraph skip instad of paragraph indent
     ]{scrreprt}

%%%%%%%%%%%%%%%%%%%%%%%%%%%%%%%%%%%%%%%%%%%%%%%%%%%%%%%%%%%%

% PACKAGES:

% Use German :
\usepackage[english]{babel}
% Input and font encoding
\usepackage[utf8]{inputenc}
\usepackage[T1]{fontenc}
% Index-generation
\usepackage{makeidx}
% Einbinden von URLs:
\usepackage{url}
% Special \LaTex symbols (e.g. \BibTeX):
%\usepackage{doc}
% Include Graphic-files:
\usepackage{graphicx}
% Include doc++ generated tex-files:
%\usepackage{docxx}
% Include PDF links
%\usepackage[pdftex, bookmarks=true]{hyperref}
\usepackage{csquotes}
\usepackage{color, colortbl, tabularx, ragged2e}
\definecolor{LightCyan}{rgb}{0.88,1,1}
\newcolumntype{C}{>{\raggedright\arraybackslash}X} % centered "X" column
% Fuer anderthalbzeiligen Textsatz
\usepackage{setspace}
\usepackage{multirow}
\usepackage{longtable}
\usepackage{float}

% hyperrefs in the documents
\usepackage[bookmarks=true,colorlinks,pdfpagelabels,pdfstartview = FitH,bookmarksopen = true,bookmarksnumbered = true,linkcolor = black,plainpages = false,hypertexnames = false,citecolor = black,urlcolor=black]{hyperref} 
%\usepackage{hyperref}


%%%%%%%%%%%%%%%%%%%%%%%%%%%%%%%%%%%%%%%%%%%%%%%%%%%%%%%%%%%%

% OTHER SETTINGS:

% Pagestyle:
\pagestyle{headings}

% Choose language
\newcommand{\setlang}[1]{\selectlanguage{#1}\nonfrenchspacing}

\usepackage{biblatex}
\addbibresource{references.bib}

\begin{document}

% TITLE:
\pagenumbering{roman}
\begin{titlepage}
     \vspace*{1cm}
     \begin{center}
          \vspace*{3cm}
          \textbf
          {
               \Large University of Heidelberg\\
               \smallskip
               \Large Institute for Computer Science\\
               \smallskip
               \Large Working group database systems\\
               \smallskip
          }

          \vspace{3cm}

          \textbf{\large Bachelor thesis}

          \vspace{0.5\baselineskip}
          {
               \huge
               \textbf{Integrating Identity Management Providers based on Online Zugangs Gesetz}
          }

     \end{center}

     \vfill
     {
          \large
          \begin{tabular}[l]{ll}
               Name:                 & Jonas Gann              \\
               Matriculation number: & 3367576                 \\
               Supervisor:           & Prof. Dr. Michael Gertz \\
               Date of submission:   & \today
          \end{tabular}
     }

\end{titlepage}

\onehalfspacing

\thispagestyle{empty}

\vspace*{100pt}
\noindent
I assure that I have written this bachelor thesis on my own and only used the specified sources and resources and that I followed the principles and recommendations "Responsibility in Science" of the University of Heidelberg.

\vspace*{50pt}
\noindent

\underline{\phantom{mmmmmmmmmmmmmmmmmmmm}}

\medskip
\noindent
Date of Submission: \today
\newpage

\chapter*{Zusammenfassung}

\newpage

\chapter*{Abstract}

\newpage

\tableofcontents
\cleardoublepage
\pagenumbering{arabic}

\chapter{Introduction}

\section{Context}
Since invention of the World Wide Web, the number of its users increased rapidly. Today, more than 90 percent of the German population use the internet \cite{Onlinestudie}. Enterprises and organizations, recognised this potential to provide their services to a large number of people as Service Providers (SP). A frequently used method of SPs is "online self-service" (OSS). Specialized software tools enable customers to use services through the internet without direct human interaction.

Many tools require identification of a user in case interactions or initiated business processes have to be associated with him. A system for management of identities of users therefore is part of numerous system architectures. Service Providers usually enable users to manage their partial identities using online self-service through so called User Profiles. Today, customers of SPs are associated with multiple partial identities and user profiles - one for each SP

The result is variety of problems in management of digital identities \cite{IdentityCrisis}. Here are some examples:
\begin{itemize}
    \item Multiple SPs collect, use and share parts of identities which leads to distributed and fragmented identities.
    \item As a result of the distributed identities, risk of data breaches increases.
    \item Managing multiple use profiles is very inconvenient for users.
    \item It is often unclear who stores which data and for what purpose.
    \item SPs have an incentive to collect and share more user data than they need.
\end{itemize}

An improved model for identity management is necessary. In recent years, multiple new models have been invented. One model called Social Login became popular. It enables customers to create User Profiles for SPs through existing User Profile of Social Networks like Facebook. This improves identity management by enabling SPs to rely on the external profile for unique identification, authorisation and personal attributes. Users can therefore manage identity related information through their Social Profile which is accessed by the SP.

This approach, however, still leaves the requirement of User Profiles for each SP because, depending on the provided service, additional attributes which are not part of the Social Profile, have to be manageable by the user. SPs also might require additional online self-service tools the social network does not provide.

The solution to this problem is an Identity Management Provider (IMP) which manages the whole online identity of a person and can replace the individual partial identities and User Profiles. The system itself is provided and hosted by a SP for consumption by other SPs. To replace existing User Profiles, the IMP can not only rely on systems for registration, authentication, authorisation and provisioning. Additional requirements of SPs like communication, data wallet and management of additional attributes have to be satisfied.

In order for an SP to use identities and services of an IMP, an integration into the system architecture of the SP has to take place. Due to the complex nature of system architectures and identity management, this is a difficult task.

A currently relevant example for usage of online self-service with particular interesting requirements for identities is the German "Online Access Law" (Online Zugangs Gesetz: OZG). It requires all administrative services of the German federal republic, each member state and commune to be digitally available through interoperable user profiles. The current plan is to make the profiles only available for usage in context of the OZG. From a user perspective, this would be yet another partial identity and user profile to manage.

\section{Objective}
Based on the prominent example of the OZG, the bachelor thesis will construct a message based integration architecture which enables system architectures of service providers with existing user profiles to make their services usable with an Identity Management Provider. In order to make the integration architecture suitable to real life requirements, the OZG is selected as an example due to its current relevance and high requirements regarding identity management. The integration architecture is however not only applicable in cases relevant for OZG but also for other Service Providers with similar requirements. As requirements of SPs are not guaranteed to be similar to those of the OZG, the integration architecture is built to be expendable.

The \textbf{integration architecture} for the OZG enables \textbf{IMP functionalities} to be usable for selected \textbf{OZG scenarios} by integrating a \textbf{connector} in the \textbf{system architecture} of a member state.

\begin{itemize}
    \item OZG scenarios are based on documented digitalization efforts of administrative services in the context of the OZG and are for example the application for an administrative service or the communication with a governmental institution.
    \item The system architecture is based on documentation about architecture plans of member states.
    \item The functionalities of the IMP are based on a requirements analysis of OZG scenarios and additional research.
    \item The connector is defined based on functionalities the integration architecture requires from the IMP.
    \item The integration architecture is the main scientific contribution of this bachelor thesis. It is based on messaging patterns in order to utilize a combination of established and tested messaging solutions. In order to save investments it also reuses as many system components as possible and modifies as few system components as necessary.
\end{itemize}

The integration architecture ...

\section{Structure of Work}
The "Fundamentals" chapter explains terminology and basic concepts.

In the following chapter called "Identity Management System" the thesis analyzes requirements for identity management through the perspective of the user and through the perspective of service providers for the OZG. It is presented, how an identity management provider could solve current identity management problems and which challenges for an integration exist.

The chapter "Integration Proposal" first documents a possible IMP system and connector, based on the requirement analysis of the previous chapter. It then presents the integration architecture with iteratively increasing requirements through textual description, multiple flow diagrams and messaging architectures.

A "Solution Evaluation" analyzes benefits and problems of the proposed solution.

The "Outlook" will conclude the thesis by presenting possibilities for advanced integration architectures.

\chapter{Background and Related Work}

\section{Terminology}

\paragraph{Service Provider}
Service providers (SP) are entities like enterprises or organisations which make services accessible over the internet.

\paragraph{Online Self-Service}
Online self-service (OSS) describes a method used by Service Providers to make their services available to the user. The method is characterized by being available over the internet, requiring no direct human interaction but instead enabling the user to access services in a self-reliant way. The user is usually supported by a variety of software solutions.

\paragraph{Self-Service Tools}
Self-service tools are software solutions for customers to enable a online self-service experience. Depending on the use case, the tools can be transactional or informing. Informing tools help users to retrieve relevant information, often replacing human customer support. Transactional tools help users to interact with services in a persistent way to for example save personal data or trigger business processes.

Information tools can be for example so called "WiKi" pages or search fields which assist the user in his process of finding information. Transactional tools can be for example online forms which assist the user in his process of triggering an order placement business process.

In contrast to information tools, transactional tools often require identification of users in order to associate them with changes they made to the system: If a user triggers the business process of ordering a product, the SP requires identification of the user.

\paragraph{Digital Identity}
The digital identity, for simplicity called identity in the thesis, is the sum of all digital personal information. Each piece of personal information is an attribute. Attributes can be identifiers which are able uniquely identify an entity. It is also possible to use an aggregation of attributes as identifier.

Identifiers which are commonly used by SPs are the phone number and E-Mail address. Attributes are for example nicknames, hobbies, interests or education. The full name of a person is often not sufficient unique identification, the aggregation of name, age and home address can therefore be used as identifier.

\paragraph{Partial Digital Identity}
A partial digital identity is a subset of an identity. Partial identities are commonly used by SPs to manage identity information which is relevant for them. 

\paragraph{Identity Management}
Identity management is the creation, utilization / reading , updating and deletion of identities or partial identities. The possible utilization of identities depends on the system for management of identities. It could be for example authorization for a service or communication through an inbox.

\paragraph{Identity Management System}
Identity management systems are solutions which enable SPs and customers to manage identities or partial identities.
They usually provide the following functionalities an properties:
\begin{itemize}
    \item 
\end{itemize}

\paragraph{User Profile}
User profiles are one method for identity management. They are characterized by each SP providing a separate identity management system which often consist of an identity provider (IdP) for storing identities, a customer relation management system (CRM) for accessing identities internally and a website with online self-service tools for providing access to customers.

\section{Identity Management Provisioning (IMP)}
Identity management provisioning (IMP) is a method for identity management which is characterized by an identity management provider hosting an identity management system for "non-partial" identities. The identity management system provides one-stop online self-service for users and manages the storage of identities. SPs are provided access to the identities of the identity management system.

The following are functionalities of an IMP system in addition to those of conventional identity management systems:

\paragraph{Attribute Management}
It is common for identity management systems to enable users to manage a predetermined set of attributes. IMP systems however do not just manage partial identities and therefore enable users to manage any possible attribute of their identity. Users therefore can add, remove and modify any attribute.

As often, attributes of a digital identity are defined by service providers, they are also able to create new attributes or delete and modify existing ones, if approved by the user. 

\paragraph{Online Self-Service}
Many attributes, especially those created by SPs, do not make sense to the user without context and my be heavily dependant on systems of a SP. Therefore, the IMP provides solutions for online self-service which supports the management of attributes.

\paragraph{Integration}
Service providers which use IMP systems as solution for identity management require access to its services and have to make their existing systems work with them. The IMP solution therefore provides interfaces which enable the SP to access services. As it is in the best interest of the IMP solution to integrate SPs, advanced integration methods reducing the integration effort for SPs can be provided. 

\section{Online Access Law (OZG)}
In 2017 the German government passed the law for improvement of online access to administrative services (Online Zugangs Gesetz: OZG) which requires federal republic and member states to execute the following regulations until 2022 \cite{BMI:OZG_Wortlaut}:
\begin{enumerate}
    \item \textbf{Digital availability of administrative services} \\
    An administrative service is the electronic processing of administrative procedures which are available from outside the governmental institution.  As it is not clear which administrative services exactly are meant by the definition of the OZG, the BMI created a catalogue \cite{BMI:Verwaltungsleistungen}. The OZG requires these services to be digitally available. As a guideline to what is considered sufficient availability, the BMI defined a maturity model \cite{BMI:Digitale_Services}.
    \item \textbf{Digital access to administrative services through administration portals of a portal network} \\
    Federal republic, each member state and each commune must provide an administration portal. Portals of communes must be linked to the portal of the corresponding member state. Portals of federal republic and member states must be connected through a portal network. \cite{BMI:Portalverbund} Each portal must provide a "seek and find" feature, which enables users to find all administrative services provided by any administration portal \cite{Cotar:Drucksache_19/19089}. 
    \item \textbf{Interoperable user profiles for accessing administrative services} \\
    Federal republic and member states must provide user profiles which can be used to identify the corresponding person while requesting access to administrative services, to save personal information according to the once-only principle, to receive and send messages via a digital mailbox and to pay for services \cite{Cotar:Drucksache_19/19089}. The user profiles must be interoperable for every administration portal of the portal network.
\end{enumerate}

Execution of the OZG can be separated into two projects:

\paragraph{Digitalization}
Digitalization of administrative services focuses on accessibility towards users but not modernization of its internal execution by administrative institutions. The goal of digitalization in this case is, to make administrative services available towards a user in a digitized way:

A digitized administrative service can be accessed by a user through a website. The website is hosted either by the federal republic or a member state and is called "Administration Portal". Access is usually provided through an application form. The administrative service can be managed through a user profile which is provided by the federal republic or the member state. Management of an administrative service usually includes starting the service by sending in a form, communicating with responsible institutions through the inbox of the profile and receiving a result. The user profile can also enable users to upload documents to a data wallet and to save personal information for automatically filling in forms.

Digitalization of an administrative service is the modification of the underlying processes to incorporate the usage of the described features of the user profile and administration portal. In total, the BMI lists 575 relevant services, some of them provided by the federal republic, some by the member states and yet other by the communes \cite{BMI:Onlinezugangsgesetz}.

\paragraph{Networking}
The networking focuses on connecting governmental systems to make all digitized administrative services available for every user. This includes most importantly the connection of administration portals to a portal network through an online gateway and the interoperability of user profiles.

In order to save investments a method called "one for all" is used when hosting administrative services. One member state or the federal republic provide access to a service on their administration portal and distribute the requests to the responsible institutions "under the hood". As administration portals are connected through an "online gateway", each portal contains a search feature, which enables users to find all administrative services through any portal. Interoperable user profiles enable usage of each profile for management of administrative services on all portals.

\subsection{Basic Use Case}
The BMI lists a total of almost 600 administrative services, each being described by a different process \cite{BMI:Informatiosplattform}. Each process consists of multiple sub-processes. When comparing the processes, one can see, that some sub-processes occur very often and in similar arrangements. Those arrangements can be called the basic use case of OZG services.

The basic use case describes the submission of an application for an administrative service and consists of the following arrangement of sub-processes \cite{NRW:Umsetzung}:

\paragraph{1. Selection of Administrative Service}
The user visits any administration portal of the portal network and types a search term in the corresponding search field. The user is then presented with a list of search results consisting of administrative services. If the user selects a search result, he is forwarded to the correct page inside the domain of the administration portal which hosts the selected service. This page provides information about the administration service along with instructions on how to access the form for application. The interactive form, usually provided by a form-server, can be for example included in the page through a web-component.

\paragraph{2. Login to User Profile}
Independent of the member state the user created the interoperable user profile for, he can use it in the current administration portal. The user has to finish authentication steps, which can consist for example of a username and password. Depending on the authentication method, a trust level of the current session is determined. With approval from the user, personal information from his profile is provided towards the form-server.

\paragraph{3. Filling in Application}
Personal information from the user profile is used by the form-server to fill as many fields of the form as possible. The form with prefilled information is then presented to the user. The interactive form can then be modified by the user. Depending on the form-server, additional functionalities like conditions between fields or sanity checking can be provided.

\paragraph{4. Uploading Attachments}
The user can upload documents to his user profile and reference them in the form.

\paragraph{5. Submission of Application}
The user again authenticates for his user profile and authorises the form-server to submit the application. The form-server saves the application until he is notified to delete it. The form-server submits the application to the administration portal which hosts the administrative service.

\paragraph{6. Reception of Application by Administration Portal}
The administration portal receives the application from a form-server and usually just forwards it to the data-exchange platform and notifies the form-server that the he can delete the application.

\paragraph{7. Submission of Application to Data-Exchange Platform}
The data-exchange platform distributes the application to the correct administrative institution responsible for processing it.

\paragraph{8. Sending of Acknowledgment}
The data-exchange platform notifies the user through the inbox functionality of the user profile if the application was submitted successfully.

\subsection{System Architecture}
This subsection describes the system architecture of a member state relevant for the execution the basic use case of the OZG based on the system architecture of Nordrhein-Westfalen. Systems, which are responsible for coordination of OZG execution between member states and the federal republic are not the focus.

\subsubsection{System Components}
The system architecture consists of the following system components. Each component consists of a list of services it provides either to the user or to other components. Components are defined based on strongly connected categories of services \cite{NRW:Umsetzung}:

\paragraph{Form-Server}
The from-server provides digital forms associated to administrative services. They can be accessed by a user profile through online self-service and submitted as applications to administrative institutions. The main services are \cite{dNRW:Standardisierungskonzeptzur}

\begin{itemize}
    \item \textbf{Retrieval of a digital form}
    
    There exist many administrative services and depending on the member state or commune, the required layout of the application can be different. Therefore, a system called federal information management (Föderales Informationsmanagement: FIM) is used to standardize administrative services. FIM consists of three categories of building blocks.
    
    \textit{Service blocks} contain human readable descriptions about a administrative service.
    
    \textit{Data field blocks} contain the standardized description of data fields required for the application of administrative services.
    
    \textit{Process blocks} contain standardized descriptions about the process of an administrative service.

    Relevant for the form-server are data field blocks. A data field block can contain one element of five categories.
    
    \textit{Data fields} are the "smallest entity" of a data field block and describes one standardized piece of information. Depending on the type of information - if it is for example a checkbox or input field - additional metadata is included.
    
    \textit{Data field groups} consist of multiple data fields and other data field groups, relating to a category of information. A data field group can for example be "person" or "company".
    
    \textit{Rules} describe all kinds of logical conditions of and between data fields. This includes for example the automatic validation of the correctness of an entered value or the activation and deactivation of data fields depending on an entered value.
    
    \textit{Code lists} are lists of predefined values the user can select. This can for example be a list of all countries.
    
    \textit{Data field schemata} is the combination of entities from all previously described categories and describes the structure of a form.
    
    The building blocks are centrally managed by federal republic, member states and communes. This simplifies the creation of for example a new form by reusing existing data schema blocks and adding additional required data fields to a new data field schema.
    
    Each data field block can be uniquely identified through an ID. Therefore, if the form-server is provided with a FIM-ID, he can retrieve the corresponding data filed block from the central storage.
    
    \item \textbf{Initiation of application}
    
    The from-server can be requested to initiate an application using a certain form based on a FIM ID of a data field schemata. Along with the request, an identification and authorisation of a user profile has to be passed. Additional personal information of the user profile can be passed for automated filling in of the form.
    
    \item \textbf{Status update}
    
    The form-server sends updates regarding the current status of the application.

    \item \textbf{Automatic filling in of form}
    
    If the form server was provided with personal information during the initiation request for an application, it can automatically fill in the form.
    
    \item \textbf{Presentation of form}
    
    The form server can host a website where it displays an interactive form. The user can access this website either through an URL or through a web component.
    
    \item \textbf{Application interruption}
    
    The application can be interrupted or aborted. Interruption may occur if the user stays inactive for a certain period of time while filling in the form. The user can also manually interrupt the editing process. If the application is interrupted, the form-server stores the unfinished application for a certain amount of time. The user can then access the application at a later time.
    
    If the user decides to cancel editing the form, the application is deleted from the from-server.
    
    Depending on the actions of the user, the status of the application changes accordingly and the form server sends an status update.
    
    \item \textbf{Storage of application}
    The form-server stores applications for a specified period of time if they were interrupted or submitted.
    
    \item \textbf{Provisioning stored applications}
    
    If the form server is provided an identification and authorisation of a user, it searches all stored applications which belong to the user and sends back a list of URLs, through which the user can access them.
    
    \item \textbf{Deletion of stored application}

    If the form server is provided an identification and authorisation of a user and an identification of an application, it can delete the stored application.

    \item \textbf{Manually filling in data fields}
    
    When the form is presented to the user, it can happen, that not all information could be automatically filled in. In this case, the form can be interactively filled in by the user.
    
    \item \textbf{Uploading documents}
    
    The user can add documents as attachments to an application. The documents can either be uploaded from the local machine or the data wallet of the user profile.

    \item \textbf{Submission of application}
    
    When the user submits the application by for example pressing a "submit" button, the form-server sends a status update. The finished application can now be transmitted to a requesting system.

\end{itemize}

\paragraph{User Profile}
Each member state provides its own user profile for identity management. The main services are \cite{NRW:Umsetzung}:

\begin{itemize}
    \item \textbf{Identification and Authentification} \cite{dNRW:Anbindungsleitfaden}
    
    The user profile can be used for identification by transmitting personal information like name and address after successfully authentification. Authentification is the verification of an authentication. Users can authenticate for the user profile for example through a username / password combination.
    
    \item \textbf{Determination Trust Level} \cite{dNRW:Anbindungsleitfaden}
    
    The trust level of a logged in user profile is determined while registration and usage. The user can register and login to the user profile using a username / password combination or the German ID card. The trust level "High" is only granted, if the user registered using the German ID card \textbf{AND} logs in using the German ID card. All other combinations of registration and login result in a "normal" trust level.
    
    \item \textbf{Management Personal Information} \cite{dNRW:Anbindungsleitfaden} 
    
    Personal information of the user profile can be modified. This requires the user to authenticate. In case the user profile is connected with an online identity card, some attributes like name and age can not be changed.
    
    \item \textbf{Provisioning Personal Information} \cite{dNRW:Anbindungsleitfaden} \cite{dNRW:Schnittstellen}

    The user profile can be requested to send personal information to a specified URL. This requires the user to authenticate.
    
\end{itemize}

\paragraph{Institution}
Institutions are the entities which eventually process the incoming applications and provide the users with solutions. They receive a digital application through the data-exchange platform and send back the result as a message through the inbox.

\paragraph{Data-Exchange Platform}
The data-exchange platform delivers applications from the administration portal to the correct administrative institution.

\paragraph{Data Wallet}
The data wallet is a sub-component of the user profile and enables the user to manage documents for his user profile. The main services are:

\begin{itemize}

    \item \textbf{Upload of document}
    
    The user can upload a document from his local machine to the data wallet.
    
    \item \textbf{Show documents}
    
    The user can list all documents which are stored in the data wallet.
    
    \item \textbf{Delete documents}
    
    The user can delete documents from the data wallet.
    
    \item \textbf{Attach documents to application}
    
    The user can attach a copy of a document stored in his data wallet to an application on a from server.

\end{itemize}

\paragraph{Inbox}
The inbox is a sub-component of the user profile and enables the user to send messages through his user profile. The main services are:
    
\begin{itemize}
    
    \item \textbf{Send a message}
    
    The user can send a message to authorities. This can be for example an institution which processes an application.
    
    \item \textbf{Receive a message}
    
    The user can receive messages from authorities for example to update the status of an application.
    
    \item \textbf{List messages}
    
    The user can see the messages he received by logging in to his user profile on the administration portal.
    
    \item \textbf{Notification through E-Mail}
    
    The user can be notified of a new message through E-Mail.

\end{itemize}


\paragraph{Administration Portal}
The administration portal provides the user access to various OZG services of other system components:

\begin{itemize}

    \item \textbf{Application for administrative service}
    The administration portal provides access to services for creation and submission of applications for administration services. The portal hosts a web page for each administration service which enables the user to get information about the service and start the application process. After starting the process, the portal interacts with multiple system components to guide the user through the application.
    
    \item \textbf{Management of user profile}
    The administration portal provides access to the personal information stored in the user profile by interacting with the user profile system component. It enables the user to add, remove and update a set of attributes.
    
    \item \textbf{Management of inbox}
    The administration portal provides access to the inbox of the user profile. It enables the user to send new messages and display received messages.
    
    \item \textbf{Management of data wallet}
    The administration portal provides access to the data wallet of the user profile. It enables the user to upload and download documents and to attach copies of documents to applications.
    
\end{itemize}

The administration portal provides additional services:

\begin{itemize}
    \item \textbf{Determination of responsible institution}
    
    Administrative services can be processed by a variety of administrative institutions, depending on the content of the application or the sender. It is the responsibility of the administration portal to determine which institution receives which application.
    
    \item \textbf{Search and find Administrative Service}
    
    The administration portal provides a search and find service which enables the user to find the web page of all administrative services which relate to the entered search term.

\end{itemize}

\subsubsection{Interfaces}
In order to implement the basic use case, the components have to interact with each. The following diagram visualizes, which components interact. Each interaction is described afterwards. The arrows describe the main direction of information flow in an interaction.

\begin{center}
    \includegraphics[width=8cm]{Diagrams/Interaction Diagram.png}
\end{center}

\begin{enumerate}

    \item The web browser of the user displays the web pages hosted by the administration portal.

    \item The form-server can initiate the application if requested by the administration portal. The portal submits information for identification, authentication and authorisation of a user through an JSON Web Token and a user profile ID. In order for the form server to determine the correct form to display, the portal submits a FIM-ID which corresponds to a data field schemata. In case the user approved, the portal can also collect personal information from the user profile and submit it to the form server for automated filling in of the form.
    
    \item The form server provides the user a web page which displays an interactive form.
    
    \item The form server notifies the administration portal of changes in the status of applications. If the administration portal is notified of a submitted application, it requests the form server to transmit the application data.
    
    \item The administration portal requests the user profile for identification and authentication to login a user on the web page of the portal. The portal also requests authorisation from the user profile to for example access certain personal information.
    
    \item The administration portal enables the user to manage services of the user profile. It provides OSS tools for managing attributes stored in the user profile, to send new messages and to manage documents in the data wallet.
    
    \item The administration platform submits application data to the data exchange platform for delivery to the responsible institution.
    
    \item The data exchange platform notifies the administration portal of the delivery status of the application data.
    
    \item The data exchange platform notifies the institution of a new application and enables it to download the corresponding data.
    
    \item The institution can send messages to the inbox of a user profile to for example request additional documents.
    
    \item The user can send messages to the institution which for example asked for additional documents.
    
\end{enumerate}
    
\subsubsection{Data Objects}
This section describes what data each component processes. The property value describes a certain type of data. The source describes which component stores the value. This distinguishes the access of a property by reference and by copy. If the source of a property is another component, the property is accessed by reference. The description gives further detail on how the property is processed by the component.

\begin{table}[!h]
    \begin{tabularx}{\textwidth}{|l|l|l|C|}
    \rowcolor{LightCyan}
    \hline
    \multicolumn{4}{|l|}{Administration Portal: Service Web Page} \\
    \hline
    Property & Storage & Origin & Description  \\
    \hline
    \hline
    Name & Web Page & FIM & Name of an administrative service \\
    \hline
    URL & Portal & Portal & The URL of the Web Page \\
    \hline
    Description & Web Page & FIM & Description of an administrative service \\
    \hline
    FIM-ID & Web Page & FIM & FIM-ID of a data field schemata which describes attributes required from the user when applying for the service \\
    \hline
    \end{tabularx}
\end{table}

\begin{table}[!h]
    \begin{tabularx}{\textwidth}{|l|l|l|C|}
    \rowcolor{LightCyan}
    \hline
    \multicolumn{4}{|l|}{Administration Portal: User} \\
    \hline
    Property & Storage & Origin & Description  \\
    \hline
    \hline
    Attributes & User Profile & User Profile & The portal can manage attributes of user profiles \\
    \hline
    JWT & Portal & User Profile & The portal can retrieve a JSON Web Token from the profile after successful authentication by a user \\
    \hline
    URL Form & Portal & Form Server & The portal stores the URL where the form server hosts the interactive form \\
    \hline
    \end{tabularx}
\end{table}

\begin{table}[!h]
    \begin{tabularx}{\textwidth}{|l|l|l|C|}
    \rowcolor{LightCyan}
    \hline
    \multicolumn{4}{|l|}{Administration Portal: Application} \\
    \hline
    Property & Storage & Origin & Description  \\
    \hline
    \hline
    FMS-ID & Portal & Form Server & The portal receives the ID of applications from the form server \\
    \hline
    Status & Portal & Form Server & The portal receives status updates for each application \\
    \hline
    XML Form Data & Portal & Form Server & The portal can retrieve the data which was filled into a form \\
    \hline
    \end{tabularx}
\end{table}

\begin{table}[!h]
    \begin{tabularx}{\textwidth}{|l|l|l|C|}
    \rowcolor{LightCyan}
    \hline
    \multicolumn{4}{|l|}{User Profile: Personal Information} \\
    \hline
    Property & Storage & Origin & Description  \\
    \hline
    \hline
    Attributes & User Profile & FIM & The list of attributes the user can fill in at his profile are based on FIM data fields managed in the FIM repository. \\
    \hline
    Attribute Values & User Profile & User & The user can manually add values to a list of predetermined attributes in his profile \\
    \hline
    Credentials & User Profile & User & The User Profile stores encrypted credentials of the user to authenticate him \\
    \hline
    User Profile ID & User Profile & User Profile & The User Profile creates and stores a unique ID for the user \\
    \hline
    \end{tabularx}
\end{table}

\begin{table}[!h]
    \begin{tabularx}{\textwidth}{|l|l|l|C|}
    \rowcolor{LightCyan}
    \hline
    \multicolumn{4}{|l|}{User Profile: Authorisatiozation} \\
    \hline
    Property & Storage & Origin & Description  \\
    \hline
    \hline
    JWT & Other Component & User Profile & The user profile can create and transfer JSON Web Tokens which authorise other components to access resources of the user profile \\
    \hline
    \end{tabularx}
\end{table}

\begin{table}[!h]
    \begin{tabularx}{\textwidth}{|l|l|l|C|}
    \rowcolor{LightCyan}
    \hline
    \multicolumn{4}{|l|}{User Profile: Inbox} \\
    \hline
    Property & Storage & Origin & Description  \\
    \hline
    \hline
    Received Messages & User Profile & User & The user profile stores messages which were addressed to the user \\
    \hline
    Sent Messages & User Profile & User & The user profile stores messages which were sent by the user \\
    \hline
    \end{tabularx}
\end{table}

\begin{table}[!h]
    \begin{tabularx}{\textwidth}{|l|l|l|C|}
    \rowcolor{LightCyan}
    \hline
    \multicolumn{4}{|l|}{User Profile: Data Wallet} \\
    \hline
    Property & Storage & Origin & Description  \\
    \hline
    \hline
    Documents & User Profile & User & The user profile stores documents which were uploaded by the user \\
    \hline
    \end{tabularx}
\end{table}

\begin{table}[!h]
    \begin{tabularx}{\textwidth}{|l|l|l|C|}
    \rowcolor{LightCyan}
    \hline
    \multicolumn{4}{|l|}{Form Server} \\
    \hline
    Property & Storage & Origin & Description  \\
    \hline
    \hline
    FMS-ID & Form Server & Form Server & The form server creates a unique ID for each application \\
    \hline
    JWT & Form Server & User Profile & The form server uses a JSON Web Token to verify that the entity issuing the request is authorised \\
    \hline
    User Profile ID & Form Server & User Profile & The ID of the user who is associated to the application \\
    \hline
    FIM-ID & Form Server & FIM & The FIM ID of a data field schemata is used to determine how to construct the form \\
    \hline
    Form & Form Server & FIM & The list of attributes the user can fill in at the form are based on the FIM data field schemata. The form server maps the FIM data fields of the FIM data field schemata to attributes the form server understands \\
    \hline
    Status & Form Server & Form Server & The status describes which actions were performed on the application \\
    \hline
    Application Data & Form Server & User & The form server stores the data which the user filled in to the form as an XML file \\
    \hline
    Web Page & Form Server & Form Server & The form server hosts a web page with the interactive form  \\
    \hline 
    Web Page URL & Form Server & Form Server & The form server hosts the web page of the form at an individual URL \\
    \hline 
    \end{tabularx}
\end{table}

\subsubsection{Sequence Diagram}

\begin{figure}[!h]
    \centering
    \includegraphics[width=17cm]{Diagrams/Basic Use Case Sequence Diagram.png}
\end{figure}


\chapter{Identity Management}

\section{Functional Requirement Analysis}

\subsection{User Perspective}

\begin{itemize}
    \item The user wants to manage his personal information in one place
    \begin{itemize}
        \item The user wants to CRUD attributes
        \item The user wants SPs to CRUD his attributes
        \item The user wants a user friendly interface where he can CRUD his attributes
        \item In case the user cannot understand an attribute or its possible values, he needs OSS tools which enables him to enter the correct value
        \item The user wants to enter personal information only once (=> SPs should reuse existing attributes)
        \item The user wants a new or modified attributes to be instantly used everywhere
        \item The user wants a deleted attribute to be instantly deleted everywhere
        \item The user wants to have a list of often used attributes, in order to prefill them
    \end{itemize}
    \item The user wants to manage authentication in one place
    \begin{itemize}
        \item The user does not want to authenticate
        \item The user wants that only he has access to the identity
    \end{itemize}
    \item The user wants to manage authorisation in one place
    \begin{itemize}
        \item The user wants to manage who can perform which actions in his name (f.e CRUD attributes, place order, cancel subscription)
        \item The user wants an authorisation to only be granted if he manually approves
        \item The user wants an authorisation to be associated with a single entity, a reason, which actions can be performed and duration of the authorisation
        \item The user wants to communicate with the authorised entity
        \item The user wants to have much information about the authorised entity
        \item The user wants to be able to revoke an authorisation
        \item The user wants to save the history of authorisations
    \end{itemize}
    \item The user wants to use all OSS tools in one place
    \begin{itemize}
        \item Communication through inbox
        \item Subscribe for a service of an SP
        \item Place order in a shop of an SP
        \item ...
    \end{itemize}
    \item The user wants

\end{itemize}

\subsection{Service Provider Perspective}

\begin{itemize}
    \item SPs want to manage personal information of users
    \begin{itemize}
        \item SPs want to CRUD attributes of the user
        \item In case modification of an attribute has strong implications on the operation of a system, the modification of an attribute needs to be limited
        \item SPs want to access personal information already collected by other SPs
        \item SPs want 
    \end{itemize}
    \item identification
    \item authentication
    \item authorisation
    \item 
\end{itemize}

\subsection{OZG Perspective}

\section{IMP Solution Proposal}

This section describes a solution for an IMP system considering the previously mentioned requirements.

\subsection{Components}

\subsubsection{IMP Client}

The IMP client is a smartphone application which enables the user to manage his IMP identity. The client provides the following functionalities:
\begin{itemize}
    \item Create an identity
    \item Manage attributes of the identity
    \item Manage documents
    \item Send and receive messages
    \item Send and receive relationship requests
    \item Send and receive requests
\end{itemize}

\subsubsection{IMP Server}
The IMP server interacts with the clients in order to deliver information from one client to the other. The server has the following functionalities:
\begin{itemize}
    \item Create and store relationship template
    \item Create and store request template
    \item Deliver relationship requests
    \item Deliver requests
    \item Deliver messages
\end{itemize}

\subsection{Functionalities}

\subsubsection{Relationships}

\subsubsection{Requests}

\subsection{Data Objects}

\begin{table}[!h]
    \begin{tabularx}{\textwidth}{|l|l|l|C|}
    \rowcolor{LightCyan}
    \hline
    \multicolumn{4}{|l|}{Form Server} \\
    \hline
    Property & Storage & Origin & Description  \\
    \hline
    \hline
    \hline
    \end{tabularx}
\end{table}

\chapter{IMP Integration Architecture}

\section{Requirement Analysis}

\subsection{Integration Challenges}

The following challenges are a result of the functionalities an IMP system provides in addition to conventional identity management systems. The origin of the challenges is the interoperability of the IMP identity with multiple SPs. It is the purpose of the integration architecture to solve as many of these challenges as possible.

\subsection{Basic Use Case}

\subsection{Extensions}

\section{Integration Proposal}

\subsection{IMP Connector}

\subsection{Messaging}

\section{Integration}

Multiple integration scenarios are presented with increasing complexity and functionality.

\subsection{Scenario 1}

\begin{figure}[h]
\caption{Overview Integration Scenario 1}
    \centering
    \includegraphics[scale=0.2]{Diagrams/Integration 1/Overview.png}
\end{figure}

In this scenario, the IMP system integrates only with the administration portal to extend the usage of identity related functionalities through the IMP client. The integration architecture contains one connector which integrates with the administration portal and communicates with IMP server and IMP client.

Through the integration, the user will be able to connect his OZG user profile with an IMP identity. As a result of this connection, the personal information of the user profile will be synchronized with the attributes of the IMP identity. The user will only be able to change personal information of his user profile through the IMP client. If the user has connected an IMP identity, starting an application will require authorization through the IMP client. However, starting and filling in the application will still be done through the web page of the administration portal and form server. After the user submitted the application through the form server, the user will be able to manage active applications through the IMP client.
The user will be able to send and receive messages of the user profile inbox through the inbox of the IMP client and combine the data wallet of the IMP client and user profile.

The integration provides an improved experience of the basic OZG use case through the IMP systems by making existing functionalities of the administration portal available through the IMP client. Especially the fact that the user wont have to manage the OZG user profile separately is an advantage for the user. Communicating with governmental institutions through the smartphone instead of a web page is also more convenient.

This integration scenario can be a first step by service providers to adopt an IMP system.

\paragraph{Linking IMP ID and OZG ID}

Before the IMP client can access services of the administration portal, the user has to be able to connect his IMP identity with the user profile. This is necessary for the integration architecture to enable bidirectional communication between OZG systems and IMP client regarding a user.

The administration portal hosts an additional web page where a logged in user can connect one IMP identity to his user profile. The IMP identity can be connected with the user profile through an IMP relationship between the administration portal and the IMP identity. The relationship includes the ID of the user profile as metadata in order for the relationship to be associated to a user profile. For a relationship to be established, a relationship template has to be created. This template is created by the IMP connector and stored on the IMP server. The template can be referenced through a template ID, created by the IMP server. The relationship template contains information about the administration portal and its connector, the reason for the relationship, attributes the IMP client is required to share as part of the relationship and the ID of the user profile. Based on the template ID, the connector creates a QR code which gets displayed on the web page for any IMP client to scan, retrieve the relationship template and issue a relationship request. The request contains information about the IMP identity and values for the requested attributes. The user gets notified of the connection request through an E-Mail which includes an URL. Visiting this URL will confirm the relationship request.

After confirmation, the IMP identity is connected with the administration portal through an IMP relationship which contains the ID of a user profile. The administration portal stores the IMP identity ID, the relationship ID and the user profile ID as an entry in a database.

The relationship can be cancelled through the IMP client or through the web page of the administration portal.

\begin{figure}[h]
\caption{Overview Connection}
    \centering
    \includegraphics[scale=0.3]{Diagrams/Integration 1/Connection/Overview.png}
\end{figure}

The web server of the administration portal hosts an additional web page where a user can connect an IMP identity with his user profile. When a user tries to visit the web page, the browser sends a GET request to the server and attaches a session ID as cookie. Based on the session ID, the web server can determine the user ID and the authentication state of the session. If the user is not authenticated, the server responds with a redirection to the login page. If the user is authenticated, the web server requests a QR code from the messaging system for connecting an IMP identity to the user ID of the session.

A messaging gateway is integrated into the run-time of the web server. It provides an API for the server code to send and receive messages while hiding most complexities of messaging. After the authentication of the session is validated, the web server calls the API of the gateway to request the QR code for connecting an IMP identity to the user ID which corresponds to the session. The messaging gateway places a message on the "Connection QR Request" channel. The message contains the ID of the user (UID). Based on the request-reply pattern, the messaging gateway will receive a reply through the "Connection QR Reply" channel. In order for the gateway to match replies to requests, request and reply message contain a request ID (RID) with the same value.






As the name suggests, the relationship template is the basis for creating a relationship request. One template can therefore be used multiple times. Therefore, in order to save resources, QR codes for connecting IMP identities and user profiles should be reused. However, it still has to be possible to use new relationship templates, as the required content of future relationships might change. It could for example happen, that additional attributes are required to be shared by the IMP identity or that a user agreement attached to the relationship metadata changed.

The messaging architecture contains a QR code management system which is responsible for storing all QR codes and the corresponding user IDs. Messages sent to the "Connection QR Request" channel are consumed by a content-based router who checks if the QR code database contains an entry for the UID of the message. If this is the case, the message is sent through a content enricher and a content filter where the QR code is added, the UID is removed and the message is placed on the "Connection QR Reply" channel. Content-based router and content enricher have access to the QR code database but do not necessarily communicate through messaging.





If the content-based router does not find a QR code for the UID in the database, the message is sent to a different content enricher which adds a configuration to the message. This configuration contains information for the connector on how to create the relationship template. The message is then sent to the "Create Relationship Template" channel, where the connector is able to read it. The connector uses the configuration stored in the message to create the relationship template and store it on the IMP server. The connector exports the corresponding template ID as a QR code and publishes a message containing the user ID, the request ID and the QR code on the "Publish New Relationship QR" channel. Any application can subscribe to this channel and process the contained messages. In this case, two content filters are subscribed. One of the content filters was already described earlier where a QR code for the user ID was already stored in the database. This content filter does the same as described previously, it removes the user ID from the message and sends it to the "Connection QR Reply" channel.
The other content filter prepares the message to be added to the QR code database by removing the Response ID. Through a messaging gateway, the database is able to receive the message and store the QR code.

The messages on the "Connection QR Reply" channel are read by the messaging gateway of the web server and received by the code of the server through a return statement of an API function call. The QR code is now added to the web page and sent back to the web browser of the user.

































\begin{figure}[h]
\caption{Overview Connection Messaging}
    \centering
    \includegraphics[scale=0.3]{Diagrams/Integration 1/Connection/Messaging Overview.png}
\end{figure}

The newly created web page uses messages to communicate with the administration portal using a request-reply pattern. The request-reply pattern enables the requestor - the web page in this case - to send a message to the replier - the connector in this case - and receive an answer which correlates to the request. The purpose of the communication in this case is for the web page to receive the QR code containing the relationship template.

\begin{figure}[h]
\caption{Web Page Messaging Adapter}
    \centering
    \includegraphics[scale=0.3]{Diagrams/Integration 1/Connection/Messaging 1.png}
\end{figure}

In order for the web page to send and receive messages, it includes a messaging gateway. As the web page is added as part of the integration architecture, the gateway can be included in the web page. It provides domain specific functions to the web page and hides much of the complexity of interacting with the messaging system. In combination with the request-reply pattern, the local functions provided by the gateway can be used synchronously and asynchronously.
The web page requires the user to be logged in to his user profile and therefore expects the web browser to contain a JSON Web Token. The web page uses the messaging gateway to send a message, requesting a QR code for connecting an IMP identity. The message consists of a JSON Web Token and is placed in a datatype channel called "QR Request Channel". 

\begin{figure}[h]
\caption{Request Reply Smart Proxy}
    \centering
    \includegraphics[scale=0.3]{Diagrams/Integration 1/Connection/Messaging 2.png}
\end{figure}

In order to implement a request-reply pattern, a stateful smart proxy is used. The web page sends a message which contains a JSON Web Token in the body to the "QR Request" channel. Besides the body, the message contains a message ID in the header which uniquely identifies it. The message also contains a return address which specifies a channel to which the reply should be sent. The messaging gateway specifies a reply channel unique to the web browser instance. The temporary channel is deleted eventually. The smart proxy stores message ID and reply channel in order to correlate replies in the "QR Reply Service" channel with the request and send it to the specified reply channel. If eventually, the integration architecture is ready to send a QR code as reply, the corresponding message can be put on the "QR Reply Service" channel, where, based on the message ID, the smart proxy can correlate it to the request message and send the reply to the corresponding reply channel. The messaging gateway included in the web browser can retrieve the reply from its reply channel in multiple ways. One example would be iterative polling.

\begin{figure}[h]
\caption{Filtering and Routing Request}
    \centering
    \includegraphics[scale=0.3]{Diagrams/Integration 1/Connection/Messaging 3.png}
\end{figure}

The message from the "QR Request Channel" are processed by a filter. All messages which are invalid through for example an expired JSON Web Token are deleted. Valid messages are then routed by a stateful content-based router. The router keeps track from which users a message was received. In case a message was received previously, the current message is routed to the "Existing Request" channel. Otherwise the message is routed to the "New Request Template" channel.

\begin{figure}[h]
\caption{Retrieving Existing QR Code}
    \centering
    \includegraphics[scale=0.3]{Diagrams/Integration 1/Connection/Messaging 4.png}
\end{figure}

The messages from the "Existing Request Template" channel are consumed by a QR code management system, which stores all valid QR codes in a database. The system adds the QR code corresponding to the user to the message and sends it to the "QR Reply Service" channel, where it is delivered to the web page by the smart proxy.

\begin{figure}[h]
\caption{Requesting New Template}
    \centering
    \includegraphics[scale=0.3]{Diagrams/Integration 1/Connection/Messaging 5.png}
\end{figure}

Messages on the "New Relationship Template" channel are modified by a content enricher to contain parameters which are required by the connector to create a relationship template. The message is then send to the connector using a request-reply pattern.

\begin{figure}[h]
\caption{Creating New Template}
    \centering
    \includegraphics[scale=0.3]{Diagrams/Integration 1/Connection/Messaging 6.png}
\end{figure}

As there is only one possible channel to receive the reply from the connector, no smart proxy is required for this request-reply pattern. The message is send to the "Relationship Template Request" channel, trough which the portal connector receives it. As the connector is designed with integration in mind, he is capable of receiving and sending messages through an included messaging gateway. The connector also contains a sophisticated process manager for implementing the provided IMP functionalities through interaction with an IMP server. In this case, the process manager interacts with the IMP server to create a relationship template, store it on the IMP server, retrieve a template ID and export it as a QR code. Through a content enricher, the QR code is attached to the message and placed on the "Relationship Template Reply" channel.

\begin{figure}[h]
\caption{Storing New Template}
    \centering
    \includegraphics[scale=0.3]{Diagrams/Integration 1/Connection/Messaging 7.png}
\end{figure}

The message from the "Relationship Template Reply" channel is consumed by the QR code management system. The system stores the new QR code for the user and sends the message unchanged to the "QR Reply Service" channel, where the smart proxy delivers it to the web page.

\paragraph{Authorisation}

If the user successfully authenticated for a user profile, the administration portal receives a JSON Web Token which enables it to perform actions for the user. The user might want to manage which actions an IMP client is authorised to perform without having to approve each of them trough the website of the portal.

\begin{center}
    \includegraphics[scale=0.3]{Diagrams/Integration 1/Authorisation/Overview.png}
\end{center}

After successful connection of an IMP identity with the user profile through the introduced web page, the website can display a user interface for specifying which actions the IMP identity is authorized for. After the user finishes, the web page submits the configuration to the administration portal. The portal then creates a suitable authorisation request to the user profile which requires the token to include the relationship ID associated to the user profile. After successful authentication by the user, the user profile transmits a JSON Web Token to the portal. 

It contains the user profile ID, the relationship ID and information about the authorisations that are granted when attaching this tokens to a request. It also includes a signature with the private key of the user profile component. This enables components to verify if the token is valid by transmitting it to the user profile component.

The JSON Web Token is stored as attribute of the IMP identity. Upon receiving the token form the user profile component, the administration portal initiates two requests through the connector to the IMP identity of the relation corresponding to the relation ID. The first requests to add the token as attribute. The second requests to share the attribute.

If the user approves both requests, every time the connector requests the portal to perform an action coming from a relationship, the portal can use the JSON Web Token shared through this relation.

\paragraph{Personal Information}

One of the goals of an IMP system is to enable the user to only manage his identity in one place while all other service providers rely on the attributes of the identity. As in this scenario however, the OZG system does not rely on attributes of the IMP system but enables it to manage personal information stored in the user profile. Some synchronisation mechanism can therefore operate.

\begin{center}
    \includegraphics[scale=0.3]{Diagrams/Integration 1/Personal Information/Overview.png}
\end{center}

While configuring the authorisation of the IMP identity on the previously described website, the user can determine if the IMP identity is allowed to manage personal information stored in the user profile. If the user approves, the JSON web Token stored as attribute of the identity is authorized to update, delete and read attributes of the user profile. As the user profile determines which attributes exist, no attributes can be added.

If an attribute of an identity changes, the IMP client notifies all relationships. When the connector receives this notification, he reads all shared attributes and requests the administration portal to update the same attributes in the user profile. As the naming and meaning of attributes can differentiate between IMP systems and OZG systems, the attributes have to be mapped to each other. A simple solution would be a table, created by an administrator, which maps each attribute from the IMP system to an attribute of the OZG system.

\paragraph{Inbox}

Both IMP Client and the administration portal provide an inbox functionality. If an IMP identity has established a relationship with the administration portal, they can exchange messages through the connector as part of this relationship.

In order to notify the user of a new message to the user profile, the administration portal can send the same message through the relationship associated with this user.
The user can, however, only send messages to the administration portal and not to for example institutions.

\paragraph{Data Wallet}

\subsubsection{Messaging Architecture}

\subsection{Integration 2}

Option for user to use IMP identity INSTEAD of user profile

\section{Extension: Inbox}

\section{Extension: Collaboration}

\chapter{Solution Evaluation}

\section{Conclusion}

\chapter{Outlook: Advanced IMP Integration}

\chapter{Tables}

        \begin{table}[!h]
            \begin{tabularx}{\textwidth}{|l|C|l|}
            \hline
            Property & Description & Data Source \\
            \hline
            \rowcolor{LightCyan}
            \multicolumn{3}{|l|}{Input} \\
            \hline
            
            JSON Web Token & The portal received this JWT as a result of the login process with the user profile component. This token enables the form-server to verify the authentication of the user. & User Profile \\
            
            \hline
            
            User Profile ID & This ID is a unique identifier of a user profile and is used by the form server to associate the initiated application with a user profile. & User Profile \\
            
            \hline
            
            Personal Data & If the user approves, the administration portal submits personal data, the form-server can use to automatically fill in the form. & User Profile \\
            
            \hline
            
            FIM Schema ID & The administration portal stores a hard coded FIM-ID for every administration service. This FIM-ID corresponds to the correct data field schemata, the form-server can display & Hard Coded in Portal \\
            
            \hline
            \rowcolor{LightCyan}
            \multicolumn{3}{|l|}{Output} \\
            \hline
            
            Status & The form-server answers with the status of the initiated application. & From Server \\
            
            \hline
            
            FMS-ID & While initializing the application, the form-server creates an ID, which uniquely identifies the application. This FMS-ID is sent back to the portal so that it can request information about the application at a later time. & Form Server \\
            
            \hline
            \end{tabularx}
            \caption{Interface of the form-server for initialization of an application}
            \label{table:interface_form_initialization}
        \end{table}

        \begin{table}[!h]
            \begin{tabularx}{\textwidth}{|l|C|l|}
            \hline
            Property & Description & Data Source \\
            \hline
            \rowcolor{LightCyan}
            \multicolumn{3}{|l|}{Input} \\
            \hline
            
            FMS-ID & The FMS-ID was created by the form-server during the initialisation of the application and sent to the administration portal. Each status update includes this ID in order for the portal to associate the update with an application. & Form Server \\
            
            \hline
            
            Status & The status provides information about interactions of the user or the form-server with the application. The possible states are:
            
            \begin{itemize}
                \item \textbf{null}: This application has this status right after initialisation.
                \item \textbf{submitted}: The application has this status after the user triggered the submit functionality of the form-server. Usually the form is now locked for further modification by the user.
                \item \textbf{filed}: The form-server submitted the application to the portal but did not yet receive a confirmation.
                \item \textbf{confirmed}: The form-server received a confirmation from the portal.
                \item \textbf{deleted}: The form-server deleted the application.
            \end{itemize}
            
            & Form Server \\
            \hline
            \end{tabularx}
            \caption{Interface of the administration portal for receiving status updates by the form-server}
            \label{table:interface_status}
        \end{table}

        \begin{table}[!h]
            \begin{tabularx}{\textwidth}{|l|C|l|}
            \hline
            Property & Description & Data Source \\
            \hline
            \rowcolor{LightCyan}
            \multicolumn{3}{|l|}{Input} \\
            \hline
            
            FMS-ID & The FMS-ID was created by the form-server during the initialisation of the application and sent to the administration portal along with the status update of the submission. The administration portal sends this ID in order for the form-server to know which application data to transfer. & Form Server \\
            
            \hline
            \rowcolor{LightCyan}
            \multicolumn{3}{|l|}{Output} \\
            \hline
            
            FMS-ID & The FMS-ID is transmitted in order for the portal to associate the response with an application. & Form Server \\
            
            \hline
            
            JSON Web Token & The JWT which was initially sent from the portal to the form-server is now transmitted back again, in order for the portal to verify, that the form-server is authorized to submit an application for the user. & User Profile \\
            
            \hline
            
            User Profile ID & The user profile ID was initially transmitted to the form-server by the portal and is now transmitted back in order for the portal to associate the application with a user profile. & User Profile \\
            
            \hline
            
            XML File & The XML file contains the personal data retrieved through the form. & Form Server \\
            
            \hline
            \end{tabularx}
            \caption{Interface of the form-server for requesting application data by the administration portal.}
            \label{table:interface_application_data}
        \end{table}

\printbibliography


\end{document}
